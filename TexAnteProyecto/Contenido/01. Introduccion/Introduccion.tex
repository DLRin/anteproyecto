\section*{Introducción}
\addcontentsline{toc}{section}{Introducción}
En la ciudad de Bogotá, una gran parte de la población enfrenta un desafío crítico, el cual es la falta de conociientos financieros básicos que limitan su capacidad de administrar adecuadamente sus recursos económicos. Esta carencia en la alfabetización financiera afecta principalmente a sectores como los jóvenes, los trabajadores informales y personas que no tienen experiencia previa en finanzas, quienes enfrentan retos para elaborar presupuestos, ahorrar e invertir de forma segura. El desconocimiento de estos conceptos financieros básicos da problemas tales como el sobreendeudamiento, el uso de créditos informales y poca inclusión dentro del sector financiero formal.

Bajo este contexto, se propone un plan de negocios para un servicio de planificación financiera que busque orientar de carácter profesional, peronalizada y accesible a personas sin conocimientos financieros en la ciudad de Bogotá. Este servicios busca superar barreras tradicionales mediante el uso de herramientas digitales didácticales, como simuladores y calculadoras financieras, integrando la educación financiera práctica con la asesoría directa. La iniciativa no solo pretende facilitar la gestión financiera práctica con asesoría directa, sino también promover hábitos saludables del ahorro, la inversión y el control de presupuestos, contribuyendo a mejorar la estabilidad económica y el bienestar de los usuarios.

Esta propuesta se fundamenta en la necesidad urgente de cerrar la brecha de alfabetización financiera identificada en Bogotá y capitaliza el avance de tecnologías fintech para ofrecer soluciones innovadores que se adapten a la realidad y particularidades de los grupos vulnerables. De esta forma, el plan de negocios introduce una estrategia integral que combina la eduación, la tecnología y el asesoramiento personalizado con el objetivo de facilitar la inclusión financiera y fomentar un cambio positivo en la cultura económica de la población local.