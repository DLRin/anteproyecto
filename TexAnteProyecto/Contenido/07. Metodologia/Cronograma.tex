\subsection*{Cronograma}
\addcontentsline{toc}{subsection}{Cronograma}

Dentro del siguiente apartado se elaboró un cronograma detallado en el que se describen las actividades para el planteamiento y la implementación del plan de negocios dependiendo de cada fase de la metodología. Este cronograma sirve como una guía útil para planificar y ejecutar las actividades necesarias para el desarrollo del modelo de negocio. Durante cada fase, desde la presentación y el compromiso del equipo hacia la estructuración final del modelo, se programaron para ajustarse a plazos necesarios y llevarse a cabo de forma eficiente. Este enfoque metódico y organizado garantiza un progreso constante y una gestión efectiva del proyecto en su totalidad.

\begin{figure}[H]
	\centering
	\includegraphics[width=12cm]{Imagenes/CronogramaModelo.png}
	\caption[{Cronograma de actividades.}]{\centering Cronograma de actividades. \textit{Fuente:} Elaboración propia.}
	\label{fig:modelocanvas}
\end{figure}

Cada fase que se aplicará desde esta metodología proporciona un enfoque temático y estructurado para desarrollar el modelo de negocios, reduciendo el riesgo de errores y aumento las posibilidades de éxito a largo plazo. Las fases son:

\begin{itemize}
	\item \textbf{Fase 1, Presentación y compromiso del equipo:} Durante esta fase inicial, se reúne al equipo de trabajo clave que hará parte de la creación y desarrollo del modelo de negocio. Siendo importante que absolutamente todos los miembros del equipo entiendan la importancia del proyecto y estén comprometidos con los objetivos y estén dispuestos a colaborar de forma activa durante el proceso. Logrando una base sólida para trabajar conjuntamente y garantizando una alineación de los objetivos y experiencias desde un inicio.
	
	\item \textbf{Fase 2, Análisis de la situación:} En este apartado, el equipo realiza un análisis exhaustivo acerca del entorno empresarial dentro del cual opera la organización, esto involucrá estudiar a la competencia, identificar las tendencias del mercado, evaluar oportunidades y amenazas, así como averiguiar sobre los recursos internos y las capacidades que tiene la empresa. Este análisis logra una comprensión mucho más profunda acerca del contexto en el que se desarrollá la empresa y ayuda a establecer un inicio para tomar decisiones estratégicas informadas.
	
	\item \textbf{Fase 3, Definición de la empresa:} Durante esta fase, el equipo define su identidad y la visión de la empresa. Acá se establece los valores fundamentales, la misión y la propuesta de valor que hará de diferenciará nuestra empresa con respecto al resto del mercado. Además, se hallarán los segmentos de clientes clave a los que se dirigirá la empresa, desarrollando una compresión acerca de su potencial cliente. Ayudando a enfocar los esfuerzos de esta empresa en áreas prioritarias y alienando sus actividades con los objetivos estratégicos.
	
	\item \textbf{Fase 4, Estructuración:} En esta última etapa, se estructura el modelo de negocio en sí mismo, usando herramientas tales como el Business Model Canvas (Modelo CANVAS), que describimos anteriormente, se detallan componentes clave del modelo de negocio, como lo son los segmentos de los clientes, la propuesta del valor, los canales de distribución, las fuentes de ingresos, etc. Esta fase implica una mayor planificación detallada e iteración continua para refinar y actualizar y mejorar el modelo de negocio antes de su implementación.
\end{itemize}