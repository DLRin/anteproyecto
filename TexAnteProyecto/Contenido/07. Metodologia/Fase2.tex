\subsection*{Fase 2: Análisis de la situación}
\addcontentsline{toc}{subsection}{Fase 2: Análisis de la situación}

\begin{itemize}
	\item Identificación de la oportunidad de negocio, la definición del producto o servicio.
	
	\item Estudio del mercado.
	
	\item Análisis de riesgos.
\end{itemize}

Estos objetivos se definen en la fase relacionada para la evaluación de la situación, anticipando de esta forma los resultados deseados por medio de la ejecución de actividades específicas, como muestra la siguiente tabla:

\begin{adjustbox}{
		center,
		caption=[{Descripción de la fase 2.}]{\centering Descripción de la fase 2. \textit{Fuente:} Autores.},
		label={TablaFase2},
		nofloat=table, vspace={20px}}
	\resizebox{\textwidth}{!}{
		\begin{tabular}{|c|c|c|c|}
			\hline
			\rowcolor[HTML]{D9EAD3} 
			\textbf{Fase 2} &
			\textbf{Objetivos Especificos} &
			\textbf{Actividades} &
			\textbf{Resultados Esperados} \\ \hline
			&
			&
			\begin{tabular}[c]{@{}c@{}}Definir el funcionamiento del\\ mercado de contratacion de personal IT\end{tabular} &
			\begin{tabular}[c]{@{}c@{}}Definición de producto\\  y posibles riesgos\end{tabular} \\ \cline{3-4} 
			&
			&
			Establecer las asociaciones claves &
			Modelo de negocio \\ \cline{3-4} 
			&
			&
			&
			Propuesta de valor \\ \cline{4-4} 
			\multirow{-5}{*}{\begin{tabular}[c]{@{}c@{}}Análisis de la \\ situación actual\end{tabular}} &
			\multirow{-5}{*}{\begin{tabular}[c]{@{}c@{}}Realizar un análisis de mercado para\\ definir la factibilidad de creación \\ del producto para la gestion \\ de contratacion de personal IT\end{tabular}} &
			\multirow{-2}{*}{\begin{tabular}[c]{@{}c@{}}Efectuar la estructura \\ de costos\end{tabular}} &
			Estudio de mercado \\ \hline
		\end{tabular}
	}
\end{adjustbox}