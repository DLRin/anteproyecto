\subsection*{Conceptos de Modelo de Negocio}
\addcontentsline{toc}{subsection}{Conceptos de Modelo de Negocio}

\subsubsection*{Modelo de Negocio}
\addcontentsline{toc}{subsubsection}{Modelo de Negocio}
Nuestro servicio de planificación financiera se dirige a segmentos de clientes específicos: especialmente jóvenes adultos, trabajadores informales, nómadas digitales y en general personas con escasa educación financiera, el modelo de negocio integra lo siguiente:

\begin{itemize}
	\item \textbf{Segmentos clave:} Es la población objetivo de la que nuestro modelo de negocios se basa en un conocimiento exhaustivo acerca de las necesidades específicas de la misma.
	
	\item \textbf{Propuesta de valor:} La propuesta de valor es el factor por el cual un potencial cliente decida decantarse por una u otra empresa; la finalidad de la propuesta es buscar solucionar un problema o buscar satisfacer la necesidad que posee un cliente. Las propuestas de valor, por lo tanto, son el conjunto de productos o servicios que satisfagan los requisitos dentro de un segmento de mercado en específico.
	
	\item \textbf{Canales de distribución:} Los canales de comunicación, distribución y de venta buscan lograr el contacto entre la empresa con los clientes. Estos puntos de contacto hacia el cliente desempeñan una labor fundamental para la experiencia personal del cliente.
	
	\item \textbf{Relación con los clientes:} Las empresas buscan definir que tipo de relación buscan establecer con cada segmento del mercado, para ello, lo logran a través de la captación y la relación de los clientes, su fidelización a la empresa y, en consecuencia, la estimulación de las ventas de los servicios o productos de la empresa.
	
	\item \textbf{Fuentes de ingresos:} Se refiere al flujo de caja que generá la empresa en cada distinto segmento del mercado. Para ello existen distintas formas de ingreso tales como transacciones derivadas a partir de pagos puntuales dados por los clientes y/o pagos periódicos realizados a cambio del suministro de una propuesta de valor o de servicios pos-venta de atención dirigida hacia el cliente.
	
	\item \textbf{Actividades clave:} Este tipo de actividades son las más esenciales que debe implementar una empresa para buscar su camino al éxito, estas son necesarias para lograr el objetivo de crear y ofrecer una propuesta de valor, alcanzar los mercados de destino y lograr establecer relaciones con los clientes exitosamente y de esta forma lograr percibir ganancias en la empresa. Las actividades que se necesiten llevar a cabo varían dependiendo de la función del modelo de negocio.
	
	\item \textbf{Recursos clave:} Los modelos de negocio necesitan de recursos claves los que permiten a las diferentes empresas crear y lograr ofrecer propuestas de valor, llegar a los mercados, y conseguir establecer relaciones con segmentos de mercado y conseguir ingresos de forma similar a las actividades clave. Los recursos clave pueden ser físicos, económicos, intelecuales o humanos. Adicionalmente, la empresa puede tenerlos en su propiedad, alquilirlos u obtenerlos de otra forma gracias a sus socios clave. 
	
	\item \textbf{Socios clave:} Son las distintas alianzas que crean las empresas con otra empresas o personas con la meta de optimizar sus modelos de negocios, reducir los riesgos implicados y adquierir más recursos.
	
	\item \textbf{Estructura de costos:} La estructura de costos busca describir los principales costos de los que se incurre a la hora de trabajar dentro de un modelo de negocio determinado. La creación y entrega de valor así como el mantenimiento de las relaciones que se tiene con los múltiples clientes y la generación de las ganacias tienen un coste percibido.
\end{itemize}

\subsubsection*{Modelo de Canvas}
\addcontentsline{toc}{subsubsection}{Modelo de Canvas}
El Business Model Canvas (también conocido como el diagrama CANVA) es una herramienta de gestión estratégica desarrollada por Alexander Osterwalder e Yves Pigneur la cual tiene como objetivo permitir estructurar de forma visual los elementos clave de un negocio, como la propuesta de valor, los clientes, la infraestructura y las finanzas. Diversos estudios resaltan que incluir un modelo de negocio en un plan empresarial es fundamental para validar y fortalecer la idea antes de su implementación.

El modelo Canvas permite ser lo suficientemente sencilla para aplicarse a cualquier escenario de empresa, ya sean pequeñas, medianas o inclusive grandes empresas, independientemente de la forma de su estrategia de negocio y público objetivo. A continuación, mediante la siguiente figura observamos la estructura del modelo Canva en la cual se expone la forma en que se organiza el modelo de negocio:

\begin{figure}[H]
	\centering
	\includegraphics[width=15cm]{Imagenes/PlantillaModeloCanvas.jpg}
	\caption[{Ejemplo de modelo CANVAS.}]{\centering Ejemplo de modelo CANVAS. \textit{Fuente:} Sitio Web: (¿Cómo hacer un modelo de negocios canvas?, Redacción CEUPE, 2022)} 
	\label{fig:planmodelocanvas}
\end{figure}

En este marco, nuestro servicio dirigido a personas sin conocimientos financieros en Bogotá– surge como una propuesta innovadora que ofrece herramientas didácticas y dinámicas para la planificación financiera personal. La iniciativa se complementa con un enfoque de gamificación ligera y la construcción de una comunidad de educación financiera, buscando motivar el aprendizaje, promover hábitos saludables y fomentar la inclusión financiera.

\paragraph*{Justificación financiera base}
\addcontentsline{toc}{paragraph}{Justificación financiera base}

Para la inversión inicial base, se estima un equipo pequeño de desarrollo para 6–8 meses. Por ejemplo, 2 programadores \textbf{semi-senior} ($\sim$COP 6–7 M cada uno) y 1 diseñador UX/UI ($\sim$COP 4–5 M). A esto se suman un redactor/contenidos ($\sim$COP 3–4 M) y un especialista de marketing (digital ads, redes sociales) ($\sim$COP 3 M). También consideramos infraestructura inicial (servidores, licencias) y gastos de lanzamiento (publicidad). Los trámites legales básicos en Colombia son menores (inscripción en Cámara $\sim$COP 50–100 mil) pero agregamos honorarios de consultoría ($\sim$COP 3–5 M). En resumen, la inversión inicial desglosada por partidas clave es:

\begin{longtable}{|c|c|c|c|c|}
	\caption[{Partes del proyecto con costo estimado y justificación.}]
	{\centering Partes del proyecto con costo estimado y justificación. \textit{Fuente:} Autores.}
	\label{TablaPartesCostoJust} \\
	\hline
	\rowcolor[HTML]{D9EAD3} 
	\textbf{\begin{tabular}[c]{@{}c@{}}Partida\\principal\end{tabular}} &
	\textbf{Subpartida} &
	\textbf{\begin{tabular}[c]{@{}c@{}}Valor \\(COP)\end{tabular}} &
	\textbf{\begin{tabular}[c]{@{}c@{}}Años \\(depreciación/\\tratamiento)\end{tabular}} &
	\textbf{Justificación} \\ \hline
	\endfirsthead
	
	\multicolumn{5}{c}%
	{{\bfseries \tablename\ \thetable{} -- continuación}} \\ \hline
	\rowcolor[HTML]{D9EAD3} 
	\textbf{\begin{tabular}[c]{@{}c@{}}Partida\\principal\end{tabular}} &
	\textbf{Subpartida} &
	\textbf{\begin{tabular}[c]{@{}c@{}}Valor \\(COP)\end{tabular}} &
	\textbf{\begin{tabular}[c]{@{}c@{}}Años \\(depreciación/\\tratamiento)\end{tabular}} &
	\textbf{Justificación} \\ \hline
	\endhead
	
	\hline \multicolumn{5}{r}{{Continúa en la siguiente página}} \\ \hline
	\endfoot
	
	\hline
	\endlastfoot
	
	\multirow{3}{*}{\textbf{Diseño UX/UI}} &
	\begin{tabular}[c]{@{}c@{}}Investigación UX\\(tesis, entrevistas)\end{tabular} &
	\$1.500.000 & Gasto (0) &
	\begin{tabular}[c]{@{}c@{}}Trabajo inicial de\\investigación (gasto\\operativo).\end{tabular} \\ \cline{2-5}
	
	& \begin{tabular}[c]{@{}c@{}}Diseño UI\\(mockups,\\prototipos)\end{tabular} &
	\$2.000.000 & 3 años &
	\begin{tabular}[c]{@{}c@{}}Activo intangible\\ligado al producto.\end{tabular} \\ \cline{2-5}
	
	& \begin{tabular}[c]{@{}c@{}}Pruebas de \\usabilidad y ajustes\end{tabular} &
	\$1.500.000 & Gasto (0) &
	\begin{tabular}[c]{@{}c@{}}Pruebas y refinamiento\\prelanzamiento.\end{tabular} \\ \hline
	
	\rowcolor[HTML]{eaffe5} 
	\multicolumn{2}{|c|}{\textbf{Subtotal Diseño}} & \multicolumn{3}{|c|}{\textbf{\$5.000.000}} \\ \hline
	
	\multirow{4}{*}{\begin{tabular}[c]{@{}c@{}}\textbf{Producción de}\\\textbf{contenidos}\\\textbf{interactivos}\end{tabular}} &
	\begin{tabular}[c]{@{}c@{}}Guión y diseño\\instruccional\end{tabular} &
	\$2.000.000 & 2 años &
	\begin{tabular}[c]{@{}c@{}}Contenido reutilizable,\\vida útil limitada.\end{tabular} \\ \cline{2-5}
	
	& \begin{tabular}[c]{@{}c@{}}Desarrollo de\\simuladores /\\programación\end{tabular} &
	\$3.000.000 & 3 años &
	\begin{tabular}[c]{@{}c@{}}Elementos de\\software/código con \\vida útil.\end{tabular} \\ \cline{2-5}
	
	& \begin{tabular}[c]{@{}c@{}}Producción \\multimedia \\(audio/video)\end{tabular} &
	\$2.000.000 & 2 años &
	\begin{tabular}[c]{@{}c@{}}Material audiovisual\\ para la plataforma.\end{tabular} \\ \cline{2-5}
	
	& \begin{tabular}[c]{@{}c@{}}Licencias y recursos \\(stock, assets)\end{tabular} &
	\$1.000.000 & 1-2 años &
	\begin{tabular}[c]{@{}c@{}}Licencia y assets\\multimedia.\end{tabular} \\ \hline
	
	\rowcolor[HTML]{eaffe5} 
	\multicolumn{2}{|c|}{\textbf{Subtotal Contenidos}} & \multicolumn{3}{|c|}{\textbf{\$8.000.000}} \\ \hline
	
	\multirow{3}{*}{\begin{tabular}[c]{@{}c@{}}\textbf{Equipos de}\\\textbf{cómputo}\\\textbf{y hardware}\end{tabular}} &
	\begin{tabular}[c]{@{}c@{}}3 estaciones de\\trabajo\\(PC/portátiles)\end{tabular} &
	\$12.000.000 & 5 años &
	\begin{tabular}[c]{@{}c@{}}$\sim$\$4.000.000 c/u, uso\\ de desarrollo.\end{tabular} \\ \cline{2-5}
	
	& \begin{tabular}[c]{@{}c@{}}Monitores,\\ periféricos y\\ accesorios\end{tabular} &
	\$1.500.000 & 3-5 años &
	\begin{tabular}[c]{@{}c@{}}Monitores, teclados,\\ ratones, etc.\end{tabular} \\ \cline{2-5}
	
	& \begin{tabular}[c]{@{}c@{}}Almacenamiento / \\backup(NAS, \\discos)\end{tabular} &
	\$1.500.000 & 5 años &
	\begin{tabular}[c]{@{}c@{}}Infra para backups y \\pruebas.\end{tabular} \\ \hline
	
	\rowcolor[HTML]{eaffe5} 
	\multicolumn{2}{|c|}{\textbf{Subtotal Equipos}} & \multicolumn{3}{|c|}{\textbf{\$15.000.000}} \\ \hline
	
	\multirow{4}{*}{\begin{tabular}[c]{@{}c@{}}\textbf{Marketing y}\\\textbf{lanzamiento}\end{tabular}} &
	\begin{tabular}[c]{@{}c@{}}Estrategia y\\branding(diseño\\marca)\end{tabular} &
	\$4.000.000 & Gasto(0) &
	\begin{tabular}[c]{@{}c@{}}Actividad promocional\\inicial.\end{tabular} \\ \cline{2-5}
	
	& \begin{tabular}[c]{@{}c@{}}Campañas digitales\\(Google Ads /\\RRSS)\end{tabular} &
	\$8.000.000 & Gasto(0) &
	\begin{tabular}[c]{@{}c@{}}Inversión en\\adquisición de\\usuarios.\end{tabular} \\ \cline{2-5}
	
	& \begin{tabular}[c]{@{}c@{}}Producción \\audovisual / \\material\\promocional\end{tabular} &
	\$4.000.000 & Gasto(0) &
	\begin{tabular}[c]{@{}c@{}}Videos, promos\\ materiales gráficos.\end{tabular} \\ \cline{2-5}
	
	& \begin{tabular}[c]{@{}c@{}}Eventos \\demostraciones y\\material físico\end{tabular} &
	\$2.000.000 & Gasto(0) &
	\begin{tabular}[c]{@{}c@{}}Lanzamiento / ferias /\\demos.\end{tabular} \\ \hline
	
	\rowcolor[HTML]{eaffe5} 
	\multicolumn{2}{|c|}{\textbf{Subtotal Marketing}} & \multicolumn{3}{|c|}{\textbf{\$18.000.000}} \\ \hline
	
	\multirow{3}{*}{\begin{tabular}[c]{@{}c@{}}\textbf{Consultoría}\\\textbf{especializada}\\\textbf{y mentoría}\end{tabular}} &
	\begin{tabular}[c]{@{}c@{}}Asesoría legal y\\cumplimiento\end{tabular} &
	\$2.000.000 & Gasto(0) &
	\begin{tabular}[c]{@{}c@{}}Constitución, contratos, \\ revisiones legales.\end{tabular} \\ \cline{2-5}
	
	& \begin{tabular}[c]{@{}c@{}}Seguridad de la\\información / datos\end{tabular} &
	\$2.000.000 & 1-2 años &
	\begin{tabular}[c]{@{}c@{}}Auditoría e\\implementación de\\controles.\end{tabular} \\ \cline{2-5}
	
	& \begin{tabular}[c]{@{}c@{}}Mentoría\\estratégica y \\producto (advice)\end{tabular} &
	\$2.000.000 & Gasto(0) &
	\begin{tabular}[c]{@{}c@{}}Sesiones de\\ mentoría / validación\end{tabular} \\ \cline{2-5}
	
	\rowcolor[HTML]{eaffe5} 
	\multicolumn{2}{|c|}{\textbf{Subtotal Consultoría}} & \multicolumn{3}{|c|}{\textbf{\$6.000.000}} \\ \hline
	
	\multirow{4}{*}{\begin{tabular}[c]{@{}c@{}}\textbf{Creación de}\\\textbf{infraestructura}\\\textbf{y}\\\textbf{arquitectura}\\\textbf{(AWS y}\\\textbf{licencias)}\end{tabular}} &
	\begin{tabular}[c]{@{}c@{}}Setup inicial en la\\nube (arquitectura,\\deploy)\end{tabular} &
	\$3.000.000 & 3 años &
	\begin{tabular}[c]{@{}c@{}}Configuración, IaC\\despliegue inicial.\end{tabular} \\ \cline{2-5}
	
	& \begin{tabular}[c]{@{}c@{}}Licencias software /\\herramientas /\\analytics\end{tabular} &
	\$2.000.000 & 1-3 años &
	\begin{tabular}[c]{@{}c@{}}Licencias SaaS,\\herramientas de\\analítica.\end{tabular} \\ \cline{2-5}
	
	& \begin{tabular}[c]{@{}c@{}}Dominios,\\certificados SSL,\\servicios menores\end{tabular} &
	\$500.000 & 1 año &
	\begin{tabular}[c]{@{}c@{}}Renovaciones\\anuales.\end{tabular} \\ \cline{2-5}
	
	& \begin{tabular}[c]{@{}c@{}}Implementación\\DevOps /\\automatización\end{tabular} &
	\$2.500.000 & 3 años &
	\begin{tabular}[c]{@{}c@{}}Pipelines, CI/CD\\infraestructura como\\código.\end{tabular} \\ \hline
	
	\rowcolor[HTML]{eaffe5} 
	\multicolumn{2}{|c|}{\textbf{Subtotal Infraestructura}} & \multicolumn{3}{|c|}{\textbf{\$8.000.000}} \\ \hline
	
	\rowcolor[HTML]{D9EAD3} 
	\multicolumn{2}{|c|}{\textbf{TOTAL INVERSIÓN INICIAL}} & \multicolumn{3}{|c|}{\textbf{\$60.000.000}} \\ \hline
\end{longtable}

Este valor de inversión inicial junta la codificación, las pruebas y ajustes de la app/web, la creación de infraestructura y arquitectura, manejo de consultorías, marketing, equipos de cómputo, entre muchos otros más. Las cifras se basan en precios y rangos salariales actuales en Colombia, además cotizaciones de mercado para agencias digitales. El total ($\sim$COP 60,000,000) corresponde a la suma de todos los ítems anteriores.

\paragraph*{Costos operativos mensuales}
\addcontentsline{toc}{paragraph}{Costos operativos mensuales}
Tras el lanzamiento, los costos fijos mensuales son principalmente \textbf{sueldos del equipo} y gastos de plataforma. Consideramos:

\begin{itemize}
	\item \textbf{Sueldos de personal:} 2 desarrolladores (dando un total de 12,000,000 COP) los cuales serán trabajadores para el montaje de la plataforma para el primer año de la empresa.
	
	\item \textbf{Infraestructura:} Sostenimiento de los servidores en la nube y el manejo de licencias y dominios, se estima un gasto operacional desde el año de operación con un valor de 9,000,000 COP cuyo gasto aumenta cada trimestre en una tasa del 6\%.
	
	\item \textbf{Soporte:} Manejo de soporte del funcionamiento del negocio, este empieza a ser relevante desde el año de operación con un valor de 6,000,000 COP cuyo gasto aumenta cada trimestre en una tasa del 2\%.
	
	\item \textbf{Intereses sobre crédito:} Los intereses dados del préstamo bancario, el cual tendrá una tasa del 4,16\% sobre el valor total del préstamo no pagado hasta ese punto, se estima que al mes 32 (tras 10 períodos de amortiguación) se ha dado la totalidad del pago del préstamo y por ende, los intereses.
	
	\item \textbf{Depreciación:} La depreciación, reparación y actualización de las inversiones iniciales, se estima un total de 3,712,000 COP por cada trimestre en gastos.
	
	\item \textbf{Impuestos directos:} Impuestos pagos al estado, cuyo valor serán las ganancias netas gravables de cada trimestre por un 4\% de tasa de sobre renta.
\end{itemize}

En total, estimamos costos operativos \textbf{aproximados de COP 7–9 M mensuales}. Mientras que, durante el proceso de construcción del programa, los costos operativos serán de \textbf{aproximados COP 15-16 M mensuales.} Cabe recalcar que los salarios considerados son acordes a los rangos reportados en el sector IT colombiano.

\paragraph*{Ingresos proyectados}
\addcontentsline{toc}{paragraph}{Ingresos proyectados}
El modelo de ingresos es \textbf{suscripción mensual premium}. Tomando como referencia plataformas fintech, se propone un plan Premium de alrededor \textbf{COP 20.000 mensuales por usuario.} (Por ejemplo, la app de inversiones Trii cobra COP 27.900 mensuales por su plan PRO.) Se asume un crecimiento gradual de clientes: inicio de ingresos moderado en el mes 10, y escalada progresiva gracias al marketing y recomendaciones. En un escenario conservador, se proyecta lo siguiente:

\begin{itemize}
	\item \textbf{Tasa de crecimiento:} Partiendo de unos cientos de usuarios, se dobla la base cada 3–4 meses por campañas y viralidad.
	
	\item \textbf{Ingresos mensuales:} Ya estando en el mes 13 se esperan unos COP 21 M/mes (juntando las ganancias obtenidas por suscripciones y de sponsors inversores); en el mes 25 alrededor de COP 43 M/mes; y al cierre de los 39 meses superar COP 70 M/mes en suscripciones, según proyección.
	
	\item \textbf{Otras fuentes:} Se considera marginalmente ingresos por publicidad en la app o ventas de sesiones con expertos, pero el grueso proviene de suscripciones.
\end{itemize}

Este crecimiento está respaldado por el interés observado: por ejemplo, la fintech peruana Inspira logró 70.000 descargas antes de su lanzamiento oficial\cite{Bibl014}, sugiriendo un mercado ávido de educación financiera. En la tabla siguiente se muestra un escenario detallado de \textbf{ingresos y egresos mensuales} a 39 meses (cuya actualización es dada trimestralmente).\\

\begin{longtable}{|c|c|c|}
	\caption[{Partes del proyecto con ingresos y egresos mensuales. }]{\centering Partes del proyecto con ingresos y egresos mensuales. (Valores ilustrativos a mostrar). \textit{Fuente:} Autores.}
	\label{TablaIngresosProyectados} \\
	
	\hline
	\textbf{Mes} & \textbf{Ingresos (COP)} & \textbf{Egresos (COP)} \\ \hline
	\endfirsthead
	\hline
	\textbf{Mes} & \textbf{Ingresos (COP)} & \textbf{Egresos (COP)} \\ \hline
	\endhead
	\hline
	\endfoot
	\hline
	\endlastfoot
	\textbf{1}  & 13,000,000 & 20,000,000 \\ \hline
	\textbf{2}  & 13,000,000 & 20,000,000 \\ \hline
	\textbf{3}  & 13,000,000 & 20,000,000 \\ \hline
	\textbf{4}  & 0  & 15,078,000 \\ \hline
	\textbf{5}  & 0 & 15,078,000 \\ \hline
	\textbf{6}  & 0 & 15,078,000 \\ \hline
	\textbf{7}  & 0 & 15,024,000 \\ \hline
	\textbf{8}  & 0 & 15,024,000 \\ \hline
	\textbf{9}  & 0 & 15,024,000 \\ \hline
	\textbf{10} & 5,916,000 & 8,970,000 \\ \hline
	\textbf{11} & 5,916,000 & 8,970,000 \\ \hline
	\textbf{12} & 5,916,000 & 8,970,000 \\ \hline
	\textbf{13} & 7,289,000 & 7,984,000 \\ \hline
	\textbf{14} & 7,289,000 & 7,984,000 \\ \hline
	\textbf{15} & 7,289,000 & 7,984,000 \\ \hline
	\textbf{16} & 9,672,000 & 8,228,000 \\ \hline
	\textbf{17} & 9,672,000 & 8,228,000 \\ \hline
	\textbf{18} & 9,672,000 & 8,228,000 \\ \hline
	\textbf{19} & 11,064,000 & 8,449,000 \\ \hline
	\textbf{20} & 11,064,000 & 8,449,000 \\ \hline
	\textbf{21} & 11,064,000 & 8,449,000 \\ \hline
	\textbf{22} & 12,662,000 & 8,689,000 \\ \hline
	\textbf{23} & 12,662,000 & 8,689,000 \\ \hline
	\textbf{24} & 12,662,000 & 8,689,000 \\ \hline
	\textbf{25} & 14,496,000 & 8,948,000 \\ \hline
	\textbf{26} & 14,496,000 & 8,948,000 \\ \hline
	\textbf{27} & 14,496,000 & 8,948,000 \\ \hline
	\textbf{28} & 16,603,000 & 9,230,000 \\ \hline
	\textbf{29} & 16,603,000 & 9,230,000 \\ \hline
	\textbf{30} & 16,603,000 & 9,230,000 \\ \hline
	\textbf{31} & 19,022,000 & 9,536,000 \\ \hline
	\textbf{32} & 19,022,000 & 9,536,000 \\ \hline
	\textbf{33} & 19,022,000 & 9,536,000 \\ \hline
	\textbf{34} & 21,800,000 & 8,569,000 \\ \hline
	\textbf{35} & 21,800,000 & 8,569,000 \\ \hline
	\textbf{36} & 21,800,000 & 8,569,000 \\ \hline
	\textbf{37} & 26,858,000 & 9,105,000 \\ \hline
	\textbf{38} & 26,858,000 & 9,105,000 \\ \hline
	\textbf{39} & 26,858,000 & 9,105,000 \\ \hline
	
\end{longtable}

Con las proyecciones anteriores encontradas en la tabla \ref{TablaIngresosProyectados}, se calcula el flujo de caja mensual neto y el saldo acumulado, incluyendo la inversión inicial como un flujo negativo dentro del mes 0 de alrededor de unos \$21,000,000 COP. Esto se justifica teniendo en cuenta un total de \$60,000,000 COP dedicados a la inversión inicial, los cuales seran amortiguados con un préstamo bancario relativo al 65\% de nuestra inversión inicial, lo que significa un total de \$39,000,000 COP.

\begin{longtable}{|c|c|c|}
	\caption[{Cálculo de flujo de caja mensual. }]{\centering Cálculo de flujo de caja mensual. \textit{Fuente:} Autores.}
	\label{TablaFlujoCajaMensual} \\
	
	\hline
	\textbf{Mes} & \textbf{Flujo neto mensual (COP)} & \textbf{Saldo acumulado (COP)} \\ \hline
	\endfirsthead
	\hline
	\textbf{Mes} & \textbf{Flujo neto mensual (COP)} & \textbf{Saldo acumulado (COP)} \\ \hline
	\endhead
	\hline
	\endfoot
	\hline
	\endlastfoot
	\textbf{1}  & -7,000,000 & -7,000,000 \\ \hline
	\textbf{2}  & -7,000,000 & -14,000,000 \\ \hline
	\textbf{3}  & -7,000,000 & -21,000,000 \\ \hline
	\textbf{4}  & -15,078,000 & -36,078,000 \\ \hline
	\textbf{5}  & -15,078,000 & -51,156,000 \\ \hline
	\textbf{6}  & -15,078,000 & -66,234,000 \\ \hline
	\textbf{7}  & -15,024,000 & -81,258,000 \\ \hline
	\textbf{8}  & -15,024,000 & -96,283,000 \\ \hline
	\textbf{9}  & -15,024,000 & -111,307,000 \\ \hline
	\textbf{10} & -3,053,000 & -114,360,000 \\ \hline
	\textbf{11} & -3,053,000 & -117,413,000 \\ \hline
	\textbf{12} & -3,053,000 & -120,467,000 \\ \hline
	\textbf{13} & -694,000 & -121,162,000 \\ \hline
	\textbf{14} & -694,000 & -121,857,000 \\ \hline
	\textbf{15} & -694,000 & -122,552,000 \\ \hline
	\textbf{16} & 1,443,000 & -121,108,000 \\ \hline
	\textbf{17} & 1,443,000 & -119,665,000 \\ \hline
	\textbf{18} & 1,443,000 & -118,221,000 \\ \hline
	\textbf{19} & 2,614,000 & -115,607,000 \\ \hline
	\textbf{20} & 2,614,000 & -112,992,000 \\ \hline
	\textbf{21} & 2,614,000 & -110,377,000 \\ \hline
	\textbf{22} & 3,973,000 & -106,404,000 \\ \hline
	\textbf{23} & 3,973,000 & -102,431,000 \\ \hline
	\textbf{24} & 3,973,000 & -98,458,000 \\ \hline
	\textbf{25} & 5,547,000 & -92,910,000 \\ \hline
	\textbf{26} & 5,547,000 & -87,362,000 \\ \hline
	\textbf{27} & 5,547,000 & -81,814,000 \\ \hline
	\textbf{28} & 7,372,000 & -74,441,000 \\ \hline
	\textbf{29} & 7,372,000 & -67,069,000 \\ \hline
	\textbf{30} & 7,372,000 & -59,696,000 \\ \hline
	\textbf{31} & 9,485,000 & -50,211,000 \\ \hline
	\textbf{32} & 9,485,000 & -40,725,000 \\ \hline
	\textbf{33} & 9,485,000 & -31,240,000 \\ \hline
	\textbf{34} & 13,230,000 & -18,009,000 \\ \hline
	\textbf{35} & 13,230,000 & -4,778,000 \\ \hline
	\textbf{36} & 13,230,000 & 8,452,000 \\ \hline
	\textbf{37} & 17,753,000 & 26,205,000 \\ \hline
	\textbf{38} & 17,753,000 & 43,958,000 \\ \hline
	\textbf{39} & 17,753,000 & 61,711,000 \\ \hline
\end{longtable}

En este escenario, el flujo neto se torna positivo a partir del mes 16 y el saldo acumulado rompe el punto de equilibrio aproximadamente al mes 36, superando COP 0. Para el cierre del mes 39 se proyecta un saldo acumulado de aproximadamente \textbf{COP 61 M} (en términos netos), lo que indica que el proyecto habría recuperado la inversión inicial y generado utilidad. Estas cifras son aproximadas y dependen de alcanzar las métricas de usuarios y suscriptores asumidas, pero reflejan un escenario factible con retorno en mediano plazo. 