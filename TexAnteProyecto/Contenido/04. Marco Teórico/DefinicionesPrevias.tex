\subsection*{Definiciones previas}
\addcontentsline{toc}{subsection}{Definiciones previas}

\subsubsection*{Finanzas personales}
\addcontentsline{toc}{subsubsection}{Finanzas personales}
La gestión de las finanzas personales se refiere a la administración que realiza cada individuo o familia sobre sus recursos económicos, incluyendo el control de ingresos, gastos, ahorro e inversión. Su organización requiere de un enfoque metódico y constancia para evitar desequilibrios, pues mantener una economía doméstica saludable exige “tiempo, metodología y paciencia” y seguir pasos
ordenados. En este sentido, la digitalización ha jugado un papel clave: las nuevas tecnologías han facilitado enormemente el manejo de finanzas cotidianas, permitiendo, por ejemplo, realizar transacciones y elaborar presupuestos sin
necesidad de acudir a oficinas físicas.

Tal como destaca el BBVA, la digitalización es
«una fuente de oportunidades y una herramienta que facilita el uso de los servicios financieros», brindando más facilidades a los usuarios para gestionar sus finanzas personales. Esto reduce la fricción en tareas como el seguimiento de gastos y la obtención de información financiera, mejorando la salud financiera del usuario. Sin embargo, varios estudios (e.g. CFA Institute, 2024) señalan que la alfabetización financiera en la población general aún es limitada: menos de la mitad de los adultos tiene conocimientos adecuados de finanzas básicas. Dado que el nivel de
comprensión financiera se mantiene alrededor de un 52\% en evaluaciones como el índice P-Fin de Estados Unidos, se refuerza la importancia de servicios educativos y de asesoría que apoyen a personas sin experiencia previa en finanzas. En suma, la teoría de las finanzas personales resalta la necesidad de acompañar al cliente en la planificación presupuestaria y de ahorro, promoviendo decisiones informadas que mejoren su seguridad económica a largo plazo.

\subsubsection*{Nómadas digitales}
\addcontentsline{toc}{subsubsection}{Nómadas digitales}
Los nómadas digitales constituyen un perfil emergente de trabajadores remotos que, gracias a la tecnología, desempeñan sus labores desde cualquier lugar del mundo. Como señalan \cite{Bibl008} y refleja el \cite{Bibl009}, este concepto define «un estilo de vida en el que, gracias a los avances tecnológicos, los profesionales pueden desempeñar su labor desde cualquier lugar y en cualquier momento». Se estima que actualmente existen unos 40 millones de nómadas digitales en el mundo, la mayoría jóvenes (30-39 años) con altos ingresos sector tecnológico.

En América Latina esta tendencia también crece: países como Colombia han implementado visas especiales para atraer a estos profesionales itinerantes. En este contexto, la población objetivo en Bogotá incluye a muchos
migrantes y trabajadores móviles que carecen de una asesoría financiera tradicional. Para estos usuarios, gestionar ingresos variables y gastos
internacionales presenta desafíos únicos. Por ello, el plan de negocio debe reconocer que los nómadas digitales valoran la flexibilidad financiera y requieren soluciones adaptables (por ejemplo, manejo de distintas monedas, ahorro para emergencias médicas o planes de retiro internacionales). Atender a este segmento implica ofrecer servicios de planificación financiera personalizados que contemplen su estilo de vida dinámico, ayudándoles a estabilizar sus finanzas pese a la movilidad geográfica.

\subsubsection*{Plan de negocio}
\addcontentsline{toc}{subsubsection}{Plan de negocio}
Un plan de negocios es el documento rector donde se articula la propuesta de valor y la viabilidad de un emprendimiento. Constituye la “tarjeta de presentación” de la empresa ante inversores y colaboradores potenciales. En él se definen los objetivos estratégicos, la misión, el mercado meta, la estructura organizativa y los recursos necesarios, así como las proyecciones financieras a medio plazo. Según \cite{Bibl009}, un plan de negocios bien elaborado debe identificar la oportunidad de mercado, analizar su viabilidad técnica, económica y financiera, y describir las estrategias para convertir la idea en empresa real. 

En esencia, su objetivo principal es ayudar al emprendedor a reflexionar sobre todos los aspectos estratégicos de su proyecto. Esto incluye delinear un cronograma de resultados esperados y los pasos clave a seguir. Por ejemplo, se suelen estructurar apartados como resumen ejecutivo,
análisis de mercado, descripción del servicio (planificación financiera), modelo de ingresos, plan operativo y proyecciones financieras (cuentas de resultados y flujo de caja). Cada sección contribuye a garantizar que el proyecto sea factible: desde cuantificar la demanda en Bogota hasta prever inversiones en tecnología o capacitación. Un plan de negocios claro y detallado reducciona la incertidumbre y provee un camino definido para tomar decisiones, convirtiendo una idea abstracta en una estrategia coherente para crear valor sostenible en el mercado.

\paragraph*{Asesoramiento financiero}
\addcontentsline{toc}{paragraph}{Asesoramiento financiero}
El asesoramiento financiero es un servicio profesional cuyo objetivo es guiar a los individuos en la toma de decisiones económicas y patrimoniales, con el fin de alcanzar sus metas (compra de vivienda, jubilación, ahorro, etc.) y mejorar su salud financiera. En palabras del \cite{Bibl010}, “la función de un asesor es guiar a las
personas para que puedan conseguir los objetivos que se hayan planteado, buscando siempre la mejora de la salud financiera de sus clientes”. Existen diferentes modalidades de asesoría según el vínculo del profesional:
\begin{itemize}
	\item \textbf{Asesor independiente:} No está ligado a ninguna entidad financiera. Este tipo de asesor actúa de forma imparcial, pudiendo recomendar productos de cualquier banco o fondo. Generalmente trabaja bajo contrato y no cobra comisiones por venta, lo que busca garantizar que el consejo sea objetivo.
	\item \textbf{Asesor dependiente o gestor comercial:} Trabaja dentro de una entidad financiera y suele ofrecer exclusivamente los productos de esa institución, percibiendo comisiones asociadas. Aunque su formación puede ser similar, su independencia está limitada por su relación con la empresa matriz.
\end{itemize}
Un buen asesor debe poseer conocimientos especializados en inversiones, gestión de riesgos y planificación patrimonial, y sobre todo debe adaptar el servicio al perfil de cada cliente. Entre las características señaladas por BBVA destaca la
personalización del servicio: el asesor financiero analiza la capacidad de ahorro, nivel de endeudamiento, tolerancia al riesgo y situación socioeconómica del cliente, para diseñar recomendaciones a su medida. Asimismo, emplea un lenguaje claro y evita tecnicismos innecesarios, explicando abiertamente riesgos y beneficios de cada alternativa.

Todo ello busca generar confianza y transparencia con el cliente. Por otro lado, es importante mencionar que existe un marco regulatorio (p. ej. directivas MiFID en Europa) que protege al inversor minorista: estas normas
exigen al asesor evaluar la conveniencia de los productos según el perfil del cliente, fomentando prácticas éticas y reduciendo conflictos de interés. En síntesis, el asesoramiento financiero combina expertise profesional con un enfoque centrado en el cliente, proporcionando orientación continua que construye seguridad y confianza en la gestión económica personal.

\subsubsection*{Herramientas tecnológicas para el asesoramiento financiero}
\addcontentsline{toc}{subsubsection}{Herramientas tecnológicas para el asesoramiento financiero}
La revolución digital ha dado lugar a diversas herramientas tecnológicas que facilitan tanto la gestión cotidiana de las finanzas personales como el mismo proceso de asesoría. Entre las más relevantes se encuentran:

\begin{itemize}
	\item \textbf{Aplicaciones de gestión de gastos:} Apps móviles como Toshl o Money Lover permiten registrar fácilmente ingresos y egresos diarios desde smartphones y visualizar el presupuesto a través de gráficos interactivos. Estas	aplicaciones animan al usuario a llevar un seguimiento disciplinado de sus gastos, al tiempo que hacen el proceso más ameno mediante interfaces lúdicas.
	\item \textbf{Plataformas agregadoras de cuentas bancarias:} Herramientas como BBVA Bconomy, Ahorro.net o Fintonic centralizan en un solo panel todas las cuentas y tarjetas del usuario. Al consolidar la información financiera, estas plataformas analizan patrones de gasto e ingresos en tiempo real y ofrecen recomendaciones personalizadas de ahorro. Por ejemplo, Bconomy evalúa la ``salud financiera'' del cliente y sugiere planes a medida, mientras que Ahorro.net se define como un ``asesor financiero de bolsillo'' que monitorea múltiples cuentas para aconsejar al usuario sobre cómo optimizar sus ahorros. Fintonic, por su parte, sincroniza automáticamente los movimientos bancarios y brinda alertas de movimiento y consejos personalizados de ahorro.
	
	\begin{figure}[H]
		\centering
		\includegraphics[width=15cm]{Imagenes/LogoBConomy.png}
		\caption[{Logotipo de BBVA Bconomy.}]{\centering Logotipo de BBVA Bconomy. \textit{Fuente:} Sitio Web: (Entre nosotros, BBVA, 2024)} 
		\label{fig:logobconomy}
	\end{figure}
	
	\item \textbf{Chatbot asesores:} Se trata de plataformas automatizadas basadas en algoritmos que crean y gestionan carteras de inversión sin intervención humana continua. Según análisis de \cite{Bibl011}, estos servicios ofrecen asesoría financiera masiva a menor costo, orientada generalmente a inversión pasiva en fondos indexados (ETFs), pero adaptada al perfil de riesgo de cada cliente. Los robo-advisors facilitan el acceso a la planificación de inversiones a segmentos con menores patrimonios o con afinidad por canales digitales, al ofrecer transparencia en costos y recomendaciones objetivas.
	\item \textbf{Calculadoras y simuladores en línea:} Existen numerosas herramientas web o móviles que ayudan a proyectar escenarios financieros básicos (simuladores de ahorros, préstamos, hipotecas, jubilación, etc.). Estas calculadoras apoyan la educación financiera práctica, al permitir evaluar rápidamente cómo variaciones en la tasa de interés, inflación o ingresos afectan el presupuesto personal. Si bien no sustituyen al asesor humano, complementan el proceso de planificación, empoderando al usuario para entender conceptos clave.
\end{itemize}

\paragraph*{Alfabetización financiera}
\addcontentsline{toc}{paragraph}{Alfabetización financiera}
La alfabetización financiera se entiende como el conjunto de conocimientos, habilidades, actitudes y comportamientos que permiten a las personas tomar decisiones financieras informadas y responsables en su vida cotidiana \cite{Bibl014}. En este sentido, un individuo alfabetizado financieramente comprende conceptos básicos como presupuesto, interés, deuda y ahorro, y aplica esa comprensión para gestionar sus ingresos, gastos y ahorros. Estudios internacionales demuestran que menores niveles de alfabetización financiera se asocian con menor bienestar económico, por lo que los servicios de planificación financiera deben enfocarse en mejorar estos conocimientos básicos \cite{Bibl015}; \cite{Bibl016}. En suma, aumentar la alfabetización financiera implica empoderar al usuario para enfrentar decisiones financieras cotidianas y extraordinarias con mayor seguridad y criterio.

\paragraph*{Educación financiera}
\addcontentsline{toc}{paragraph}{Educación financiera}
La educación financiera es el proceso mediante el cual los individuos mejoran su comprensión sobre productos financieros, conceptos y riesgos, desarrollando habilidades y confianza para gestionar sus finanzas. Según la \textit{OCDE (2005)}, este proceso combina información, instrucción y/o asesoría objetiva, de modo que las personas se hagan más conscientes de las oportunidades y riesgos financieros, aprendan dónde buscar ayuda y tomen acciones efectivas para mejorar su bienestar económico. En la práctica, la educación financiera básica incluye la enseñanza de presupuestos, ahorro, inversión sencilla y manejo de deuda, usando lenguajes adecuados al público objetivo \cite{Bibl017}. Por ello, un servicio de planificación financiera para personas sin conocimientos previos debe incorporar componentes educativos claros y accesibles, que faciliten el aprendizaje continuo de conceptos financieros fundamentales \cite{Bibl015}.

\paragraph*{Inclusión financiera}
\addcontentsline{toc}{paragraph}{Inclusión financiera}
La inclusión financiera se define como el proceso de promover el acceso asequible, oportuno y adecuado a una amplia gama de productos y servicios financieros regulados, ampliando su uso entre todos los segmentos de la sociedad. Dicho de otro modo, la inclusión financiera busca que individuos y negocios utilicen productos financieros (ahorro, crédito, pagos, seguros, etc.) que respondan a sus necesidades y les permitan gestionar riesgos y oportunidades \cite{Bibl034}. Este concepto cobra especial relevancia en poblaciones con baja alfabetización: la inclusión no solo abarca acceso físico a los bancos, sino también a herramientas y educación financiera que habiliten un uso efectivo de dichos servicios. Al integrar la inclusión financiera en el proyecto, se enfatiza la necesidad de reducir barreras (económicas, educativas y tecnológicas) para que más personas participen plenamente en la economía formal, mejorando así su estabilidad económica y posibilidades de crecimiento \cite{Bibl014}.

\paragraph*{Tecnología financiera (FinTech)}
\addcontentsline{toc}{paragraph}{Tecnología financiera (FinTech)}
El término \textit{FinTech} (tecnología financiera) se refiere al uso de la tecnología y la innovación para proporcionar productos y servicios financieros. En la práctica, esto incluye desde aplicaciones móviles de banca y presupuestos hasta plataformas de asesoría automática (robo-advisors) o sistemas de pagos digitales. Las herramientas FinTech facilitan el acceso a información financiera en tiempo real y permiten servicios personalizados a bajo costo, lo que es clave para usuarios con poco conocimiento previo \cite{Bibl018}. Por ejemplo, una app de finanzas personales puede mostrar gráficos simples de gastos o enviar alertas de presupuesto, reduciendo la complejidad técnica. En el contexto del servicio propuesto, integrar tecnología financiera significa aprovechar plataformas digitales (apps, simuladores, chatbots, etc.) para apoyar la educación financiera y el asesoramiento personalizado, haciendo el aprendizaje más interactivo y accesible \cite{Bibl019}; \cite{Bibl020}.

\paragraph*{Bienestar financiero}
\addcontentsline{toc}{paragraph}{Bienestar financiero}
El bienestar financiero se describe como un estado en el que la persona puede cumplir completamente con sus obligaciones financieras presentes y futuras, se siente segura respecto a su futuro económico y es capaz de tomar decisiones que le permitan disfrutar de la vida. Es decir, no se trata solo de tener ingresos, sino de tener control sobre el dinero día a día, capacidad para enfrentar imprevistos (fondo de emergencia) y libertad para alcanzar metas personales. Este concepto enfatiza tanto aspectos objetivos (cumplir pagos, niveles de ahorro o inversión) como subjetivos (confianza y satisfacción con la situación económica). Para el servicio de planificación, mejorar el bienestar financiero del usuario implica fortalecer su estabilidad económica doméstica: por ejemplo, ayudándole a organizar un presupuesto realista, ahorrar para emergencias y entender opciones de inversión seguras. De este modo, el proyecto contribuye a que los usuarios sientan una mejora concreta en su tranquilidad y autonomía financiera \cite{Bibl021}; \cite{Bibl022}.

\paragraph*{Presupuesto personal}
\addcontentsline{toc}{paragraph}{Presupuesto personal}
El presupuesto personal es un plan financiero que detalla cómo se asignarán los ingresos a distintas categorías de gastos y ahorros durante un periodo determinado. En otras palabras, es una herramienta que permite visualizar claramente la relación entre ingresos, gastos y ahorro, facilitando la toma de decisiones económicas. Elaborar un presupuesto personal ayuda a identificar gastos innecesarios, asignar una parte de los ingresos al ahorro programado (por ejemplo, para emergencias o metas a largo plazo) y planificar prioridades financieras (como pago de deudas o inversiones). 

Para personas con bajo nivel financiero, aprender a hacer un presupuesto es fundamental: les proporciona control sobre sus finanzas cotidianas y les enseña hábitos de ahorro y planificación. En la propuesta de servicio, orientar al usuario en la elaboración de su presupuesto digital (usando apps o plantillas interactivas) es clave para mejorar su salud financiera desde la práctica.


El servicio de \textbf{planificación financiera dirigido a personas sin conocimientos en finanzas} enfrenta un entorno competitivo complejo. En Colombia \textbf{la banca tradicional} sigue dominando el ecosistema financiero con extensa cobertura y recursos (cuentas de ahorro, crédito, asesoría básica), pero su oferta educativa suele ser genérica. En contraste, el floreciente sector \textbf{fintech} –con alrededor de \textit{394 startups en 2024 y 410 en 2025} está captando cuota de mercado con soluciones digitales innovadoras. Neobancos y billeteras digitales (por ejemplo, Nequi y Daviplata) han crecido exponencialmente, Nequi reportó \textbf{más de 20 millones de usuarios (15 millones activos)} en agosto de 2024, y Daviplata suma decenas de millones de descargas. 

\begin{figure}[H]
	\centering
	\includegraphics[width=10cm]{Imagenes/LogoNequi.png}
	\caption[{Logotipo de Nequi.}]{\centering Logotipo de Nequi. \textit{Fuente:} Sitio Web: (No son sus datos, Nequi está fallando, Alerta Caribe, 2025)} 
	\label{fig:logonequi}
\end{figure}

Otras fintech relevantes incluyen \textit{Addi, Nu Colombia (Nubank), RappiPay}, que ofrecen créditos de consumo, pagos y ahorro a través de apps. Además, hay aplicaciones educativas y de presupuesto personal: por ejemplo, el startup colombiano \textbf{Finkü} ofrece un gestor de finanzas personales y talleres de educación financiera, y la fundación WWB lanzó la app \textbf{Miga} enfocada en presupuesto y ahorro personal. Finalmente, entidades gubernamentales y ONG participan: la Banca de las Oportunidades impulsa la educación financiera desde la infancia (p. ej. “Global Money Week 2025” promovida por la OCDE), y programas de inclusión (ICETEX, Innpulsa) integran asesorías financieras básicas.

La \textbf{posición de los actores actuales} varía según el segmento. Los grandes bancos combinan canales presenciales y digitales, pero su orientación suele privilegiar productos tradicionales sobre contenidos formativos. En cambio, las fintech se apoyan en tecnología móvil y datos para llegar a públicos jóvenes y no bancarizados. Estudios recientes señalan que “las empresas fintech operan en ámbitos como la banca móvil, los pagos, gestión de finanzas personales, etc”, contribuyendo a la inclusión financiera. Asimismo, informes regulatorios destacan la creación de marcos de open banking en Colombia que facilitan la oferta de nuevos servicios basados en datos financieros. 

Sin embargo, persisten \textbf{barreras de entrada} importantes: existe alta concentración en el sector crédito y depósitos. Por ejemplo, en microcrédito un solo banco controla más del 50\% del mercado, lo que eleva barreras para nuevas fintech y puede encarecer el acceso al crédito. Esta concentración limita la competencia (a pesar de políticas para fomentarla), y aumenta los costos de adquisición de usuarios para un emprendimiento nuevo. Otras barreras incluyen la regulación financiera (licencias de la SFC) y la \textbf{brecha digital}: aunque el 65,6\% de los hogares tenía internet en 2024, muchos potenciales clientes carecen de acceso o habilidades digitales. En síntesis, el competitivo local abarca tanto a jugadores consolidados (bancos tradicionales con amplia red y clientela) como a fintech ágiles con propuestas específicas de valor. Para posicionarse será clave diferenciarse mediante la educación financiera, la personalización y alianzas estratégicas.