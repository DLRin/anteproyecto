\subsection*{Marco referencial}
\addcontentsline{toc}{subsection}{Marco referencial}
El proyecto se enmarca en la literatura y las políticas actuales de \textbf{inclusión financiera y educación económica}. Diversos estudios señalan que la alfabetización financiera es baja entre la población colombiana, lo cual motiva la necesidad de un servicio de planificación para no expertos. Por ejemplo, un análisis del Banco de la República (2024) encontró que solo el 16,4\% de los encuestados respondió correctamente a tres preguntas básicas sobre interés, inflación y diversificación del riesgo. Menos de la mitad comprendió el cálculo de intereses o los beneficios de diversificar inversiones, mientras que un 79\% entendió bien el efecto de la inflación. Estas cifras evidencian brechas de conocimiento financiero que justifican los servicios educativos.

La \textbf{OECD} y organismos internacionales fomentan la educación financiera como instrumento de desarrollo. Colombia es miembro de la OECD y participa activamente en eventos como la \textit{Global Money Week}, impulsando programas educativos desde la infancia. Además, en 2022 se promulgó el \textit{Decreto 1297} y otras normativas de \textit{Open Banking} que facilitan la oferta de productos centrados en el usuario. Según la OCDE, estas reformas buscan \textit{“mejorar la inclusión financiera”} y alentar la competencia mediante APIs abiertas y sandboxes (por ejemplo, InnovaSFC). En este marco, la empresa proyectada se alinea con tendencias globales: ofrecerá un servicio innovador apoyado en tecnología de punta para empoderar a usuarios inexpertos.

La \textbf{teoría económica} relevante incluye conceptos de finanzas conductuales y ciclos de vida del consumidor. Sabemos que el 66\% de los programas de educación financiera en Colombia aplican “nudges” conductuales (recordatorios de ahorro, mensajes positivos) para fomentar buenos hábitos. Nuestro enfoque puede incorporar dichas estrategias (por ejemplo, notificaciones de ahorros regulares). Además, la \textbf{ley de inclusión financiera} de 2014 buscó “aprovechar la digitalización para fomentar el acceso” a servicios financieros, por tanto, existe una base legal y social que apoya iniciativas de este tipo.

A nivel local, la \textbf{bibliografía académica} y de política pública ha resaltado la relación entre educación y bienestar financiero. Estudios de Asobancaria documentan que un mayor grado educativo correlaciona con mejor manejo de deudas y planificación, y que la inclusión financiera (68\% de encuestados con al menos un producto bancario) no implica automáticamente capacidad de manejo de crédito. Por ello, el marco referencial incorpora tanto datos estadísticos (DANE, BanRep, Asobancaria) como enfoques conceptuales: la importancia de la planificación familiar (p. ej. presupuesto y ahorros a largo plazo) en economías con alta informalidad y vulnerabilidad socioeconómica.

Finalmente, actores sectoriales como la \textbf{superintendencia y gremios} han llamado la atención sobre los riesgos de sobreendeudamiento. El nuevo plan de gobierno financiero exige que los ciudadanos cuenten con “al menos un producto y conocimiento suficiente para usarlo”. De hecho, plataformas estudiantiles como One2Credit impulsan la idea de que el financiamiento educativo debe ir acompañado de contenidos de gestión financiera. En consecuencia, el servicio de planificación proyectado se fundamenta en evidencia empírica y en mejores prácticas: combinará asesoría personalizada con elementos pedagógicos para cerrar las brechas detectadas en los estudios citados.
En suma, el marco referencial del negocio toma en cuenta que \textit{``la educación financiera desde la infancia facilita la adopción de hábitos responsables''}, y que un mayor conocimiento personal de finanzas impacta positivamente en el bienestar socioeconómico. La iniciativa propuesta busca alinearse con esas políticas públicas y apoya la visión de la OCDE de que \textit{tomar decisiones informadas es clave para el bienestar financiero}.