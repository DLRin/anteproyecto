\section*{Anexos}
\addcontentsline{toc}{section}{Anexos}

\begin{figure}[H]
	\centering
	\includegraphics[width=12cm]{Imagenes/MetodoCuadros.png}
	\caption[{Método de los cuadros.}]{\centering Método de los Cuadros. \textit{Fuente:} Autores.}
	\label{fig:metodocuadros}
\end{figure}

\begin{adjustbox}{
		center,
		caption=[{Matriz de problemas y objetivos.}]{\centering Matriz de problemas y objetivos. \textit{Fuente:} Autores.},
		label={MatrizProblemasObjetivos},
		nofloat=table, vspace={20px}}
	\resizebox{\textwidth}{!}{
		\begin{tabular}{|c|c|}
			\hline
			\rowcolor[HTML]{F5B8B8}
			\multicolumn{2}{|c|}{\begin{tabular}[c]{@{}c@{}} \textbf{Matriz de problemas y objetivos}\end{tabular}} \\ \hline
			\rowcolor[HTML]{FAD4D4} 
			\textbf{Problemas} & \textbf{Objetivos} \\ \hline
			\rowcolor[HTML]{F7E9E9} 
			\textit{Problema General} & \textit{Objetivo General} \\ \hline
			\begin{tabular}[c]{@{}c@{}}¿Cómo se puede abordar la falta de acceso a servicios de \\ planificación financiera personalizados y comprensibles \\ para personas sin conocimientos financieros en Bogotá? \end{tabular} & \begin{tabular}[c]{@{}c@{}}Diseñar e implementar un plan de negocios para un servicio de \\ planificación financiera el cual facilite la educación financiera y el \\ acceso a herramientas y asesoría  personalizada para personas \\ sin conocimientos de finanzas en  Bogotá.\end{tabular} \\ \hline
			\rowcolor[HTML]{F7E9E9} 
			\textit{Problemas Específicos} & \textit{Objetivos Específicos} \\ \hline
			\begin{tabular}[c]{@{}c@{}}¿Cómo superar la baja alfabetización financiera que \\ limita a las personas a tomar decisiones económicas \\ informadas?\end{tabular} & \begin{tabular}[c]{@{}c@{}} Incrementar la alfabetización mediante uso de herramientas educativas \\ claras, simuladores y asesoría adaptadas a sus necesidades \end{tabular} \\ \hline
			\begin{tabular}[c]{@{}c@{}}¿Cómo diseñar un servicio accesible y adaptado que \\ incluya asesoría y herramientas digitales para ayudar a la \\ gestión  financiera personal?\end{tabular} & \begin{tabular}[c]{@{}c@{}} Desarrollar una plataforma digital amigable la cual tenga asesoramiento \\ financiero  personalizado para mejorar el manejo de \\ presupuesto, ahorro y deudas. \end{tabular} \\ \hline
			\begin{tabular}[c]{@{}c@{}}¿Cómo disminuir la exclusión financiera de grupos vulerables \\ como jóvenes, trabajadores independientes e informales y \\ personas en  la economía de Bogotá\end{tabular} & \begin{tabular}[c]{@{}c@{}} Facilitar la inclusión financiera de grupos vulnerables al \\ ofrecer un servicio que atienda a sus particularidades superando \\ barreras tecnológicas. \end{tabular} \\ \hline
		\end{tabular}
	}
\end{adjustbox}