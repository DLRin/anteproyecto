\section*{Planteamiento del Problema}
\addcontentsline{toc}{section}{Planteamiento del Problema}
La falta de alfabetización financiera limita la capacidad de los individuos para enfrentar la complejidad de productos bancarios, créditos e inversiones. Estudios nacionales resaltan esta carencia, debido a los bajos niveles de educación financiera en el país, evidenciados por las altas tasas de interés que pagan los hogares. El desconocimiento y desinformación generalizados sobre temas económicos y financieros impiden a la ciudadanía tomar decisiones fundamentadas, afectando negativamente su bienestar personal y familiar \cite{Bibl001}. Ante este panorama, organismos oficiales como la Banca de las Oportunidades destacan la necesidad de elevar los conocimientos financieros de la población para que “tomen decisiones informadas y responsables” en materia de ahorro, gasto e inversión.

En el caso de la ciudad de Bogotá se observa un panorama crítico dentro del bajo promedio nacional. Según la encuesta de carga financiera de 2009, aplicada a hogares bancarizados de la capital, encontró que apenas el 0,76\% de los encuestados respondió correctamente las preguntas del módulo de educación financiera, concluyendo que hay un bajo nivel de educación financiera en la población bogotana bancarizada.
Este déficit en formación financiera se traduce en prácticas riesgosas tales como el sobreendeudamiento y el uso de créditos informales, por lo que se ha subrayado la conveniencia de complementar las políticas públicas con servicios especializados. En este sentido, aunque la estrategia nacional promueve educación financiera para mejorar la toma de decisiones, los datos señalan que en Bogotá es urgente ofrecer un servicio profesional de planificación financiera que oriente a las familias en la estructuración de presupuestos, el ahorro estratégico y las inversiones seguras. De ese modo se busca cerrar la brecha de alfabetización financiera y fortalecer la estabilidad económica de los hogares bogotanos \cite{Bibl002}.

\begin{figure}[H]
	\centering
	\includegraphics[width=10cm]{Imagenes/ArboldeProblemas.png}
	
	\caption[{Árbol del problema}]{\centering Árbol del problema. \textit{Fuente:} Autores.} 
	\label{fig:arboldeproblemas}
\end{figure}

\subsection*{Descripción del Problema}
\addcontentsline{toc}{subsection}{Descripción del Problema}
En Colombia, el acceso a servicios financieros ha avanzado en cobertura, claramente observado por el porcentaje de adultos que tienen algún producto financiero, el cual es del 94.6\% en el año 2023. Con más de 30.8 millones de adultos teniendo al menos una cuenta de ahorro para ese mismo año \parencite{BancadeOportunidades01}.

Sin embargo, persiste una brecha notable en la comprensión y gestión efectiva del manejo del dinero en el tiempo, a través de un análisis de comparación hecho por el Banco Mundial, bajo una escala de 0 (menor capacidad) a 100 (mayor capacidad) se encontró que actividades relacionadas con respecto al ahorro y el seguimiento de los gastos obtuvieron los puntajes más bajes mientras que la capacidad de cubrir los gastos imprevistos y la impulsividad se encontraron en el medio \parencite[p.~34-35]{BancoMundial01}. Adicionalmente, con respecto a conocimientos financieros, en aquel mismo informe, se evidencio que las personas no eran capaces de hacer un cálculo de tasa de interés (con una tasa de éxito del 35\%) y solamente el 26\% lograron responder sobre el concepto de interés compuesto. Por lo que esta falta de compresión hace dudar de la capacidad de las personas para tomar decisiones totalmente conscientes sobre productos financieros \parencite[p.~29]{BancoMundial01}.

%pendiente por redactar%
%Desde la perspectiva de la Ingeniería de Sistemas, esta problemática se agrava por la falta de ecosistemas digitales que integren herramientas de \textit{adaptive learning} y personalización basada en datos, lo que impide una cultura de ahorro estratégica.%

Esta situación, se ve agravada por la ausencia de una plataforma digital que combine las condiciones actuales del usuario, la falta de herramientas interactivas y la personalización del usuario en sí, por lo que, esta carencia contribuye a la falta de seguimiento, ahorro y manejo de conceptos financieros. Por lo cual se vuelve fundamental abordar esta problemática para buscar fomentar una cultura de ahorro personalizada y estratégica.

\input{Contenido/02. Planteamiento/FormulacionDelProblema}

\subsection*{Justificación del Problema}
\addcontentsline{toc}{subsection}{Justificación del Problema}
El panorama de inclusión financiera en Colombia muestra avances cuantitativos importantes pero también persistencia de notables brechas. Hoy en día alrededor del 95\% de los adultos colombianos reportan tener al menos un producto financiero formal (cuentas, depósitos o crédito), pero esa cobertura varía dramáticamente según el contexto socioeconómico y regional. En las zonas urbanas principales (ciudades y municipios intermedios, incluyendo Bogotá), mientras que en los municipios rurales apenas ronda el 65–70\%, lo que implica una brecha urbano-rural de casi 34 puntos porcentuales \cite{Bibl002}.

A su vez, las mujeres adultas tienen niveles de acceso inferiores a los hombres, aunque la diferencia se ha ido reduciendo con los años; por ejemplo, en 2023 el 97,7\% de los hombres adultos tenía algún producto financiero, frente a 91\% de mujeres, dejando una brecha de 6,6 puntos porcentuales \cite{Bibl004}.

Estas disparidades se agravan por la alta informalidad laboral: según el DANE en el primer trimestre de 2025 el 57,2\% de los ocupados a nivel nacional estaban en la informalidad y ese porcentaje superaba el 84\% en las zonas rurales \cite{Bibl005} mientras que en las 13 principales ciudades (entre ellas Bogotá) la informalidad alcanzó el 43,1\%.

En la práctica, los trabajadores informales o con bajos ingresos, sin acceso permanente a la seguridad social ni a canales formales de crédito tienen muchas limitaciones para abrir cuentas, acceder a microcréditos o seguros, lo que reduce aún más su inclusión financiera efectiva.

Por otra parte, la alfabetización financiera de la población colombiana es muy baja, lo que dificulta que los usuarios conozcan y utilicen bien los productos formales disponibles. Un estudio reciente del Banco de la República basado en una encuesta nacional encontró que apenas el 16,4\% de los colombianos adultos responde correctamente un conjunto básico de preguntas de conocimiento financiero \cite{Bibl006}.

Este indicador es particularmente bajo en jóvenes, en hogares de menores ingresos o nivel educativo, y en personas dedicadas al trabajo informal, lo cual confirma que la falta de entendimiento de conceptos financieros es una barrera importante para la inclusión. No sorprende entonces que, según esa investigación, existe una ``urgencia de desarrollar políticas dirigidas a incrementar la educación económica y financiera, particularmente en grupos vulnerables, de tal manera que les permita a las personas ahorrar, invertir, endeudarse menos y asegurar su patrimonio''. En otras palabras, mejorar la capacidad de ahorro y manejo del presupuesto personal es clave para que más personas se incorporen al sistema formal y aprovechen sus beneficios.

Ante este escenario, la implementación de un servicio de planificación financiera personalizado en Bogotá se presenta como una solución viable y necesaria. Al enfocarse en usuarios con bajo o nulo conocimiento financiero, dicho servicio podría explicarles de manera clara cómo elaborar un presupuesto, manejar deudas, ahorrar e invertir de acuerdo con su realidad, todo ello por un profesional de confianza. Así se cerraría parcialmente la brecha generacional de conocimiento mencionada, pues la evidencia muestra que mejorar la educación financiera tiene impacto directo en comportamientos más saludables de gasto y ahorro \cite{Bibl007}. Además, al apoyar a quienes trabajan en la economía informal a comprender los productos formales, se reducirían las barreras prácticas de acceso y se fomentaría la bancarización efectiva.