\subsection*{Justificación del Problema}
\addcontentsline{toc}{subsection}{Justificación del Problema}
El panorama de inclusión financiera en Colombia muestra avances cuantitativos importantes pero también persistencia de notables brechas. Hoy en día alrededor del 95\% de los adultos colombianos reportan tener al menos un producto financiero formal (cuentas, depósitos o crédito), pero esa cobertura varía dramáticamente según el contexto socioeconómico y regional. En las zonas urbanas principales (ciudades y municipios intermedios, incluyendo Bogotá), mientras que en los municipios rurales apenas ronda el 65–70\%, lo que implica una brecha urbano-rural de casi 34 puntos porcentuales \cite{Bibl002}.

A su vez, las mujeres adultas tienen niveles de acceso inferiores a los hombres, aunque la diferencia se ha ido reduciendo con los años; por ejemplo, en 2023 el 97,7\% de los hombres adultos tenía algún producto financiero, frente a 91\% de mujeres, dejando una brecha de 6,6 puntos porcentuales \cite{Bibl004}.

Estas disparidades se agravan por la alta informalidad laboral: según el DANE en el primer trimestre de 2025 el 57,2\% de los ocupados a nivel nacional estaban en la informalidad y ese porcentaje superaba el 84\% en las zonas rurales \cite{Bibl005} mientras que en las 13 principales ciudades (entre ellas Bogotá) la informalidad alcanzó el 43,1\%.

En la práctica, los trabajadores informales o con bajos ingresos, sin acceso permanente a la seguridad social ni a canales formales de crédito tienen muchas limitaciones para abrir cuentas, acceder a microcréditos o seguros, lo que reduce aún más su inclusión financiera efectiva.

Por otra parte, la alfabetización financiera de la población colombiana es muy baja, lo que dificulta que los usuarios conozcan y utilicen bien los productos formales disponibles. Un estudio reciente del Banco de la República basado en una encuesta nacional encontró que apenas el 16,4\% de los colombianos adultos responde correctamente un conjunto básico de preguntas de conocimiento financiero \cite{Bibl006}.

Este indicador es particularmente bajo en jóvenes, en hogares de menores ingresos o nivel educativo, y en personas dedicadas al trabajo informal, lo cual confirma que la falta de entendimiento de conceptos financieros es una barrera importante para la inclusión. No sorprende entonces que, según esa investigación, existe una ``urgencia de desarrollar políticas dirigidas a incrementar la educación económica y financiera, particularmente en grupos vulnerables, de tal manera que les permita a las personas ahorrar, invertir, endeudarse menos y asegurar su patrimonio''. En otras palabras, mejorar la capacidad de ahorro y manejo del presupuesto personal es clave para que más personas se incorporen al sistema formal y aprovechen sus beneficios.

Ante este escenario, la implementación de un servicio de planificación financiera personalizado en Bogotá se presenta como una solución viable y necesaria. Al enfocarse en usuarios con bajo o nulo conocimiento financiero, dicho servicio podría explicarles de manera clara cómo elaborar un presupuesto, manejar deudas, ahorrar e invertir de acuerdo con su realidad, todo ello por un profesional de confianza. Así se cerraría parcialmente la brecha generacional de conocimiento mencionada, pues la evidencia muestra que mejorar la educación financiera tiene impacto directo en comportamientos más saludables de gasto y ahorro \cite{Bibl007}. Además, al apoyar a quienes trabajan en la economía informal a comprender los productos formales, se reducirían las barreras prácticas de acceso y se fomentaría la bancarización efectiva.