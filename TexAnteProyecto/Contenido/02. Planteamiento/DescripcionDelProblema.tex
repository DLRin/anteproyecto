\subsection*{Descripción del Problema}
\addcontentsline{toc}{subsection}{Descripción del Problema}
En Colombia, el acceso a servicios financieros ha avanzado en cobertura, alcanzando un 72.38\% de la población bancarizada en 2023, pero si embargo persiste una brecha notable en la comprensión y gestión efectiva del manejo del dinero en el tiempo, debido a que 68.45\% de los usuarios no comprende conceptos básicos como tasas de interés o planes de amortización. Esto causa indicadores alarmantes, el 17.874\% de los créditos de consumo presentan morosidad, solo el 11.278\% de los hogares invierten sus recursos financieros de manera formal, y el 61.98\% carece de un presupuesto estructurado. 

La raíz del problema radica en la exclusividad de los asesores tradicionales, vinculados a entidades bancarias, que atienden apenas al 9.75\% de la población de bajos ingresos, dejando fuera a grupos prioritarios como jóvenes (el 41.337\% de los desempleados son menores de 30 años, trabajadores independientes (38.59\% de la fuerza laboral) y emprendedores, quienes requieren orientación práctica para gestionar flujos de caja o historiales crediticios. Paralelamente, las iniciativas gubernamentales, como campañas de educación financiera, solo alcanzan una efectividad del 23.608\% según métricas de retención de conocimiento, debido a su enfoque genérico y desconexión con realidades regionales (el 67.8\% de los usuarios en zonas rurales desconocen estos programas). Esta brecha evidencia la necesidad de una plataforma digital que combine ``micro elecciones interactivas'' (basadas en algoritmos), simuladores de escenarios financieros y ``asesoría en tiempo real'' mediante herramientas interactivas con el usuario, diseñada fomentar una cultura de ahorro estratégico y la inversión con rentabilidad.