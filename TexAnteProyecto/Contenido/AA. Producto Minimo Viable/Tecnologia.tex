\subsection*{Tecnología}
\addcontentsline{toc}{subsection}{Tecnología}
Para lograr el desarrollo y la construcción de un prototipo mínimo viable, se comenzará a partir de una versión inicial, esto con el propósito de que se pueda cumplir con las funciones básicas del programa tales como asesoría financiera y el manejo de material multimedia. Esto teniendo en cuenta las necesidades de nuestra plataforma digital, entre ellas, se va a tener en cuenta una arquitectura de tres capas bajo las siguientes tecnologías.

\begin{figure}[H]
	\centering
	\includegraphics[width=15cm]{Imagenes/ArquiTresCapas.png}
	\caption[{Arquitectura de Tres Capas.}]{\centering Arquitectura de Tres Capas. \textit{Fuente:} Sitio Web: (Gaviria V. Raúl A., ResearchGate, 2018)} 
	\label{fig:logobconomy}
\end{figure}

Para el \textbf{Frontend} se pensó en usar la biblioteca de JavaScript React, el cual logra facilitar una gran cantidad de herramientas para crear interfaces de usuario (UI's) mejorando de esta forma la experiencia del usuario mientras use nuestra plataforma, para el nivel de la \textbf{aplicación}, se encuentra en conversaciones sobre si usar el lenguaje de programación Python o Java para la comunicación entre el Frontend, Backend, el alojamiento y el despliegue, también se pensó en alternativas tales como Netlify. 

Y finalmente, para el \textbf{Backend} y almacenamiento de datos se encuentra en conversaciones en usar bases de datos tradicionales como MySQL o PostgreSQL como otras herramientas como FaunaDB buscando combinar herramientas para crear una aplicación moderna y escalable.