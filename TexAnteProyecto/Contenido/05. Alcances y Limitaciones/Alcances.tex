\subsection*{Alcances}
\addcontentsline{toc}{subsection}{Alcances}
El \textbf{alcance} del proyecto se centra en elaborar un plan de negocios completo para un servicio de planificación financiera digital dirigido a usuarios sin conocimientos financieros en Bogotá. Esto incluye la definición detallada de un servicio con herramientas interactivas (simuladores financieros y calculadoras personalizadas) para gestionar presupuesto, deudas, ahorro e inversión. Como parte del alcance se realizará un exhaustivo análisis de mercado que identifique brechas en educación financiera, especialmente en segmentos críticos como jóvenes y trabajadores independientes. Además, se establecerá un modelo de negocio sólido (propuesta de valor, canales de distribución, fuentes de ingresos, estrategia de crecimiento). Finalmente, el alcance incluye proyectar los recursos técnicos, humanos y financieros necesarios y realizar un análisis de viabilidad económica en horizontes de corto, mediano y largo plazo.

\begin{itemize}
	\item Diseño del plan de negocios para un servicio de asesoría financiera digital en Bogotá, dirigido a usuarios con bajo nivel financiero.
	
	\item Análisis de mercado para identificar necesidades y brechas de los segmentos meta (jóvenes, trabajadores independientes, etc.) en materia de educación financiera.
	
	\item Definición del modelo de negocio: propuesta de valor, canales de distribución, fuentes de ingreso y estrategia de crecimiento sostenible.
	
	\item Proyección de recursos técnicos, humanos y económicos necesarios, además de un análisis financiero que evalúe la viabilidad del proyecto en el corto, mediano y largo plazo.
\end{itemize}

Estos alcances delimitan claramente lo que se abordará en el proyecto y lo que queda fuera. Por ejemplo, se abarca la concepción y planeación del servicio, pero \textbf{no se ejecuta} el desarrollo real de la plataforma ni la implementación de las herramientas, lo cual queda fuera del alcance académico del plan de negocios (esto es parte del alcance negativo o ``lo que no se hará'' en el proyecto). Al definir estos límites se previene la confusión de requisitos y se alinea el equipo con los objetivos esperados.