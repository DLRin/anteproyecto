\subsection*{Análisis de competencia}
\addcontentsline{toc}{subsection}{Análisis de competencia}

\begin{itemize}
	\item \textbf{Banca tradicional:} Bancolombia, BBVA, Davivienda, Banco de Bogotá, entre otros, concentran la mayor parte de los activos financieros del país. Tienen extensa presencia física y cartera de clientes masiva. Ofrecen programas de educación financiera (seminarios, talleres) y apps bancarias básicas (p. ej., Resuelve de Bancolombia), pero sus servicios de asesoría suelen estar orientados a clientes con productos contratados. Aunque tienen escalas y confianza de marca, su modelo no está diseñado específicamente para ``novatos'' financieros.
	\item \textbf{Fintech y neobancos:} El ecosistema fintech crece rápidamente (el número de fintech aumentó 5,8\% en 2024). Los neobancos digitales son competidores directos para segmentos jóvenes y tecnificados. Por ejemplo, \textbf{Nequi} (Grupo Bancolombia) y \textbf{Daviplata} (Davivienda) han “revolucionado la manera en que los colombianos manejan su dinero” usando aplicaciones móviles, Nequi tiene más de 20 millones de usuarios, y Daviplata superó los 10 millones de descargas según datos de mercado. Otras fintech locales como \textit{Addi} (crédito de consumo y POS virtual) o \textit{RappiPay} (servicios de pago del aplicativo Rappi) han crecido con atractivas propuestas. Además, nuevas neobancos globales, como \textbf{Nubank}, han entrado al mercado local. 
	
	Estas fintech buscan atender a usuarios subatendidos: por ejemplo, ofrecen financiación al consumo con procesos 100\% digitales, o permiten transacciones sin comisiones. Muchas incluyen funcionalidades de planificación básica: alertas de pago, objetivos de ahorro, e incluso planeación de cuotas de crédito. Según un análisis de Colombia Fintech, plataformas dirigidas a estudiantes universitarios integran \textit{“educación, inclusión financiera y herramientas digitales”}, con énfasis en \textit{planificación financiera y autogestión}.
	\item \textbf{Aplicaciones educativas y consultorías digitales:} Existen apps específicamente diseñadas para la educación y gestión financiera personal. Ejemplo: \textbf{Finkü} (fintech colombiana) y \textbf{Miga} (de WWB) promueven el registro de ingresos/gastos y brindan contenido didáctico; Finkü, por ejemplo, combina registro de gastos con talleres prácticos y premia hábitos de ahorro. Otras iniciativas incluyen cursos online gratuitos, blogs y plataformas de ONG (p. ej. \textit{Aprende+} del gobierno, contenidos de Bancolombia y BBVA). Aunque aún son de nicho, estas herramientas impactan la capacitación financiera del consumidor básico.
	\item \textbf{Competencia potencial y aliados:} Adicionalmente, hay proveedores de software global (por ej. \textbf{Personal Capital, Mint}) que podrían ingresar, y plataformas de inversión móvil. Sin embargo, su penetración en Colombia es limitada. Por otro lado, asociaciones gremiales (Asobancaria, ColombiaFintech) y organismos estatales impulsan políticas de inclusión financiera (leyes de inclusión digital, sandboxes regulatorios). Un plan de negocios exitoso debe considerar tanto la rivalidad actual como la posibilidad de alianzas (p. ej. vincularse con fundaciones, bancos o corporaciones de tecnología).
\end{itemize}

En abstracción, el mercado local de servicios financieros es competitivo y heterogéneo. Si bien los grandes bancos proveen estabilidad y amplia cobertura, las fintech han captado nichos clave mediante innovación y atención al cliente digital. Las barreras de entrada incluyen una competencia concentrada y requerimientos regulatorios, así como la necesidad de ganar confianza de consumidores sin cultura financiera. Por ello, ofrecer \textbf{valor diferencial} (asesoría personalizada, guías simples, tecnología amigable) será esencial para destacar frente a la banca tradicional, otros neobancos y apps de gestión financiera.