\subsection*{Análisis económico}
\addcontentsline{toc}{subsection}{Análisis económico}

El contexto macroeconómico de Colombia (y en particular de Bogotá) en 2024–2025 se caracteriza por crecimiento moderado, inflación en descenso y mercado laboral en recuperación. Según el Banco de la República, el \textbf{PIB nacional} creció cerca de \textit{1,8–2,1\% anual} en 2024–2025. En Bogotá D.C., motor económico del país, el PIB aumentó \textbf{2,8\%} en el cuarto trimestre de 2024 respecto al año anterior. Estas cifras indican una recuperación económica, pero aún por debajo de la expansión previa a la pandemia. Los sectores con mayor impulso han sido comercio al por mayor/minorista, transporte y alojamiento.

En materia de \textbf{inflación}, tras alcanzar un pico del 9,28\% en 2023, Colombia ha experimentado una significativa moderación. El año 2024 cerró con una inflación anual de \textit{5,20\%}, por debajo del rango meta y previendo futuros ajustes de precios. A mediados de 2025 la inflación anual rondaba el \textit{4,8–5,0\%}, acercándose al rango meta del Banco Central (2–4\%). En respuesta, la tasa de interés de política se redujo a \textbf{10,25\%} en septiembre de 2024, buscando incentivar la reactivación. Este entorno de costos financieros moderados favorece la oferta de crédito de consumo y ahorro (una ventaja para servicios financieros que requieran fondeo).

El \textbf{mercado laboral} muestra tendencia positiva. La tasa nacional de desocupación cayó al \textbf{8,8\% en julio de 2025}, nivel más bajo desde 2001. En Bogotá y áreas metropolitanas ese índice también disminuyó (aproximadamente 8,4\% en julio 2025). Sin embargo, la informalidad se mantiene alta (54,8\% nacional en julio 2025). Para nuestro servicio dirigido a personas sin conocimientos financieros, este contexto es relevante: un gran segmento de la población (particularmente informal y de ingresos bajos) necesita herramientas de gestión de dinero y acceso a crédito sencillo. El ingreso laboral promedio en Bogotá supera en un 18\% al nacional (según DNP), pero la heterogeneidad es amplia. El salario mínimo aumentó 9,5\% para 2025, superando la inflación, lo cual, si bien mejora el poder adquisitivo de trabajadores informales y estratos bajos, podría presionar costos de servicios (incluidos los financieros).

En cuanto a \textbf{pobreza y acceso}, la pobreza monetaria alcanzó el \textit{31,8\%} en 2024, un mínimo histórico, indicando que casi dos tercios de la población supera la línea de pobreza. La inclusión financiera es alta: para 2024, alrededor del 95–96\% de adultos tenía al menos un producto bancario o de crédito. Sin embargo, esta inclusión no siempre se traduce en conocimiento financiero o planificación: estudios de salud financiera indican brechas considerables (ver ``Marco referencial''). Por otro lado, la conectividad digital ha mejorado, pero todavía \textbf{solo el 65,6\% de los hogares tenía acceso a internet en 2024}. En Bogotá la cifra es mayor, pero la brecha rural-urbana persiste. Esto implica que un plan de negocio basado en tecnología requiere estrategias para educación y soporte offline/in-persona en poblaciones vulnerables.

Finalmente, las \textbf{condiciones del sistema financiero local} son dinámicas. Se observa un fuerte impulso a la digitalización (con iniciativas de \textit{open finance} y Banca Abierta aprobadas en 2022), lo que abre oportunidades para nuevos modelos de negocio basados en datos. Por ejemplo, Nequi reporta funcionalidades avanzadas (análisis de gastos, metas de ahorro, “Resumen Nequi”) para sus usuarios, aprovechando el ecosistema digital. No obstante, la alta concentración bancaria (los cinco bancos más grandes controlan más de 75\% de los activos financieros) puede encarecer el crédito a mediano plazo. A nivel regulatorio, la Superintendencia Financiera exige sistemas de prevención de lavado de activos (SARLAFT) y ciberseguridad, lo que se traduce en costos iniciales para emprendimientos financieros.

En resumen, el \textbf{entorno económico colombiano para 2024–2025} presenta un moderado crecimiento con inflación en descenso, descenso del desempleo y mejora en las condiciones de ingreso real. Las perspectivas macroeconómicas –PIB proyectado ~1,9\% (2024) y 2,9\% (2025) según BanRep apuntan a un ambiente estable pero competitivo. Para nuestro servicio, esto significa un mercado con \textbf{potencial de demanda} (pues la población tiene mayor liquidez disponible y alto índice de bancarización), pero también con consumidores exigentes y costos operativos (tasa de interés de mercado, inflación) que deben considerarse en la estructura financiera del plan.