\subsection*{Estudio ambiental}
\addcontentsline{toc}{subsection}{Estudio ambiental}

El estudio ambiental de nuestra compañía buscará centrarse en el cumplimiento de las normativas dadas en temas de asegurar el correcto manejo del impacto social y el cumplimiento de los requisitos legales que se asocian al desarrollo de nuestro software y las actividades dadas virtualmente. Como es perteneciente al proceso de invención y distribución de soluciones digitales, es de gran importancia tener claro las características relacionadas con el manejo de recursos, el impacto tecnológico y la protección de los derechos de autor.

De acuerdo al Decreto 1360 de 1989, bajo el artículo 1, presenta los siguiente: \textit{"De conformidad con lo previsto en la Ley 23 de 1982 sobre Derechos de Autor, el soporte lógico (software) se considera como creación propia del dominio literario"}. Lo que significa que todo software desarrollado por EduFinanzas deberá ser sometido bajo un proceso de registro por obra, el cual será suministrado por la Dirección Nacional de Derechos de Autor.

Por lo cual, bajo este registro se asegura que los derechos de autor del software así como las herramientas digitales sean protegidos, lo que es importante para la seguridad legal de EduFinanzas y la integridad de las soluciones que presenta. Junto a ello, este paso colabora a evitar la utilización indebida o copia no autorizada del software, logrando de esta forma asegurar la propiedad intelectual de EduFinanzas sobre sus desarrollos.

Como otros emprendimientos buscando comprometer la responsabilidad social y ambiental, EduFinanzas se centrará a la gestión ambiental con la tarea de solucionar, mitigar y/o evadir problemas que se deriven del manejo de EduFinanzas. Enfocándonos en aspectos e impactos ambientales que puedan surgir del manejo de nuestras actividades, la empresa se centrará en el uso responsable sobre los recursos tecnológicos y garantizando un desarrollo sostenible. Para lograr acaparar el buen cumplimiento de estos principios, se juntarán estrategias basadas en normativas internacionales acerca de la calidad y gestión ambiental.

\begin{itemize}
	\item \textbf{Norma Técnica Colombiana ISO 9001:2015:} Bajo esta norma, se proporciona un marco de referencia para sistemas de gestión de calidad. EduFinanzas, por lo tanto, adoptará este estándar para asegurarse de que las tareas del desarrollo de software, manejo de recursos de educación financiera y su operación logren cumplir con los requisitos de calidad que satisfagan las expectativas de los clientes y usuarios.
	
	\item \textbf{Norma Técnica Colombiana ISO 9001:2015:} En esta norma, se dan criterios para manejar sistemas efectivos de gestión ambiental. EduFinanzas, entonces, aplicará esta norma para la gestión de aspectos ambientales de su operación, garantizando de que se identifique y controle los impactos que pueda ocasionar en el medio ambiente. Al continuar este estándar, nuestra empresa demostrará el compromiso por mantener la sostenibilidad y reducir su huella ambiental.
\end{itemize}

\subsubsection*{Impactos ambientales}
\addcontentsline{toc}{subsubsection}{Impactos ambientales}

Para lograr que nuestra empresa logre cumplir con normativas ambientales y minimizar impactos negativas que puedan derivarse de la operación, se hallan y se eliminan posibles impactos ambientales con la implementación de distintas medidas.

\begin{table}[H]
	\centering
	\includegraphics[width=12cm]{Imagenes/TablaImpactosAmbientales.png}
	\caption[{Impactos ambientales.}]{\centering Tabla de impactos ambientales. \textit{Fuente:} Autores.}
	\label{ImpactosAmbientales}
\end{table}

\subsubsection*{Políticas ambientales}
\addcontentsline{toc}{subsubsection}{Políticas ambientales}

EduFinanzas va a regirse bajo la norma internacional NTC ISO 14001:2015 (ISO 14001, 2015), de aplicación voluntaria. La norma mencionada anteriormente establece que los requisitos que deben ser cumplidos por nuestra organización para gestionar de forma efectiva la prevención de la contaminación y el control de tales actividades, procesos y productos que sean capaces de dar un impacto negativo dentro del ambiente, por tanto nos comprometemos a:

\begin{itemize}
	\item Demostrar el cumplimiento de los requisitos de la norma, esto por medio de la implementación de un Sistema de Gestión Ambiental (SGA), que busque gestionar de forma eficiente los recursos y minimizar los impactos negativos.
	
	\item Asegurar nuestro compromiso y la responsabilidad sobre la protección del medio ambiente, buscando promover una gestión ambiental consciente en cada una de las fases de nuestras operaciones.
	
	\item Buscar establecer una política ambiental clara y efectiva que pueda orientar las actividades de EduFinanzas hacia la sostenibilidad y la mejora continua dentro de la gestión ambiental.
	
	\item Evitar contaminar y además usar los recursos de forma sostenible, con claras miras hacia la mitigación del cambio climático, la protección de los ecosistemas y buscar preservar la biodiversidad.
	
	\item Asegurar la conformidad con la política ambiental dada, garantizando que las acciones de EduFinanzas se alineen con los objetivos de sostenibilidad y de protección por el medio ambiente.
	
	\item Promover una mejor y mayor protección ambiental al tener en cuenta esta parte como un elemento estratégico esencial dentro de la toma de decisiones.
	
	\item Promover y disminuir la generación de residuos, optimizando el uso de recursos y buscando minimizar el impacto ambiental dentro de las actividades de EduFinanzas.
\end{itemize}

