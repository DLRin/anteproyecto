\subsection*{Estudio técnico}
\addcontentsline{toc}{subsection}{Estudio técnico}

\subsubsection*{Tamaño}
\addcontentsline{toc}{subsubsection}{Tamaño}

De acuerdo al Decreto 957 del 5 de junio del 2019, se presentan los criterios para la clasificación de micro, pequeña, mediana y grande empresa, con esto presentaremos una clasificación del tamaño empresarial según la ley colombiana dependiendo de las variables de ingresos y costos relacionados. También se determina el tamaño inicial de la compañia al limitar sus servicios a un nivel distrital y bajo el estudio de mercado desarrollado se tendrán las acciones pertinentes para crecer y desarrollar el producto, teniendo en cuenta factores tales como:

\begin{itemize}
	\item \textbf{Recursos humanos:} La compañía contiene una estructura organizacional en la cual se presentan departamentos administrativos, de marketing, de ventas y de desarrollo, para que así, de esta forma, el recurso humano sea estratégicamente organizado para un eficiente desempeño y posterior crecimiento.
	
	\item \textbf{Potencial del mercado:} En la actualidad, es un mercado con demanda y con el desarrollo de alianzas estratégicas y planes de mercadeo, se puede obtener un gran impacto, logrando de esta forma un buen posicionamiento por la prestación del servicio.
	
	\item \textbf{Factores de producción:} Dentro de los principios de la empresa, la materia prima son equipos de estándar medio, pero que, sin embargo, cumplen con su función para los desarrolladores y el área encarga de la planificación en la organización.
\end{itemize}

A continuación se presenta la tabla en la cual se establece el tamaño de las empresas a partir del criterio de ingresos por actividades ordinarias, clasificadas en tres macro sectores económicos: \textit{manufactura, servicios y comercio} \cite{Bibl042}. Tener en cuenta que para el año 2025, el valor de la unidad de valor tributario (UVT) es equivalente a \$ 49.799 pesos colombianos.

\begin{adjustbox}{
		center,
		caption=[{Clasificación de empresas en Colombia.}]{\centering Clasificación de empresas en Colombia. \textit{Fuente:} Decreto 957 de 2019, Gobierno Nacional de Colombia.},
		label={ClasificacionEmpresasColombia},
		nofloat=table, vspace={20px}}
	\resizebox{\textwidth}{!}{
		\begin{tabular}{|c|c|c|c|}
			\hline
			\rowcolor[HTML]{D9EAD3} 
			\textbf{Tamaño} & \textbf{Manufactura} & \textbf{Servicios} & \textbf{Comercio} \\ \hline
			\textbf{Microempresa} & \begin{tabular}[c]{@{}c@{}}Hasta 23.563 \\ UVT\end{tabular} & \begin{tabular}[c]{@{}c@{}}Hasta 32.988 \\ UVT\end{tabular} & \begin{tabular}[c]{@{}c@{}}Hasta 44.769 \\ UVT\end{tabular} \\ \hline
			\begin{tabular}[c]{@{}c@{}} \textbf{Pequeña} \\ \textbf{Empresa}\end{tabular} & \begin{tabular}[c]{@{}c@{}}Desde 23.563 UVT \\ hasta 204.995 UVT\end{tabular} & \begin{tabular}[c]{@{}c@{}}Desde 32.988 UVT \\ hasta 131.951 UVT\end{tabular} & \begin{tabular}[c]{@{}c@{}}Desde 44.769 UVT \\ hasta 431.196 UVT\end{tabular} \\ \hline
			\begin{tabular}[c]{@{}c@{}} \textbf{Mediana} \\ \textbf{Empresa}\end{tabular} & \begin{tabular}[c]{@{}c@{}}Desde 204.995 UVT \\ hasta 1'736.565 UVT\end{tabular} & \begin{tabular}[c]{@{}c@{}}Desde 131.951 UVT \\ hasta 483.034 UVT\end{tabular} & \begin{tabular}[c]{@{}c@{}}Desde 431.196 UVT \\ hasta 2'160.692 UVT\end{tabular} \\ \hline
		\end{tabular}
	}
\end{adjustbox}

El objetivo principal de este marco regulatorio es buscar promover la formación formal de las empresas y poder asegurar una adecuada diversidad dentro del sistema empresarial colombiano. Esta tiene en cuenta la diversidad de los ingresos y la productividad de las empresas de distintos sectores económico. Además contribuye a mejorar la implementación de medida gubernamentales y programas de apoyo las cuales cubren a todo tipo de empresas. Ayudando de esta forma a mejorar y promover el desarrollo sostenible y la competitividad del sector empresarial en Colombia. Nuestra empresa, entonces, se encuentra en el sector de los servicios, ya que es una plataforma virtual dedicada a la educación financiera en Colombia para nómadas digitales, personas jóvenes e interesados en las finanzas.

Debido a que, nuestra empresa va a estar en sus etapas iniciales y teniendo en cuenta el enfoque geográfico limitado al que trabajaremos, es razonable clasificar la empresa como una \textbf{microempresa} dentro del sector de los servicios, de acorde a los lineamientos y normativas regidas por la ley colombiana vigente.

\subsubsection*{Localización}
\addcontentsline{toc}{subsubsection}{Localización}

La ubicación de nuestra empresa es uno de los factores más importantes para que el impacto que generé nuestro producto o servicio representativo y suficiente como para poder entrar dentro del mercado además de conseguir y asegurar posicionamiento, esto influye en la competencia del mercado y apoya la primera fase de crecimiento dentro de la empresa. A pesar de que no se indispensable contar con una ubicación estratégica precisa, es bastante recomendable que se encuentre dentro de la ciudad de Bogotá, dado que será el principal alcance de nuestro proyecto. Para elegir la sede inicial, es de importancia considerar varios aspectos, entre los cuales están:

\begin{itemize}
	\item Facilidad de acceso del lugar y comunicación.
	
	\item Ambiente laboral cómodo.
	
	\item Costos accesibles.
	
	\item Disponibilidad de servicios básicos tales como internet, electricidad y sanitarios. 
\end{itemize}

\paragraph*{Lugares candidatos}
\addcontentsline{toc}{subsubsection}{Lugares candidatos}

Para dar con la elección de los lugares candidatos, se tienen en cuenta los espacios diseñados para trabajar colaborativamente dentro de ambientes agradables y que colaboren con el flujo de las actividades continuas y sostenidas, actualmente espacios como los \textbf{coworking} logran cumplir aquella necesidad y la infraestructura buscada, además de contar junto a espacios en los que se pueda manejar la interdisciplinariedad de las distintas partes que comparten el ambiente de trabajo, lo que permite aportar valor al producto y mejorar las oportunidades de apoyo de nuestro modelo. Los lugares candidatos son:

\begin{itemize}
	\item \textbf{WeWork:} Son espacios de trabajo compartidos cuyo objetivo es ofrecer soluciones de trabajo en las que todos podamos ser más productivos, motivados y lograr generar mejores proyectos. Además de contar con soluciones de espacio para pequeñas y medianas empresas, inclusive a grandes empresas también. Los cuales no solamente se limita a una oficina en particular sino que también logran ofrecer sus servicios a distintas ciudades de Latinoamérica.
	
	\begin{adjustbox}{
			center,
			caption=[{Oficinas de WeWork.}]{\centering Oficinas de WeWork. \textit{Fuente:} WeWork.com.co},
			label={TablaWeWork},
			nofloat=table, vspace={20px}}
			\begin{tabular}{|c|c|}
				\hline
				\rowcolor[HTML]{D9EAD3} 
				\textbf{Servicio} & \textbf{Precio x Mes} \\ \hline
				Plan All Access & \$390.000 + IVA \\ \hline
				Plan All Access Plus & \$600.000 + IVA \\ \hline
			\end{tabular}
	\end{adjustbox}
	
	\item \textbf{Office ToGo:} Son espacios de trabajo compartidos dentro de los cuales las pequeñas y medianas empresas y emprendedores y trabajadores independientes tienen la oportunidad de poder utilizar espacios de trabajo equipados y organizados con todo lo necesario, además de encontrarse en las mejores ubicaciones de Bogotá.
	
	\begin{adjustbox}{
			center,
			caption=[{Oficinas de OfficeToGo.}]{\centering Oficinas de OfficeToGo. \textit{Fuente:} OfficeToGo.co},
			label={TablaOfficeToGo},
			nofloat=table, vspace={20px}}
			\begin{tabular}{|c|c|}
				\hline
				\rowcolor[HTML]{D9EAD3} 
				\textbf{Servicio} & \textbf{Precio x Mes} \\ \hline
				Planes virtuales & \$99.000 + IVA \\ \hline
				Plan emprendedor & \$169.000 + IVA \\ \hline
				Oficinas generales & \$1.100.000 + IVA \\ \hline
				Oficina semi-privada & \$850.000 + IVA \\ \hline
			\end{tabular}
	\end{adjustbox}
	
	\item \textbf{ParqueSoft:} ParqueSoft es un proveedor multisectorial de servicios de las TIC (tecnologías de la información) que cuenta con el apoyo de más de 50 empresas de base dentro de Bogotá, que trabajando juntas generan una oferta de valor integral logrando un ecosistema de apoyo, buscando crear soluciones de calidad, a la vanguardia de las últimas tendencias digitales. Además ParqueSoft es el clúster de ciencia y tecnología informática más grande de Colombia y uno de los líderes más importantes de temas de apoyo a proyectos de emprendimiento con base tecnológica, impulsándolos y colaborando a que crezcan.
	
	\begin{adjustbox}{
			center,
			caption=[{Oficinas de ParqueSoft.}]{\centering Oficinas de ParqueSoft. \textit{Fuente:} ParqueSoftBogota.com},
			label={TablaParqueSoft},
			nofloat=table, vspace={20px}}
			\begin{tabular}{|c|c|}
				\hline
				\rowcolor[HTML]{D9EAD3} 
				\textbf{Servicio} & \textbf{Precio x Mes} \\ \hline
				Pago Mensual & \$96.000 al mes $\sim$ \$1.152.000 al año \\ \hline
				Pago Semestral & \$78.000 al mes $\sim$ \$932.000 al año \\ \hline
				Pago Anual & \$69.000 al mes $\sim$ \$828.000 al año \\ \hline
			\end{tabular}
	\end{adjustbox}
	
	\item \textbf{Elección:} Considerando las ofertas que tenemos disponibles de sitios en los cuales puede consolidarse la empresa, y habiendo estudiado los beneficios que tiene consigo en cuanto a la ubicación, los servicios ofrecidos, la accesibilidad, la seguridad, la estructura física, entre otros más. Se decide que la mejor opción entre las disponibles es \textbf{ParqueSoft} debido a que sus servicios son los apropiados para nuestra propuesta de emprendimiento que apenas esta surgiendo, además de que sus costos y precios son accesibles y el tema de la capacitación y apoyo nos aportarán madurez y experiencia dentro de la empresa.
\end{itemize}

\subsubsection*{Tipo de emprendimiento}
\addcontentsline{toc}{subsubsection}{Tipo de emprendimiento}

Para clasificar el emprendimiento a desarrollar con el proyecto, se va a tomar como un Startup usando la \textit{Metodología Lean Startup}, la cual busca centrarse en el desarrollo ágil y en la iteración constante del producto basado en la retroalimentación directa con los usuarios. Debido a que somos un desarrollo de un producto emergente e innovador que tiene implementación de las tecnologías de la información, nos permite un desarrollo centrado con el cliente, reduciendo los riesgos y además mejorando la optimización de nuestros recursos, además de facilitar la adaptabilidad y la escalabilidad. Con ello nos aseguramos de que nuestra empresa logre responder velozmente a las necesidades tanto del usuario como del mercado, logrando crear un producto que evoluciona constantemente en materia de retroalimentación y aprendizaje real.

Por tanto, el desarrollo de esta metodología nos permitirá alcanzar un crecimiento escalable y sostenible a lo largo del tiempo, respaldado por el desarrollo de un modelo de negocios sólido.

\begin{figure}[H]
	\centering
	\includegraphics[width=12cm]{Imagenes/MetodologiaLeanStartUp.png}
	\caption[{Metodología Lean Startup.}]{\centering Metodología Lean Startup. \textit{Fuente:} Sitio Web: (El método Lean Startup: que és y cómo puede ayudarte a evitar fracasar, ThePowerMBA, 2025)} 
	\label{fig:metodologialeanstartup}
\end{figure}

A través de este ciclo iterativo y con el uso de metodologías ágiles, se propone desarrollar negocios en los cuales los emprendedores crean un \textit{Producto Mínimo Viable (PMV)} basado en hipótesis, este PMV nos permite, con una inversión mínima, poder evaluar si nuestra idea sera aceptada dentro del mercado, si estas satisfacen las necesidades de los clientes y además si tiene cabida y potencial para escalar. En caso de necesitar ajustar o cambiar el rumbo del enfoque, se considera necesario pivotar dentro de ese marco de trabajo.

\subsubsection*{Equipo de trabajo}
\addcontentsline{toc}{subsubsection}{Equipo de trabajo}

Dentro de la planeación e implementación de este proyecto, estamos conformados por:

\begin{itemize}
	\item \textbf{Gerente general:} El encargado de planificar las actividades que ocurran dentro de la empresa. Organizar los recursos dentro de la entidad y definir en que dirección se va a dirigir la empresa en el corto, mediano y largo plazo, entre muchas otras tareas más.
	
	\item \textbf{Líder del área de ingeniería:} Responsable de liderar un equipo de desarrollo y el responsable acerca de la calidad de sus producto. Busca establecer una visión técnica con el equipo de desarrollo y trabaja con ellos para buscar alcanzar el objetivo.
	
	\item \textbf{Director del departamento de investigación, desarrollo e innovación (I+D+I):} Es la persona encargada de dirigir la investigación, la innovación y el desarrollo. Por lo tanto es responsable del avance tecnológico e investigativo centrados hacia el avance de la sociedad, siendo una de las partes más elementales dentro de las tecnologías informativas. Junto a este rol, también tiene la tarea de coordinar y manejar los equipos interdisciplinares para desarrollar nuevos productos o tecnologías.
\end{itemize}

A continuación el equipo de trabajo será el siguiente:

\begin{adjustbox}{
		center,
		caption=[{Equipo de trabajo.}]{\centering Equipo de trabajo \textit{Fuente:} Autores},
		label={TablaEquipoTrabajo},
		nofloat=table, vspace={20px}}
	\resizebox{\textwidth}{!}{
		\begin{tabular}{|c|c|}
			\hline
			\rowcolor[HTML]{D9EAD3} 
			\textbf{Integrante} & \textbf{Rol} \\ \hline
			David Leonardo Rincón Ovalle & Gerente general, líder del área de ingeniería \\ \hline
			Leidy Marcela Morales Segura & Director del departamento de investigación, desarrollo e innovación \\ \hline
		\end{tabular}
	}
\end{adjustbox}

