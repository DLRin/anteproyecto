\subsection*{Estudio legal}
\addcontentsline{toc}{subsection}{Estudio legal}

El estudio legal de este proyecto y, por lo tanto, nuestra empresa, tiene el objetivo de buscar averiguar la viabilidad ante las normas o leyes que se rigen dentro de la constitución colombiana, reconociendo la empresas, sus alcances, limitaciones y obligaciones que se relacionen con la naturaleza de los mismo, se busca definir en la menor cantidad de palabras posibles su formalización ante las entidades estatales. Asegurándose, de esta forma que la empresa logre funcionar dentro del marco legal colombiano, alcanzando y cumpliendo las regulaciones requeridas.

\subsubsection*{Tipo de sociedad}
\addcontentsline{toc}{subsubsection}{Tipo de sociedad}

Nuestra empresa será constituida bajo la sociedad por acciones simplificadas (S.A.S). Sociedad que fue creada dentro de la legislación colombiana mediante la Ley 1258 de 2008, Este tipo de sociedades se ha convertido en una de las más usadas dentro del país gracias a las ventajas que tiene en materia de simplicidad y flexibilidad. Adicionalmente, para efectos tributarios, las S.A.S. se rigen por las reglas aplicables a las sociedades anónimas.

Este tipo de sociedad es el mejor para nuestra empresa, especialmente debido a las ventajas que trae dentro del ámbito tributarios. Ya que desde la entrada en vigencia de la Ley 1258, alrededor del 54\% de las empresas colombianas lo hacen bajo esta modalidad. La elección de esta sociedad nos permite simplificar los trámites burocráticos, de esta forma disminuyendo los costos del inicio y facilitando el proceso de formalización de nuestra empresa. Junto a ello, las S.A.S. no necesitan un revisor fiscal, permitiendo un ahorro significativo, y estas sociedades pueden ser llevadas a cabo tanto por personas naturales como jurídicas, brindando flexibilidad hacia los emprendedores.

\subsubsection*{Especificaciones generales}
\addcontentsline{toc}{subsubsection}{Especificaciones generales}

Nuestra empresa será constituida bajo las siguientes pautas:

\begin{itemize}
	\item \textbf{Socios:} Leidy Marcela Morales Segura con Cédula de Ciudadanía No. y David Leonardo Rincón Ovalle con Cédula de Ciudadanía No. 1001093774.
	
	\item \textbf{Nombre de la empresa:} EduFinanzas.
	
	\item \textbf{Duración:} Tiempo indefinido.
	
	\item \textbf{Objeto social:} Empresa cuyo objeto se encuentra en el desarrollo e innovación de herramientas tecnológicas.
	
	\item \textbf{Responsabilidad sobre los aportes:} 50/50 entre socios.
	
	\item \textbf{Representante legal:} David Leonardo Rincón Ovalle.  
\end{itemize}

\subsubsection*{Obligaciones legales}
\addcontentsline{toc}{subsubsection}{Obligaciones legales}

Las obligaciones legales que se presentan a continuación se encuentran enmarcadas en la Ley 1258 de 2008, ``por medio de la cual se crea la Sociedad por Acciones Simplificada (S.A.S.)''. Esta ley establece cuales son las responsabilidades y deberes que debe cumplir nuestra empresa para operar legalmente bajo esta figura jurídica en Colombia.

\begin{enumerate}
	\item Asamblea Ordinaria de Accionistas, por lo menos una (1) vez al año.
	
	\item Renovación de la matrícula mercantil dentro de los primeros tres (3) meses del año en la Cámara de Comercio.
	
	\item Firma electrónica de los representantes legales, resolución 070 de la Dirección de Impuestos y Aduanas Nacionales (DIAN).
	
	\item Agentes de retención en la frente a título de renta, IVA, ICA, etc.
	
	\item Están obligados a expedir facturas.
	
	\item Gravamen a los movimientos financieros.
	
	\item Están obligados a llevar la contabilidad de su empresa.
	
	\item Están obligados a tener revisor fiscal según el monto de sus ingresos o activos.
\end{enumerate}

\subsubsection*{Proceso para disolución de la empresa}
\addcontentsline{toc}{subsubsection}{Proceso para disolución de la empresa}

Para la disolución de nuestra empresa, empresa constituida dentro de la figura de Sociedad por Acciones Simplificadas (S.A.S.), se seguirán los próximos pasos de acuerdo a la normativa legal colombiana:

\begin{enumerate}
	\item En el casos de las S.A.S., Empresa Unipersonal y Sociedades de la Ley 1014 de 2006, la declaratoria de disolución por mutuo acuerdo podrá realizarse mediante acta o documento privado. Este documento deberá estar firmado por las partes involucradas y ser aprobado por la Junta de Accionistas.
	
	\item Posteriormente, será necesario realizar el registro del acta de disolución en la Cámara de Comercio. A partir de este momento, la empresa pasará a tener el sufijo “en liquidación”. Se entregará copia del acta en la Cámara de Comercio, mientras que el original quedará en los archivos de nuestra empresa.
	
	\item El siguiente paso es el reporte a la Oficina de Cobranzas de la DIAN sobre cualquier deuda fiscal que tenga la sociedad. El liquidador de la empresa deberá hacer este reporte dentro de los diez (10) días posteriores al registro de la disolución en la Cámara de Comercio.
	
	\item Emitir avisos públicos que informen que la sociedad está en trámite de liquidación. Esto es necesario para dar conocimiento a terceros sobre el estado legal de la empresa.
	
	\item Elaborar un inventario del patrimonio social y el balance final de la empresa, tarea que corresponde al liquidador de la misma. Este inventario debe detallar todos los activos y pasivos de la sociedad.
	
	\item Pagar el pasivo externo, es decir, las deudas con terceros, algo que también debe gestionar el liquidador. En este paso se incluyen las obligaciones fiscales pendientes y la declaración de renta final de nuestra empresas.
	
	\item Una vez pagadas las deudas, el liquidador procederá a distribuir los remanentes entre los socios o accionistas de la sociedad.
	
	\item Elaborar un proyecto de liquidación. Este proyecto debe contener como mínimo:
	
	\begin{itemize}
		\item Inventarios.
		
		\item Balance general.
		
		\item Estado de pérdidas y ganancias.
		
		\item Pasivos de la entidad.
		
		\item Se solicita el estado de cuenta a la DIAN.
		
		\item Pago de pasivos.
		
		\item Indicación del remanente.
		
		\item Destinación del remanente.
	\end{itemize}
	
	\item Convocar una reunión de la Junta de Socios o Asamblea de Accionistas para aprobar el proyecto de liquidación presentado por el liquidador.
	
	\item Finalmente, el último paso consiste en realizar el registro del acta de la cuenta final de liquidación ante la Cámara de Comercio. Se deberá cancelar una tarifa de impuesto de registro del 0.7\% sobre el valor de los remanentes de la empresa, después
	de pagar su pasivo externo. En caso de que no haya remanente a repartir, se pagará como un acto sin cuantía.
\end{enumerate}

\subsubsection*{Propiedad intelectual}
\addcontentsline{toc}{subsubsection}{Propiedad intelectual}

De acuerdo al Decreto 1360 de 1989, artículo 1, presenta lo siguiente: ``De conformidad con lo previsto en la Ley 23 de 1982 sobre Derechos de Autos, el soporte lógico (software) se considera como creación propia del dominio literario''. De tal forma que todo software se ha de desarrollar deberá ser sometido hacia procesos de registro por obra, el cual será suministrado por la Dirección Nacional de Derechos de Autor.

\subsubsection*{Permisos y licencias requeridos}
\addcontentsline{toc}{subsubsection}{Permisos y licencias requeridos}

Para poder ejecutar nuestro comercio, necesitamos los siguientes permisos y licencias, cabe resaltar, que a pesar de que la naturaleza de nuestro negocio no es exactamente un e-commerce tradicional, si integramos componentes digitales de prestación de servicios online que nos acercan a este modelo:

\begin{itemize}
	\item \textbf{Registro de Unidad Tributaria (RUT):} Es el registro que realizamos frente a la DIAN para obtener una Unidad Tributaria, que será el número de identificación fiscal de nuestra compañia. Este registro es obligatorio para cualquier empresa que busque operar en Colombia.
	
	\item \textbf{Registro de Empresas Unificadas de Servicios (RUES):} Es el registro con el que se realiza frente a la Cámara de Comercio para poder inscribirse en la sociedad del Registro Mercantil. Este registro es obligatorio para cualquier empresa que busque operar en Colombia.
	
	\item \textbf{Registro del Sistema de Administración de Riesgo de Lavado de Activos y Financiación del Terrorismo (SARLAFT):} Es el registro realizado ante la Superintendencia Financiera bajo la cual se específica que buscamos prevenir que las actividades de nuestra empresas no se utilicen para lavar activos o financiar terroristas. Este registro es obligatorio para cualquier empresa (principalmente de carácter financiero) que busque operar en Colombia.
	
	\item \textbf{Registro de Comercio:} Es el registro que se hace en la Cámara de Comercio con el objetivo de tener una licencia de comercio. Esta licencia es obligatoria para cualquier empresa que desee operar en Colombia.
	
	\item \textbf{Registro de Contribuyente:} Es el registro que se hace en la DIAN con el objetivo de tener una licencia de comercio. Esta licencia es obligatoria para cualquier empresa que desee operar en Colombia.
	
	\item \textbf{Registro de Actividades Económicas:} Es el registro que se hace frente a la Cámara de Comercio para conseguir un número de identificación fiscal. Esta licencia es obligatoria para cualquier empresa que desee operar en Colombia.
	
	\item \textbf{Registro de Seguridad y Salud en el Trabajo:} Es el registro que se hace ante el Ministerio de Trabajo para conseguir una licencia de seguridad y salud en el trabajo. Esta licencia es necesaria para cualquier empresa que desee operar en Colombia.
	
	\item \textbf{Registro de Impuestos:} Es el registro realizado ante la DIAN para obtener una licencia de impuestos. Esta licencia es necesaria para cualquier empresa que desee operar en Colombia.
	
	\item \textbf{Registro de Comercio Electrónico:} Es el registro realizado ante la Superintendencia de Industria y Comercio para obtener una licencia de comercio electrónico. Esta licencia es necesaria para cualquier empresa que plantee operar un e-commerce en Colombia.
	
	\item \textbf{Registro de Protección de Datos:} Es el registro que se hace ante la Superintendencia de Industria y Comerico para ganar una licencia de protección de datos. Esta licencia es obligatoria para empresas que deseen operar un e-commerce en Colombia.
	
	\item \textbf{Registro de Seguridad de la Información:} Es el registro realizado ante la Superintendencia de Industria y Comercio para obtener una licencia de seguridad de la información. Esta licencia es obligatoria para cualquier empresa que desee operar un e-commerce en Colombia.
\end{itemize}

\subsubsection*{Normas técnicas y de calidad}
\addcontentsline{toc}{subsubsection}{Normas técnicas y de calidad}

Para poder ejecutar nuestro comercio, las normas técnicas y de calidad necesarias son las que están a continuación:

\begin{itemize}
	\item \textbf{Norma Técnica Colombiana (NTC) ISO 9000:} Describe los fundamentos sobre los sistemas de gestión de la calidad y especifica la terminología de los sistemas de gestión de la calidad. Esta norma es fundamental para garantizar la calidad y satisfacción del cliente.
	
	\item \textbf{Norma ISO 9001:} Es el estándar de calidad más aplicado dentro de todo el mundo. Acá se describen la implementación de los procesos y procedimientos destinados para lograr la satisfacción del cliente. La certificación ISO 9001 demuestra que EduFinanzas tiene un compromiso para garantizar la calidad, la consistencia y la autenticidad.
	
	\item \textbf{Norma ISO 10008:} Es el estándar de gestión específico para el comercio electrónico. Asegura que se implementen los procesos y tareas con el objetivo de proteger la información y datos de nuestros clientes.
	
	\item \textbf{Norma ISO 27001:} Es un estándar de seguridad de la información que busca garantizar la protección de la información y los datos de nuestro clientes. Esta norma es esencial para proteger la información y datos de EduFinanzas.
	
	\item \textbf{Norma ISO 20000:} Es un estándar de gestión de continuidad de los negocios que busca garantizar la continuidad de los negocios en caso de que ocurran desastres y situaciones críticas. Esta norma es esencial para proteger la continuidad de los negocios.
	
	\item \textbf{Norma ISO 22301:} Es un estándar de gestión para continuidad de los negocios que busca garantizar la continuidad de los negocios en caso de que ocurran desastres o situaciones críticas. Esta norma es fundamental para proteger la continuidad de los negocios.
	
	\item \textbf{Norma ISO 27002:} Es el estándar de seguridad de la información que proporciona recomendaciones prácticas para implementar medidas de seguridad en la información de un negocio.
\end{itemize}

