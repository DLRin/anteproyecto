\subsection*{Estudio de mercado}
\addcontentsline{toc}{subsection}{Estudio de mercado}

\subsubsection*{Identificación del producto}
\addcontentsline{toc}{subsubsection}{Identificación del producto}
El producto de nuestra empresa consiste en una plataforma digital integral de planificación financiera diseñada específicamente para personas que tengan poco o nulo conocimiento sobre finanzas que residan dentro de la ciudad de Bogotá. La solución busca combinar la educación financiera práctica junto a herramientas tecnológicas avanzadas, para democratizar el acceso a la asesoría financiera.

Las funcionalidades que ofrece este servicio incluye:

\begin{itemize}
	\item Simuladores financieros interactivos que permiten a los usuarios experimentas con múltiples escenarios de ahorro, inversión y de presupuesto.
	
	\item Calculadoras personalizadas para proyecciones financieras específicas dependiendo del perfil del usuario.
	
	\item Herramientas de presupuesto digital con interfaces intuitivas e incorporando gamificación ligera para fomentar buenos hábitos.
	
	\item Sistemas de alertas inteligentes para hacer recordatorios de pagos, metas de ahorro y también oportunidades de optimización.
	
	\item Contenido didáctico interactivo con explicaciones claras acerca de los conceptos financieros básicos.
	
	\item Canales de chat personalizados entre los usuarios con asesores financieros.
\end{itemize}

\subsubsection*{Demanda}
\addcontentsline{toc}{subsubsection}{Demanda}
Nuestra plataforma se dirige a todas aquellas personas interesadas dentro del mundo financiero, cuyo público objetivo incluye tanto a personas jóvenes que busquen mayor independencia financiera desde temprana edad, pasando por los trabajadores informales, junto a los nómadas digitales con ingresos variables y otras personas más. Nuestro objetivo es ayudar a las personas a aprender acerca de la educación financiera entregando opciones de inversión, educación y asesoría personalizada para crear y formar hábitos financieros sostenibles. Nos enfocamos en la población de la ciudad de Bogotá, facilitando la conexión de personas interesadas en el mundo financieros con las condiciones financieras de la ciudad.

\subsubsection*{Oferta}
\addcontentsline{toc}{subsubsection}{Oferta}
Nuestra plataforma busca ofrecer una plataforma \textit{freemium}, en la cual se permita al usuario utilizar la plataforma de forma gratuita con la opción de hacer una suscripción mensual para tener mayor acceso a las características de la plataforma.

\subsubsection*{Plan de Marketing}
\addcontentsline{toc}{subsubsection}{Plan de Marketing}
En toda empresa los planes de marketing son de gran importancia, principalmente durante sus primeros años, pues es en estos años en los cuales se busca llegar a ese primer grupo de clientes. Con el uso del marketing se permite la construcción de una imagen empresarial y obtener el reconocimiento suficiente para lograr consolidarse dentro del mercado.

El plan consiste en objetivos y tácticas analizadas a continuación:

\paragraph*{Objetivos}
\addcontentsline{toc}{paragraph}{Objetivos}

\begin{itemize}
	\item Lanzar al mercado la plataforma con el uso de suscripciones \textit{freemium}.
	
	\item Obtener una buena acogida del servicio dentro de los potenciales clientes.
	
	\item Fidelizar a los clientes potenciales y generar el interés de otros clientes.
	
	\item Adquirir reconocimiento dentro del mercado y mantenerse en el mismo.
	
	\item Maximizar los beneficios generados por el servicio a corto y a largo plazo.
\end{itemize}

\paragraph*{Tácticas de marketing}
\addcontentsline{toc}{paragraph}{Tácticas de marketing}
Para nuestro proyecto vamos a hacer uso del marketing digital que atraigan nuestro potenciales clientes hacia nuestra plataforma, las fases del marketing digital se ilustran mediante la figura \ref{fig:pasosinboundmarketing}, proponiendo la aplicación del \textit{inbound marketing} para lograr atraer más clientes hacia nuestra plataforma. Dado que el principal público objetivo de nuestra organización son las personas jóvenes, los trabajadores informales y los nómadas digitales. Es esencial la comunicación bidireccional para ganar y obtener su confianza y satisfacción. 

Se define el término ``Inbound Marketing'' como una metodología de marketing digital que combina las técnicas no intrusivas del marketing y de la publicidad con el objetivo de contactar a un usuario al inicio de su proceso de compra y poder acompañarlo hasta la transacción final. El uso de esta metodología nos representa una ventaja para nuestra empresa para lograr el contacto mediante las vías de comunicación con nuestros clientes para facilitar los temas de educación financiera, ofreciendo un asesoramiento personalizado y mayores oportunidades de inversión. En vez de que los clientes se contacte con nuestra empresa en el momento antes de la compra, el Inbound Marketing les permite obtener que el contacto ocurra al momento de cierre, optimizando los recursos obtenidos en interacciones previas dadas con el cliente.

\begin{figure}[H]
	\centering
	\includegraphics[width=12cm]{Imagenes/PasosInboundMarketing.jpeg}
	\caption[{Pasos del Inbound Marketing.}]{\centering Pasos del Inbound Marketing. \textit{Fuente:} Sitio Web: (Los 4 Pasos del Inbound Marketing, RedProfesionalesdeMKT, 2024).}
	\label{fig:pasosinboundmarketing}
\end{figure}

Las 4 fases de estas estrategias de marketing de acorde a Hubspot \cite{Bibl041} son las siguientes:

\begin{itemize}
	\item \textbf{Atraer:} Nuestro servicio de planificación financiera buscara atraer a personas sin conocimientos acerca de finanzas mediante la creación de contenido educativo valioso y relevante que aborde sus necesidades específicas. Esto incluye artículos, guías, videos y tutoriales interactivos sobre gestión financiera tanto en nuestra plataforma como en redes sociales especializadas como TikTok, Instagram, etc. Además de realizar partnerships educativos y colaboraciones con universidades y empresas para incentivar nuevos clientes.
	
	\item \textbf{Convertir:} Una vez que se haga logrado captar la atención de los clientes potenciales y lleguen a nuestra plataforma digital, los convertiremos en usuarios registrados al ofrecerles herramientas personalizadas y experiencias altamente relevantes. En las cuales los visitantes podrán acceder a simuladores financieros interactivos gratuitos, calculadoras de presupuesto personalizadas y un contenido educativo progresivo en base al conocimiento de educación financiera que poseen.
	
	\item \textbf{Cerrar:} Se busca cerrar una relación duradera con la empresa como una plataforma para la educación de la planificación financiera. Además se busca incluir estrategias sobre el manejo de los canales de atención al usuario. En la cual se asegure proporcionar al usuario con ayuda en temas financieros. Haciendo que la aplicación pueda entregarle al usuario conocimiento y acceso a temas financieros con accesibilidad representando en ventajas reales para el usuario y, en consecuencia, a nosotros como empresa.
	
	\item \textbf{Fiderizar:} Nuestra estrategia de fidelización se basa en que los usuarios garanticen el mantenimiento de la suscripción y estos se conviertan en embajadores activos de nuestro servicio. Para ello podríamos implementar un programa integral incluyendo beneficios exclusivos, contenido premium y una comunidad de aprendizaje colaborativo el cual llevé a un mayor valor percibido y genere un engagement constante.
\end{itemize}

La implementación de estas cuatro fases del Inbound Marketing garantizan un enfoque integral que acompañará al usuario desde su descubrimiento inicial hasta la fidelización a largo plazo, maximizando tanto el impacto social como la sostenibilidad del modelo de negocio.

\paragraph*{Precios}
\addcontentsline{toc}{paragraph}{Precios}
El precio de la plataforma es esencial para contactar personas interesadas en nuestro producto. Como expresamos anteriormente, ofreceremos una plataforma \textit{freemium} por lo que el uso de la plataforma será gratuita con la opción de suscribirse mensualmente a un modelo premium. Teniendo en cuenta y de referencia el precio de la variedad de servicios de plataformas fintech. Se ha calculado el costo de la suscripción mensual premium. Cuyo precio ha tenido en cuenta los siguientes factores:

\begin{itemize}
	\item Fidelización de los clientes.
	
	\item Personalización del servicio ofrecido.
	
	\item Optimizar los beneficios de la empresa tanto a corto como a largo plazo.
	
	\item Realizar una incursión eficaz y eficiente dentro del mercado.
	
	\item Hacer mejoras tecnológicas continuas.
	
	\item Bajar los costos de mantenencia.
	
	\item Averiguar sobre productos similares de la competencia.
	
	\item Demanda que estimamos va a tener el servicio.
	
	\item Mantener la calidad del servicio.
	
	\item Soporte técnico y asesoría financiera.
\end{itemize}

Como expresamos, nuestro modelo de negocio incluirá servicios gratuitos como modelo de suscripción, el cual sería el siguiente:

\begin{adjustbox}{
		center,
		caption=[{Precio de suscripción mensual.}]{\centering Precio de suscripción mensual. \textit{Fuente:} Autores.},
		label={PreciosSuscripcionMensual},
		nofloat=table, vspace={20px}}
	\resizebox{\textwidth}{!}{
		\begin{tabular}{|b{8cm}|b{8cm}|}
			\hline
			\rowcolor[HTML]{D9EAD3} 
			\textbf{Suscripción} & \textbf{Precio} \\ \hline
			Mensualidad & \$ 30.000 \\ \hline
		\end{tabular}
	}
\end{adjustbox}

