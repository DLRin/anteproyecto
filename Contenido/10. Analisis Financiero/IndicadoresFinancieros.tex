\subsection*{Indicadores financieros}
\addcontentsline{toc}{subsection}{Indicadores financieros}

Los indicadores financieros se basan en la comparación de los dos valores esenciales dentro de documentos internos de la empresa: el \textit{balance general} y el \textit{presupuesto de la empresa}. Gracias a los indicadores financieros y su correcta interpretación, una empresa puede conocer el camino que necesita tomar con base a los datos históricos analizados.

Dentro de los indicadores financieros tomados en cuenta, se tienen los siguientes:

\begin{itemize}
	\item \textbf{Razón corriente:} Permite determina la capacidad que tiene la compañía para cubrir sus deudas con los activos que tiene.
	
	\item \textbf{Nivel de endeudamiento total:} Determina el grado y la forma de participación que tiene todos los acreedores dentro de la economía de la compañía.
	
	\item \textbf{Rentabilidad operacional:} Este indicador evidencia el porcentaje de los ingresos recibidos que fueron convertidos en beneficios después del pago de los costos operacionales.

	\item \textbf{Rentabilidad neta:} Distingue el nivel de las ganancias que se dan con respecto a las ventas netas una vez realizado el pago de los costos fijos y variables
	
	\item \textbf{Rentabilidad de patrimonio:} Evidencia cual es el nivel de ganancia que tiene la empresa desde la inversión hecha por los accionistas.
\end{itemize}

En la tabla \ref{IndicadoresFinancieros} se proyectó estos indicadores hacia 5 años.

\begin{table}[H]
	\centering
	\caption[{Indicadores financieros}]{\centering Indicadores financieros. \textit{Fuente:} Autores.}
	\includegraphics[width=13cm]{Imagenes/IndicadoresFinancieros.png}
	\label{IndicadoresFinancieros}
\end{table}

Además, se tomaron los indicadores financieros VAN, TIR y la relación costo-beneficio con el objetivo de demostrar la viabilidad de inversión que se presenta hacia los accionistas. Los resultados obtenidos a raíz del flujo de caja se deben tener en cuenta para los datos ilustrados dentro de la tabla \ref{FlujoCajaLibre}.

\begin{table}[H]
	\centering
	\caption[{Flujo de caja libre}]{\centering Flujo de caja libre. \textit{Fuente:} Autores.}
	\includegraphics[width=9cm]{Imagenes/FlujoCajaLibre.png}
	\label{FlujoCajaLibre}
\end{table}

Teniendo en cuenta la tasa interna de oportunidad, la cual es la tasa de rendimiento mínimos que se espera recibir desde los accionistas, se diseña de forma subjetiva asignándole un valor del 10\% para compensar con la inversión hecha.

\begin{table}[H]
	\centering
	\caption[{VAN, TIR y RBC}]{\centering VAN, TIR y RBC. \textit{Fuente:} Autores.}
	\includegraphics[width=9cm]{Imagenes/VANTIR.png}
	\label{VANTIR}
\end{table}

Revisando los resultados dados por el análisis financiero, se presenta que el proyecto tiene viabilidad dentro de esta área, y su éxito dependerá del atractivo que se genere hacia los futuros inversionistas. Esto colaborará a la empresa a poder tener una mayor capacidad en un mercado competitivo que presenta una alta demanda, constituyendo una oportunidad significativa.

Tanto el Valor Actual Neto (VAN) como la Tasa Interna de Retorno (TIR) resultaron positivos a largo plazo. Además, la mayor liquidez reflejada en la tabla de proyección de incremento de ventas sugiere que la compañía mejorará su liquidez, junto a un nivel de endeudamiento que muestra una tendencia hacia la baja y una rentabilidad que se evidencia de forma exponencial a lo largo de los años.