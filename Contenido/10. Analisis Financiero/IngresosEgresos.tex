\subsection*{Ingresos y egresos}
\addcontentsline{toc}{subsection}{Ingresos y egresos}

\subsubsection*{Ingresos}
\addcontentsline{toc}{subsubsection}{Ingresos}

Para desarrollar el plan de negocios propuesto, se definieron 2 planes para acceder la funcionalidad de las ventas. Empezando con el plan gratuito, dentro del cual se presenta a forma no solamente de prueba de uso general, la cuál esta dirigido a usuarios qué deseen probar la plataforma con un número limitado de acceso a contenidos interactivos y menor cantidad de opciones de inversión. La suscripción habilita funcionalidades como el acceso a mayor cantidad de contenidos, asesoría y mayores opciones de inversión en comparación con los usuarios del plan gratuito. 

Dentro de la tabla \ref{TablaIngresos} se da a conocer de forma más detallada el análisis mensual y anual de la compañía.

\begin{table}[H]
	\centering
	\caption[{Tabla de ingresos}]{\centering Tabla de ingresos. \textit{Fuente:} Autores.}
	\includegraphics[width=14cm]{Imagenes/TablaIngresos.png}
	\label{TablaIngresos}
\end{table}

\subsubsection*{Egresos}
\addcontentsline{toc}{subsubsection}{Egresos}

Dentro del primer año se deben tener las siguientes consideraciones: Las dos personas que van a liderar este proyecto y que, por este motivo, se cuenta con un personal reducido y cuyos pagos contaran con sus respectivas prestaciones, aportes a parafiscales, cesantías, primas y otros aspectos correspondientes a los recursos humanos de la empresa. Teniendo en cuenta la actual normativa colombiana. En otro lado se suma además la infraestructura del trabajo de forma virtual. Esto se explica con mayor detalle en la tabla \ref{TablaEgresos}.

\begin{table}[H]
	\centering
	\caption[{Tabla de egresos}]{\centering Tabla de egresos. \textit{Fuente:} Autores.}
	\includegraphics[width=15cm]{Imagenes/TablaEgresos.png}
	\label{TablaEgresos}
\end{table}