\subsection*{Amortización y depreciación}
\addcontentsline{toc}{subsection}{Amortización y depreciación}

\subsubsection*{Préstamo}
\addcontentsline{toc}{subsubsection}{Préstamo}

Para iniciar nuestro proyecto, se generó un préstamo con una entidad bancaria, la cual fue sumada al aporte social. En la tabla \ref{Prestamo} se invierte en los activos fijos iniciales del proyecto y en que condiciones.

\begin{table}[H]
	\centering
	\caption[{Préstamo}]{\centering Tabla de préstamo bancario. \textit{Fuente:} Autores.}
	\includegraphics[width=11cm]{Imagenes/Prestamo.png}
	\label{Prestamo}
\end{table}

\subsubsection*{Amortización}
\addcontentsline{toc}{subsubsection}{Amortización}

Se hace un resumen del pago realizado de forma anual, en la tabla se evidencia la amortización total de la deuda que se realiza en 4 años, de forma que se puede observar como se va a pagar la deuda dentro del período de tiempo establecido y con qué saldo final terminará para llegar a 0, teniendo en cuenta intereses del préstamo en sí.

\begin{table}[H]
	\centering
	\caption[{Amortización}]{\centering Tabla de la amortización. \textit{Fuente:} Autores.}
	\includegraphics[width=13cm]{Imagenes/Amortizacion.png}
	\label{Amortizacion}
\end{table}

\subsubsection*{Depreciación}
\addcontentsline{toc}{subsubsection}{Depreciación}

En la tabla \ref{Depreciacion} se considera la maquinaria que se va a usar junto con su deprecación a partir de su uso y la obsolescencias de los dispositivos. Se calcula entonces que la depreciación se da de aproximadamente el 20\% en un período de 5 años de vida útil.

\begin{table}[H]
	\centering
	\caption[{Depreciación}]{\centering Tabla de depreciación de dispositivos. \textit{Fuente:} Autores.}
	\includegraphics[width=13cm]{Imagenes/Depreciacion.png}
	\label{Depreciacion}
\end{table}