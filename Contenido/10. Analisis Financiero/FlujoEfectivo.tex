\subsection*{Flujo de efectivo}
\addcontentsline{toc}{subsection}{Flujo de efectivo}

Mediante el flujo de efectivo, se puede determinar la capacidad de la compañía para generar efectivo y cumplir con sus compromisos, hacer inversiones y realizar otros procesos de expansión.

Existen tres tipos de flujos de caja, los cuales contribuyen a calcular el flujo neto:

\begin{itemize}
	\item \textbf{Flujo de caja de operación:} Se obtiene desde la utilidad del estado de resultados, esto incluye la depreciación, la amortización, los intereses de la deuda y los impuestos sobre la renta.
	
	\item \textbf{Flujo de caja de inversión:} Encapsula a todas las inversiones dadas para el correcto funcionamiento de la empresa, considerando las inversiones tangibles y también las intangibles.
	
	\item \textbf{Flujo de caja de financiamiento:} Evidencia el dinero que ingresa a través de los inversionistas y otros pagos relacionados.
\end{itemize}

La suma que hace estos 3 tipos de flujos de caja genera como resultado la cantidad que la empresa genero dentro del año analizado.

\begin{table}[H]
	\centering
	\caption[{Flujo de efectivo}]{\centering Flujo de efectivo. \textit{Fuente:} Autores.}
	\includegraphics[width=15cm]{Imagenes/FlujoEfectivo.png}
	\label{FlujoEfectivo}
\end{table}