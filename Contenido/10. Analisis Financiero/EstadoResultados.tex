\subsection*{Estado de resultados}
\addcontentsline{toc}{subsection}{Estado de resultados}

El estado de resultados está proyectado a 5 años, de tal manera que, se conoce a detalle los movimientos de flujo de dinero que se hacen dentro de la empresa, teniendo en cuenta los gastos para generar las utilidades netas del año fiscal.

En la tabla \ref{EstadoResultados} se ilustra la utilidad bruta que se obtiene como la resta entre las ventas netas y el total del costo de ventas, también la utilidad operacional que es resultado de la utilidad bruta menos los gastos administrativos, por último, los gatos financieros y preopreativos que se dan por la utilidad antes de los impuestos resultado de la resta de los impuestos de renta dentro del año gravable 2026.

\begin{table}[H]
	\centering
	\caption[{Estados de resultados}]{\centering Estado de resultados. \textit{Fuente:} Autores.}
	\includegraphics[width=15cm]{Imagenes/EstadoResultados.png}
	\label{EstadoResultados}
\end{table}