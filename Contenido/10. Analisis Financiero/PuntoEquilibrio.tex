\subsection*{Punto de equilibrio}
\addcontentsline{toc}{subsection}{Punto de equilibrio}

Es importante calcular el punto de equilibrio ya que le permite a la empresa conocer cuantas veces es necesario poder cubrir los costos fijos de los cuales incurre para sus operaciones. Es de gran importancia determinar en que momento las ventas generan ganancias para la empresa. En la tabla \ref{CostosFijos} se proyecta los costos fijos y variables para las ventas del primer año.

\begin{table}[H]
	\centering
	\caption[{Costos fijos y variables}]{\centering Costos fijos y variables. \textit{Fuente:} Autores.}
	\includegraphics[width=13cm]{Imagenes/CostosFijos.png}
	\label{CostosFijos}
\end{table}

Se calcula que el costo fijo para producir cada unidad, agregando el valor que se tiene por costos administrativos y de ventas. Se proyecta tener un valor agregado del 19\% (\$84.977) de forma que se adquiera el precio de \$532.223, de forma que se debe vender 152 suscripciones para poder recuperar la inversión realizada, después de ese tope será utilidad.

\begin{table}[H]
	\centering
	\caption[{Estimaciones del punto de equilibrio}]{\centering Estimaciones del punto de equilibrio. \textit{Fuente:} Autores.}
	\includegraphics[width=11cm]{Imagenes/EstimacionesPunto.png}
	\label{EstimacionesPunto}
\end{table}

Dentro de la gráfica observamos el punto de equilibrio con rectas de costos totales y las ventas, en el eje X se encuentra el número de unidades vendidas y dentro del eje Y está la cantidad de efectivo, por lo tanto, observamos que con 152 suscripciones se igualan ambas rectas, resultando en el punto de equilibrio.

\begin{figure}[H]
	\centering
	\caption[{Gráfica del punto de equilibrio}]{\centering Gráfica del punto de equilibrio. \textit{Fuente:} Autores.}
	\includegraphics[width=15cm]{Imagenes/GraficaPunto.png}
	\label{fig:GraficaPunto}
\end{figure}

Como resultado, teniendo los resultados anteriores, hacemos un resumen con la información necesaria para el punto de equilibrio.

\begin{table}[H]
	\centering
	\caption[{Punto de equilibrio}]{\centering Punto de equilibrio. \textit{Fuente:} Autores.}
	\includegraphics[width=10cm]{Imagenes/PuntoEquilibrio.png}
	\label{PuntoEquilibrio}
\end{table}