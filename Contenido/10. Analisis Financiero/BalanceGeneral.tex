\subsection*{Balance general}
\addcontentsline{toc}{subsection}{Balance general}

Según los investigadores Charadam \& Kale \cite{Bibl044}, \textit{'El balance general, junto con los estados de ganancias y pérdidas, cambios en el patrimonio neto y los flujos de efectivo, conforman los estados financieros básicos, cuyo propósito general es poder suministrar información sobre la situación y el desempeño financiero, de la misma forma con los flujos de efectivo, que sea útil para ua amplia gama de usuario al realizar sus decisiones económicas'}

La responsabilidad de preparar y pode presentar estos estados financieros corresponde a la sección de la administración de la empresa. Por lo tanto, nos enfocaremos en el balance general, dentro del cual se ofrece información acerca de la situación financiera de la empresa al final de un período contable.

La información que contiene un balance general se clasifica de tal forma que los usuarios puedan obtener detalles acerca de la liquidez, fecha de vencimiento de los pasivos, la cantidad de los activos que se asignan a inmuebles, maquinarias y otros equipos, y la porción de los activos dadas por los propietarios y accionistas.

Estos datos sobre la liquidez de los activos se obtienen a partir de distinguir los activos entre corrientes y no corrientes. Los activos corrientes se componen de efectivo y los recursos que se espera que se conviertan en efectivo al ser vendidos o consumidos dentro de un año de un ciclo normal de operaciones (el que sea más largo). Siendo el ciclo normal de operaciones el período de tiempo que ocurre entre la compra del inventario, el procesamiento de los inventarios para venderlos, la venta de los bienes y el cobro dado por esas ventas.

La situación financiero se da mediante una serie de recursos para poder ser utilizados por la empresa, con la denominación de activos, y las demandas sobre esos recursos representada por los pasivos y el patrimonio neto.

\begin{center}
	\textit{Patrimonio = Activos - Pasivos}
\end{center}

En la tabla \ref{BalanceGeneral} se describen los activos, pasivos y el patrimonio de cada año tomando como inicio el año 0 como una inversión inicial de forma que esta represente el inicio de los procesos de nuestro negocio. En consideración con la proyección de ventas que inician desde el año contable 1 hasta el año 5.

Este balance debe ser presentado mediante aprobación de la Asamblea General de Accionistas por el representante legal junto con demás documentos hallados en el Artículo 446 del Código de Comercio. Dentro del término establecido por la ley, el representante legal remitirá a la Superintendencia, si es el caso, una copia del balance y los anexos que lo justifican y lo expliquen, además del Acta en que hubiesen sido discutidos y aprobados.

\begin{table}[H]
	\centering
	\caption[{Balance general}]{\centering Balance general. \textit{Fuente:} Autores.}
	\includegraphics[width=15cm]{Imagenes/BalanceGeneral.png}
	\label{BalanceGeneral}
\end{table}