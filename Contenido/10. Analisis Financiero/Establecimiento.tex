\subsection*{Establecimiento}
\addcontentsline{toc}{subsection}{Establecimiento}

Bajo el siguiente apartado describimos cuales costos e inversión son necesarios para crear la empresa, teniendo en cuenta las inversiones que se deben realizar para el diseño, la creación y el despliegue del software. En base a lo anterior se necesitan recursos económicos, humanos, administrativos y pre-operativos que serán descritos a continuación:

\subsubsection*{Activos fijos tangibles}
\addcontentsline{toc}{subsubsection}{Activos fijos tangibles}

Los activos fijos tangibles se dividen en 2 partes, la área operacional y el área administrativa, las cuales ambas partes son esenciales para dar inicio al proyecto. A continuación, se detalla en la tabla \ref{TablaActivosTangibles}:

\begin{table}[H]
	\centering
	\caption[{Activos tangibles}]{\centering Tabla de activos tangibles. \textit{Fuente:} Autores.}
	\includegraphics[width=13cm]{Imagenes/ActivosTangibles.png}
	\label{TablaActivosTangibles}
\end{table}

\subsubsection*{Activos fijos intangibles}
\addcontentsline{toc}{subsubsection}{Activos fijos intangibles}

Los activos fijos intangibles se denomina a los procesos financieros y legales en los cuales una empresa tiene que tener en cuenta, están descritos en la tabla \ref{TablaActivosIntangibles}.

\begin{table}[H]
	\centering
	\caption[{Activos tangibles}]{\centering Tabla de activos intangibles. \textit{Fuente:} Autores.}
	\includegraphics[width=12cm]{Imagenes/ActivosIntangibles.png}
	\label{TablaActivosIntangibles}
\end{table}

\subsubsection*{Total activos fijos}
\addcontentsline{toc}{subsubsection}{Activos fijos intangibles}

En la tabla \ref{TablaActivosFijos} se da a conocer un resumen del total de la inversión necesaria para adquirir los activos fijos, así como el capital de trabajo requerido para poder llevar a cabo el proyecto.

\begin{table}[H]
	\centering
	\caption[{Activos fijos}]{\centering Tabla de activos fijos. \textit{Fuente:} Autores.}
	\includegraphics[width=12cm]{Imagenes/ActivosFijos.png}
	\label{TablaActivosFijos}
\end{table}

\subsubsection*{Financiamiento}
\addcontentsline{toc}{subsubsection}{Financiamiento}

Dentro de la tabla \ref{TablaFinanciamiento} se muestra la inversión que necesita la fase inicial del proyecto, la cual debe evaluarse considerando tanto el aporte social como financiamiento a partir de una entidad bancaria.

\begin{table}[H]
	\centering
	\caption[{Tabla de financiamiento}]{\centering Tabla de financiamiento. \textit{Fuente:} Autores.}
	\includegraphics[width=11cm]{Imagenes/TablaFinanciamiento.png}
	\label{TablaFinanciamiento}
\end{table}


