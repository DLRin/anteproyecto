\subsection*{Alcances}
\addcontentsline{toc}{subsection}{Alcances}

El \textbf{alcance} del proyecto consiste en diseñar un plan de negocios completo para una plataforma digital de aprendizaje financiero dirigida a jóvenes (18-24 años), trabajadores informales, nómadas digitales y personas interesadas sobre temas de finanzas en Bogotá que cuentan con un bajo nivel de alfabetización financiera. Esto abarca herramientas interactivas como simuladores financieros, calculadoras personalizadas, motores de recomendaciones basados en machine learning y seguimiento de hábitos para gestionar presupuestos, deudas, ahorros e inversiones. Se incluye un análisis exhaustivo de mercado para identificar brechas en educación financiera, definición del modelo de negocio (propuesta de valor, canales, ingresos y crecimiento), proyecciones de recursos técnicos, humanos y financieros, y evaluación de viabilidad económica a corto, mediano y largo plazo.

\begin{itemize}
	\item Diseño del plan de negocios para una plataforma de aprendizaje financiero digital en Bogotá, enfocada en usuarios principiantes como jóvenes, informales y nómadas digitales.
	​
	\item Análisis de mercado que identifica necesidades y brechas financieras en segmentos meta, respaldado por datos locales y globales.
	​
	\item Definición del modelo de negocio: propuesta de valor, canales de distribución, fuentes de ingresos y estrategia de crecimiento sostenible.
	​
	\item Proyección de recursos (técnicos como machine learning, humanos y económicos) y análisis financiero de viabilidad en el corto, mediano y largo plazo.
\end{itemize}

Estos alcances delimitan claramente el enfoque del proyecto y lo que se excluye. Abarcando la concepción y planeación del servicio, pero \textbf{sin la ejecución} del desarrollo real de la plataforma ni la implementación de las herramientas, lo cual queda fuera del alcance académico del plan de negocios. Al definir estos límites se previene la confusión de requisitos y se alinea el equipo con los objetivos esperados.