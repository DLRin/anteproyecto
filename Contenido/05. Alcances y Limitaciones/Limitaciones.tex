\subsection*{Limitaciones}
\addcontentsline{toc}{subsection}{Limitaciones}
Las \textbf{limitaciones} describen las restricciones y condiciones externas que afectan la ejecución y alcance del proyecto. Entre las más relevantes se encuentran factores de recursos y tiempo, así como condicionantes externos legales o de mercado. En concreto, se consideran las siguientes limitaciones del proyecto:

\begin{itemize}
	\item \textbf{Recursos y financiamiento:} El alcance final depende de los recursos económicos y humanos disponibles. Un presupuesto limitado o plazos académicos estrictos pueden restringir la profundidad del análisis y las herramientas desarrolladas.
	
	\item \textbf{Alcance geográfico y de mercado:} El estudio se enfoca únicamente en Bogotá, por lo que no abarca poblaciones rurales u otras regiones. Además, cambios en regulaciones financieras o en las condiciones del mercado local (tasas de interés, políticas públicas, etc.) pueden alterar los supuestos del plan.
	
	\item \textbf{Enfoque y datos disponibles:} Al tratarse de un plan de negocios académico, el proyecto se basa principalmente en fuentes secundarias y supuestos, sin validación empírica en campo con usuarios reales. Esto limita la capacidad de anticipar todas las barreras operativas (por ejemplo, resistencia al cambio de usuarios o dependencia de tecnologías externas).
\end{itemize}

Estas limitaciones ayudan a entender el marco del proyecto y orientan la interpretación de los resultados. Reconocerlas de antemano permite planificar estrategias de mitigación (p.ej. priorizar ciertas investigaciones o ajustar expectativas) y garantiza que el proyecto se mantenga realista.