\subsection*{Limitaciones}
\addcontentsline{toc}{subsection}{Limitaciones}
Las \textbf{limitaciones} corresponden a restricciones externas que impactan la ejecución, como recursos limitados, tiempo académico y factores de mercado. El proyecto se basa en fuentes secundarias y supuestos, sin validación empírica primaria, lo que restringe la precisión ante variables reales como cambios regulatorios o comportamientos usuario.

\begin{itemize}
	\item \textbf{Recursos y financiamiento:} El alcance final depende de los recursos económicos y humanos disponibles. Un presupuesto limitado o plazos académicos estrictos pueden restringir la profundidad del análisis y las herramientas desarrolladas.
	
	\item \textbf{Alcance geográfico y de mercado:} Enfoque exclusivo en Bogotá, por lo que no abarca poblaciones rurales u otras regiones. Además, es sensible a cambios en las regulaciones financieras o en las condiciones del mercado local (tasas de interés, políticas públicas, etc.), lo que puede alterar los supuestos del plan.
	
	\item \textbf{Enfoque y datos disponibles:} Al tratarse de un plan de negocios académico, el proyecto depende principalmente en fuentes secundarias y supuestos, sin validación empírica en campo con usuarios reales. Esto limita la capacidad de anticipar todas las barreras operativas (por ejemplo, resistencia al cambio de usuarios o dependencia de tecnologías externas).
\end{itemize}

Reconocer estas limitaciones permiten entender el marco del proyecto y orientan la interpretación de los resultados. La identificación de estos riesgos permite planificar estrategias de mitigación (p.ej. priorizar ciertas investigaciones o ajustar expectativas), asegurando un marco profesional y realista del proyecto.