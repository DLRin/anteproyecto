\subsection*{Presupuesto}
\addcontentsline{toc}{subsection}{Presupuesto}
Dentro de la siguiente tabla podemos observar el análisis financiero, teniendo en cuenta lo descrito anteriormente en el documento en el que se consideró varios factores tales como la infraestructura, los recursos humanos, y otros costos extras los cuales haremos mención a continuación.

\begin{adjustbox}{
		center,
		caption=[{Tabla de presupuesto}]{\centering Tabla de Presupuesto.},
		label={TablaPresupuesto},
		nofloat=table, vspace={20px}}
	\resizebox{\textwidth}{!}{
		\begin{tabular}{|c|c|c|c|c|}
			\hline
			\rowcolor[HTML]{D9EAD3} 
			\textbf{Área} & \textbf{Concepto} & \textbf{Cantidad} & \begin{tabular}[c]{@{}c@{}}\textbf{Costo}\\\textbf{unitario}\end{tabular} & \begin{tabular}[c]{@{}c@{}}\textbf{Costo}\\\textbf{mensual}\end{tabular} \\ \hline
			
			\begin{tabular}[c]{@{}c@{}}\textbf{Recursos}\\\textbf{Humanos}\end{tabular} & Consultoría y mentoría & 1 & \$375.000 & \$375.000 \\ \hline
			
			\multirow{2}{*}{\begin{tabular}[c]{@{}c@{}}\textbf{Infraestructura}\\\textbf{Tecnológica}\end{tabular}} & Costos de Servidores (Cloud) & 1 & \$840.000 & \$840.000 \\ \cline{2-5}
			
			& Licencias de Software & 1 & \$420.000 & \$420.000 \\ \hline
			
			\begin{tabular}[c]{@{}c@{}}\textbf{Costos}\\\textbf{Fijos}\end{tabular} & Servicios (Luz, Internet) & - & \$420.000 & \$420.000 \\ \hline
			
			\multirow{2}{*}{\begin{tabular}[c]{@{}c@{}}\textbf{Extras}\end{tabular}} & Marketing y Lanzamiento & - & \$375.000 & \$375.000 \\ \cline{2-5}
			
			& Fondo de Imprevistos (10\%) & - & \$240.000 & \$240.000 \\ \hline
			
			\multicolumn{2}{|c|}{\textbf{TOTAL MENSUAL}} & \multicolumn{3}{|c|}{\textbf{\$2.675.000}} \\ \hline
		\end{tabular}
	}
\end{adjustbox}

Para ello se determinó un total de 4 tipos de gastos dentro de nuestro análisis financiero, los cuales son los siguientes:

\begin{itemize}
	\item \textbf{Recursos Humanos:} Se refiere al personal que se necesita para realizar el plan de negocios de forma efectiva y correcta, teniendo en cuenta los roles más imporantes y necesarios dentro del mismo.
	
	\item \textbf{Infraestructura tecnólogica:} Se refiere al costo promedio de los recursos tecnológicos que se usan, teniendo en cuenta que el modelo planteado estos costos serán inicialmente de 0 pesos y que, a medida que el proyecto va escalando, el valor que supondrá mantener la infraestructura aumentara, por esto, el valor establecido mediante el documento es un valor supuesto promedio.
	
	\item \textbf{Costos fijos:} Este tipo de gastos se definen los recursos necesarios para el desarrollo del plan de negocio, esto incluye los servicios básicos de internet y electricidad.
	
	\item \textbf{Extras:} Son todo costo asociado a la publicidad y/o marketing que pueda llegar a ser necesario para que el proyecto pueda alcanzar a una gran cantidad de público, invirtiendo para realizar campañas y aumentar las visibilidad del proyecto.
\end{itemize}