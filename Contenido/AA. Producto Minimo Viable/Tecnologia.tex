\subsection*{Tecnología}
\addcontentsline{toc}{subsection}{Tecnología}
Para lograr el desarrollo y la construcción de un prototipo mínimo viable, se comenzará a partir de una versión inicial, esto con el propósito de que se pueda cumplir con las funciones básicas del programa tales como asesoría financiera y el manejo de material multimedia. Esto teniendo en cuenta las necesidades de nuestra plataforma digital, entre ellas, se va a tener en cuenta una arquitectura de tres capas bajo las siguientes tecnologías.

\begin{figure}[H]
	\centering
	\includegraphics[width=15cm]{Imagenes/ArquiTresCapas.png}
	\caption[{Arquitectura de Tres Capas.}]{\centering Arquitectura de Tres Capas. \textit{Fuente:} Sitio Web: (Gaviria V. Raúl A., ResearchGate, 2018)} 
	\label{fig:logobconomy}
\end{figure}

Para el \textbf{Frontend} se optó por utilizar el framework Angular, desarrollado por Google, el cual ofrece una amplia gama de herramientas y componentes que facilitan la creación de interfaces de usuario dinámicas e intuitivas, mejorando así la experiencia del usuario durante la interacción con la plataforma.

Y finalmente, para el \textbf{Backend} y almacenamiento de datos se encuentra en conversaciones en usar bases de datos tradicionales como MySQL o PostgreSQL como otras herramientas como FaunaDB buscando combinar herramientas para crear una aplicación moderna y escalable.