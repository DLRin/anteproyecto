\subsection*{Conceptos de Modelo de Negocio}
\addcontentsline{toc}{subsection}{Conceptos de Modelo de Negocio}

\subsubsection*{Modelo de Negocio}
\addcontentsline{toc}{subsubsection}{Modelo de Negocio}

El modelo de negocio de una empresa es una herramienta previa al plan de negocios; su objetivo es dar a conocer con claridad el tipo de negocio que se va a crear, cómo se introducirá en el mercado, a quién se dirige, cómo se venderá y de qué manera se generarán los ingresos \parencite{Economipedia03}. 

Nuestra plataforma de aprendizaje financiero se dirige a segmentos específicos de clientes, especialmente jóvenes adultos entre los 18 y 24 años, trabajadores informales, nómadas digitales interesados en temas económicos y personas con escasos recursos. En general, un modelo de negocio se integra con los siguientes elementos, según \textcite{BusinessModel01}:

\begin{itemize}
	\item \textbf{Segmentos clave:} Corresponde a la población objetivo. Nuestro modelo de negocio se basa en un conocimiento exhaustivo de sus necesidades específicas.
	
	\item \textbf{Propuesta de valor:} Es el factor por el cual un potencial cliente decide decantarse por una u otra empresa. Su finalidad es solucionar un problema o satisfacer una necesidad del cliente. Por lo tanto, consiste en el conjunto de productos o servicios que satisfacen los requisitos de un segmento de mercado específico.
	
	\item \textbf{Canales de distribución:} Los canales de comunicación, distribución y venta buscan lograr el contacto entre la empresa y los clientes. Estos puntos de contacto desempeñan un papel fundamental en la experiencia del cliente.
	
	\item \textbf{Relación con los clientes:} Las empresas buscan definir qué tipo de relación desean establecer con cada segmento del mercado. Esto se logra mediante la captación de clientes, su fidelización y, en consecuencia, la estimulación de las ventas de los servicios o productos.
	
	\item \textbf{Fuentes de ingresos:} Se refiere al flujo de caja que genera la empresa en cada segmento del mercado. Existen distintas formas de ingreso, tales como transacciones derivadas de pagos puntuales realizados por los clientes y pagos periódicos efectuados a cambio del suministro de una propuesta de valor o de servicios posventa dirigidos al cliente.
	
	\item \textbf{Actividades clave:} Son las actividades esenciales que debe implementar una empresa para alcanzar el éxito. Resultan necesarias para crear y ofrecer una propuesta de valor, alcanzar los mercados objetivo, establecer relaciones con los clientes y generar ingresos. Estas actividades varían según la naturaleza del modelo de negocio.
	
	\item \textbf{Recursos clave:} Los modelos de negocio requieren recursos clave que permiten a la empresa crear y ofrecer propuestas de valor, llegar a los mercados, establecer relaciones con los segmentos y generar ingresos. Estos recursos pueden ser físicos, económicos, intelectuales o humanos. Además, la empresa puede poseerlos, alquilarlos u obtenerlos a través de sus socios clave.
	
	\item \textbf{Socios clave:} Son las alianzas estratégicas que la empresa establece con otras organizaciones o personas con el objetivo de optimizar su modelo de negocio, reducir riesgos y adquirir recursos adicionales.
	
	\item \textbf{Estructura de costos:} Describe los principales costos en los que incurre la empresa al operar bajo un modelo de negocio determinado. La creación y entrega de valor, el mantenimiento de las relaciones con los clientes y la generación de ingresos implican costos asociados.
\end{itemize}

\subsubsection*{Modelo Canvas}
\addcontentsline{toc}{subsubsection}{Modelo Canvas}

El Business Model Canvas (también conocido como diagrama Canvas) es una herramienta de gestión estratégica desarrollada por Alexander Osterwalder e Yves Pigneur, cuyo objetivo es estructurar de forma visual los elementos clave de un negocio, como la propuesta de valor, los clientes, la infraestructura y las finanzas \parencite{BusinessModel01}. Diversos estudios resaltan que incluir un modelo de negocio en un plan empresarial es fundamental para validar y fortalecer la idea antes de su implementación \parencite{CEUPE01}.

El modelo Canvas es lo suficientemente sencillo para aplicarse a cualquier tipo de empresa, ya sea pequeña, mediana o grande, independientemente de su estrategia de negocio y público objetivo. A continuación, se presenta la estructura del modelo Canvas, en la cual se expone la forma en que se organiza el modelo de negocio:

\begin{figure}[H]
	\centering
	\includegraphics[width=15cm]{Imagenes/PlantillaModeloCanvas.jpg}
	\caption[{Ejemplo de modelo Canvas.}]{\centering Ejemplo de modelo Canvas.
		\textit{Fuente:} ¿Cómo hacer un modelo de negocios Canvas?, Redacción CEUPE, 2022. Origen: https://www.ceupe.do/blog/como-hacer-un-modelo-de-negocios-canvas.html)} 
	\label{fig:planmodelocanvas}
\end{figure}

En este marco, nuestra plataforma de aprendizaje financiero dirigida a personas en Bogotá surge como una propuesta innovadora que ofrece herramientas didácticas y dinámicas para la planificación financiera personal. La iniciativa se complementa con un enfoque de gamificación ligera, el uso de machine learning y la construcción de una comunidad de educación financiera, con el objetivo de motivar el aprendizaje, promover hábitos financieros saludables y fomentar la inclusión financiera.