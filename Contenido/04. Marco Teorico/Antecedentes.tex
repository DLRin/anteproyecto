\subsection*{Antecendentes}
\addcontentsline{toc}{subsection}{Antecedentes}

\subsubsection*{Nivel regional}
\addcontentsline{toc}{subsubsection}{Nivel regional}
\begin{itemize}
	\item \textbf{Marco Regulatorio:} Dentro de nuestro país se estableció el marco legal integral para la educación financiera que se convirtió en uno de los referentes de la región. Bajo la Ley 1328 de 2009 la cual establece que la educación financiera debe ser un derecho para los consumidores colombianos, obliga a las instituciones financieras y bancos a buscar promover, entregar y capacitar programas de educación financiera de acorde a las instrucciones dadas por la Superintendencia Financiera de Colombia \parencite[p. 10]{Superintendencia01}. Además la Ley 1450 de 2011 dentro del Plan Nacional de Desarrollo capacitó al Ministerio de Educación Nacional a buscar definir habilidades financieras y económicas básicas las cuales se deberán incluir dentro de los currículos escolares colombianos \parencite[p. 15]{Superintendencia01}.
	
	Junto a estas leyes, el Decreto 457 de 2014 creó un sistema multiagencial con el objetivo de coordinar las iniciativas públicas y privadas acerca de la educación financiera, buscando establecer una Estrategia Nacional para la Educación Económica y Financiera (ENEEF). Con este marco normativo sitúa a nuestro país como uno de los pioneros en institucionalizar la educación financiera.
	
	\item \textbf{Niveles de Alfabetización Financiera:} Investigaciones conjuntas entre el Banco de la República y organismos internacionales revelan que solo el \textbf{16,4\%} de los colombianos responde acertadamente preguntas sobre conceptos financieros fundamentales \parencite[p. 22]{BancoMundial01}. Si bien el 79\% comprende el efecto de la inflación, menos de la mitad domina el cálculo de interés compuesto o los beneficios de la diversificación de inversiones, lo que evidencia brechas cognitivas que impactan directamente el bienestar económico \parencite[p. 25]{ComisionIntersectorial01}.
	
	\item \textbf{Ecosistema Fintech y Digital:} El ecosistema \textit{fintech} colombiano ha mostrado una maduración estructural, registrando un total de 394 startups locales al cierre del primer cuatrimestre de 2024, lo que representó un crecimiento del 6,8\% respecto al año anterior \parencite[p. 8]{Finnovista2024}. Según \textcite{Baca2025}, hacia el año 2025 el ecosistema ha dejado de medirse únicamente por la cantidad de nuevas empresas para enfocarse en la escalabilidad, sostenibilidad financiera e impacto real; de hecho, el 30\% de las \textit{fintech} que operan en el país son extranjeras, consolidando a Colombia como un hub de expansión regional.
	
	Colombia logró posicionarse como uno de los países más avanzados de América Latina en términos regulatorios, situándose al nivel de mercados como Chile o Brasil. Este avance se debe a la gestión de la Superintendencia Financiera de Colombia, la cual ha sido pionera en la implementación de un sandbox regulatorio que permite la experimentación controlada de nuevos modelos de negocio \parencite[p. 2]{Gaitan2023}. Asimismo, el desarrollo del marco de Finanzas Abiertas (Open Finance), formalizado mediante el Decreto 1297 de 2022, ha permitido la estandarización de APIs y ha promovido acuerdos colaborativos entre la banca tradicional y las \textit{fintech}, buscando reducir asimetrías regulatorias y fomentar la competencia en beneficio del consumidor \parencite[p. 3]{Gaitan2023}.
	
	Como se comentó anteriormente, casos exitosos en el entorno nacional son Nequi y Daviplata, los cuales han logrado revolucionar la forma en la que los colombianos gestionan su capital al reducir la dependencia del efectivo y simplificar las transacciones mediante dispositivos electrónicos \parencite[p. 1]{Gaitan2023}. Estas plataformas digitales no solo facilitan el acceso, sino que incorporan funcionalidades técnicas de planificación financiera, tales como alertas de pago, objetivos de ahorro programados e incluso herramientas para la planeación y manejo de cuotas de crédito, consolidándose como líderes en usuarios activos dentro del ecosistema de neobancos en Colombia \parencite{Pulzo01, Gaitan2023}.
	
	\item \textbf{Programas de Educación Financiera:} Colombia cuenta con un ecosistema de educación económica y financiera (EEF) en consolidación. Según el mapeo nacional del Banco de la República de 2024, se identificaron 141 entidades que ofrecen activamente estos programas \parencite[p. 2]{Banrep_Mapeo2024}. Un hallazgo crítico de este reporte es la transición hacia canales digitales: las charlas virtuales pasaron de una adopción del 86\% en 2024, mientras que el uso de videos educativos creció al 72,8\% en el mismo periodo \parencite[p. 10]{Banrep_Mapeo2024}. A pesar de este despliegue, la oferta sigue concentrada en temáticas básicas como el ahorro y el presupuesto, lo que señala una oportunidad para el desarrollo de herramientas tecnológicas que aborden la planificación financiera avanzada y la gestión de riesgos \parencite[p. 5]{Banrep_Mapeo2024}.
	
	\begin{figure}[H]
		\centering
		\includegraphics[width=10cm]{Imagenes/LogoVivaSeguro.jpg}
		\caption[{Logotipo de Viva Seguro.}]{\centering Logotipo de Viva Seguro. \textit{Fuente:} Viva Seguro - Programa de Eduación Financiera, Fasecolda, 2025, Origen: https://www.vivasegurofasecolda.com/)} 
		\label{fig:logovivaseguro}
	\end{figure}
	
	\item \textbf{Modelo de Planificación Financiera y Desafíos:} Las investigaciones recientes proponen el desarrollo de modelos que unifiquen la planificación financiera personal bajo elementos adaptados al contexto colombiano. No obstante, se reconoce una desventaja competitiva al intentar replicar modelos de economías desarrolladas, debido a la ausencia de una industria de asesoría financiera local lo suficientemente madura \parencite{RevistaCIES01}. En este sentido, la literatura señala desafíos estructurales que limitan el impacto de las soluciones actuales:

	\begin{itemize}
		\item \textbf{Efectividad y evaluación:} A pesar de la robusta oferta institucional, los programas vigentes presentan dificultades en la medición de su impacto real a largo plazo, debido a la falta de sistemas integrales de monitoreo y evaluación de resultados \parencite{Banrep_Mapeo2024}.
		
		\item \textbf{Concentración geográfica:} Se evidencia una marcada centralización del ecosistema, donde la mayoría de las iniciativas de educación financiera y servicios Fintech se concentran en Bogotá y la región Central, limitando el acceso en zonas periféricas como la Amazonía y la Orinoquía \parencite[p. 8]{Banrep_Mapeo2024}.
		
		\item \textbf{Brecha digital y demográfica:} Aunque el uso de canales digitales ha crecido exponencialmente —alcanzando una adopción del 86\% en modalidad de charlas virtuales \parencite[p. 10]{Banrep_Mapeo2024}—, persisten barreras críticas de acceso. Según el \textcite{DANE_Regiones}, las poblaciones vulnerables, los trabajadores informales y las mujeres enfrentan los mayores obstáculos de inclusión, lo que demanda el diseño de herramientas que mitiguen las asimetrías tecnológicas y sociales actuales \parencite{Banrep_Mapeo2024, DANE_Regiones}.
	\end{itemize}
\end{itemize}

\subsubsection*{Nivel global}
\addcontentsline{toc}{subsubsection}{Nivel global}
Las herramientas digitales de asesoría financiera han proliferado en los últimos años, ofreciendo a usuarios inexpertos acceso a consejos personalizados. Por ejemplo, las \textbf{plataformas digitales} pueden desempeñar un rol crucial para mejorar la educación financiera entre los grupos tradicionalmente marginados, como las poblaciones de bajos recursos \parencite[p. 28]{RevistaCientifica01}. Un caso destacado es la asistente virtual Erica de Bank of America, que ha atendido a casi 50 millones de usuarios con más de 2 mil millones de consejos personalizados sobre gastos, presupuestos y ahorros \parencite{FundsSociety01}. 

Sin embargo, estudios señalan que los robo-advisors (asesores automáticos) suelen ``compensar'' la baja cultura financiera del usuario ofreciendo inversiones pasivas, sin realmente mejorar la alfabetización financiera del cliente \parencite[p. 5-6]{RoboAdvisors01}. Por lo general, se coincide en que las plataformas deben ir acompañadas de una educación adaptada; se destaca que la eficacia de la educación financiera depende del acceso equitativo a la información, el uso de herramientas digitales y los enfoques personalizados.

\begin{itemize}
	\item \textbf{Aplicaciones móviles de educación financiera:} Existen apps educativas como \textit{Finanzoo, Mint, Fintonic} o \textit{Fortune City} que enseñan conceptos básicos, así como el seguimiento de gastos y ahorros. También hay simuladores financieros en línea y juegos interactivos (por ejemplo, simuladores de presupuesto o inversión) diseñados para principiantes.
	
	\item \textbf{Chatbots y asistentes virtuales:} Además de Erica, diversas entidades globales han lanzado chatbots basados en IA para consultas básicas. Estos bots responden preguntas sobre el estado de cuenta y ofrecen recomendaciones de ahorro o análisis del perfil crediticio. Pese a su popularidad (las interacciones automatizadas suman miles de millones), su éxito radica más en la eficiencia operativa que en una educación financiera profunda.
	
	\item \textbf{Robo-Advisors y asesoría automatizada:} El mercado de los robo-advisors está experimentando un crecimiento exponencial. El surgimiento de plataformas como \textit{Betterment, Wealthfront o Scalable Capital} ha permitido ofrecer planes de inversión automatizados a bajos costos. Aunque facilitan el acceso a inversiones, diversos estudios señalan que su impacto en la alfabetización financiera es limitado, pues no sustituyen el aprendizaje personalizado de finanzas básicas \parencite[p. 5-6]{RoboAdvisors01}.
\end{itemize}

\paragraph*{Niveles e Iniciativas Mundiales de Alfabetización Financiera}
\addcontentsline{toc}{paragraph}{Niveles e Iniciativas Mundiales de Alfabetización Financiera}

La encuesta S\&P Global FinLit Survey, que tomó en cuenta a más de 150 mil adultos en más de 140 economías, reveló que, a nivel global, solamente 1 de cada 3 adultos es financieramente alfabetizado. Esta encuesta evaluó conocimientos básicos de cuatro conceptos fundamentales: tasas de interés, capitalización de intereses, inflación y diversificación de riesgos \parencite[p. 21]{FinLit01}.

En esta misma encuesta se demostró que, dentro de los países desarrollados, en promedio el 55\% de los adultos en las principales economías avanzadas como Canadá, Francia, Alemania y Estados Unidos son financieramente alfabetizados, con rangos que van desde el 37\% en Italia hasta el 68\% en Canadá. Mientras que, en economías emergentes, este porcentaje es mucho menor, con un promedio del 28\% de adultos financieramente alfabetizados, con rangos que van desde el 24\% en India hasta el 42\% en Sudáfrica \parencite[p. 23-25]{FinLit01}. 

Para aumentar este porcentaje, a nivel institucional, organismos internacionales y gobiernos promueven programas y políticas orientadas a elevar la cultura financiera. La OCDE señala que, tras la crisis financiera de 2008, muchos países integraron la educación financiera en su agenda nacional. Ejemplos incluyen campañas nacionales, portales web y cursos en escuelas. Iniciativas emblemáticas incluyen:

\begin{itemize}
	\item \textbf{Marco global (OCDE/BID):} La OCDE/INFE promueve principios y encuestas internacionales sobre alfabetización financiera adulta. En América Latina, \textcite[p. 4]{FinLit01} remarca que la educación financiera acompaña un mejor desempeño en la planificación y el ahorro para la jubilación, menores tasas de interés y menor endeudamiento, además de contribuir al crecimiento económico al capacitar a nuevas clases medias y poblaciones vulnerables. En Colombia, por ejemplo, se han impulsado campañas mundiales como la Global Money Week para promover estrategias de educación financiera entre los más jóvenes \parencite{BancadeOportunidades03}.
	
	\item \textbf{Programas de gobierno en países en desarrollo:} Varios gobiernos han lanzado centros, cursos y concursos de finanzas personales. En India, el RBI promueve y organiza semanas anuales de alfabetización financiera y concursos escolares (``Financial Literacy Week'') que alcanzan a decenas de miles de estudiantes \parencite{BancodeIndia01}. En África y el Caribe, ONG como la Organización Internacional para las Migraciones (OIM) han apoyado apps educativas; por ejemplo, el proyecto «\$mart Finance» (2016) desarrolló una aplicación para jóvenes caribeños de 15 a 35 años para aprender sobre presupuestos y planificación de modo interactivo \parencite{IOM01}.
	
	\item \textbf{Programas en países desarrollados:} En Singapur, el programa nacional MoneySense ofrece recursos multimedia para la educación financiera. Iniciada en 2003, esta iniciativa se consolidó como un servicio digital gratuito (a través de MyMoneySense) que utiliza datos financieros personales (mediante la infraestructura SGFinDex) y brinda recomendaciones personalizadas de salud financiera \parencite{MoneySense01}. Del mismo modo, países como el Reino Unido (MoneyHelper), Australia (ASIC’s Moneysmart) y Canadá (Agencia de Consumidor Financiero) disponen de portales y simuladores en línea sobre aprendizaje financiero.
\end{itemize}

\paragraph*{Participación del sector privado y casos de éxito}
\addcontentsline{toc}{paragraph}{Participación del sector privado y casos de éxito}

El sector financiero y la industria fintech también impulsan soluciones educativas y de asesoría. Por ejemplo, iniciativas de bancos globales incorporan módulos formativos en sus apps: BBVA, Citi y Santander ofrecen tutoriales financieros a sus clientes (aunque estos esfuerzos son difíciles de cuantificar académicamente). Más innovador ha sido el surgimiento de los ``finfluencers'' en redes sociales, creadores de contenido que enseñan sobre finanzas, aprovechando espacios vacíos dejados por la banca tradicional y alcanzando a millones de jóvenes, especialmente aquellos fuera del alcance de servicios formales, a pesar del riesgo asociado con la calidad de la información que difunden \parencite{RTVE01}. Algunos casos destacados son:

\begin{itemize}
	\item \textbf{Estados Unidos:} La consultora Accenture y la Reserva Federal han destacado el rápido crecimiento de las herramientas digitales. Por ejemplo, \textit{Erica} de Bank of America no solo alcanzó 50 millones de usuarios, sino que también contribuye a reducir costes de call center y a mejorar hábitos de gasto (el 98\% de los usuarios encuentran útiles sus respuestas) \parencite{FundsSociety01}. También plataformas emergentes como \textit{Cleo} o \textit{Plum} ofrecen chatbots financieros gamificados para presupuestar y ahorrar.
	
	\item \textbf{América Latina:} Entre los casos se encuentra el desarrollo de varias iniciativas digitales para la educación e inclusión financiera en distintos países latinoamericanos. Por ejemplo, la República Dominicana creó la aplicación móvil \textit{ProUsuario}, que integra el historial crediticio del usuario con consejos financieros personalizados \parencite{RDominicana01}. Estas iniciativas demuestran que las herramientas tecnológicas accesibles potencian la inclusión financiera \parencite[p. 28]{RevistaCientifica01}.
	
	\item \textbf{Programas en países desarrollados:} En Singapur, el programa nacional MoneySense ofrece recursos multimedia y, desde 2020, la plataforma \textbf{MyMoneySense}, un servicio digital gratuito que consolida datos financieros personales (a través de la infraestructura SGFinDex) y brinda recomendaciones personalizadas de salud financiera \parencite{MoneySense01}. Del mismo modo, países como el Reino Unido (MoneyHelper), Australia (ASIC’s Moneysmart) y Canadá (Agencia de Consumidor Financiero) disponen de portales y simuladores en línea. El sector público y privado cooperan frecuentemente; por ejemplo, en Malasia, el Ministerio de Finanzas promueve enfoques holísticos que combinan educación financiera digital con inclusión y protección del consumidor \parencite{Malasia01}.
	
	\item \textbf{Asia y el Pacífico:} El ejemplo de Singapur con MyMoneySense destaca como éxito público-privado; al combinar datos bancarios con guías objetivas, ha logrado reducir barreras para que ciudadanos inexpertos inicien su planificación financiera \parencite{MoneySense01}. También se ubican a la vanguardia apps fintech como \textit{Monefy} en India o \textit{ABC Money} en China, orientadas a la alfabetización de audiencias masivas.
	
	\item \textbf{Europa:} Aunque los estudios académicos son escasos, la UE financia campañas transnacionales y el fomento de simuladores en línea. Un ejemplo es FINANCEnLAB, un proyecto que involucra empresas de España, Alemania, Rumanía y Polonia con el objetivo de mejorar la educación financiera de estudiantes y profesionales mediante el uso de herramientas digitales \parencite{UniversidadValladolid01}.
	
	\item \textbf{África:} Se observa un aumento de proyectos apoyados por agencias de desarrollo. La Fundación Mastercard, por ejemplo, financia entrenamientos financieros para pymes en Ruanda. Adicionalmente, empresas africanas como \textit{Vault22} en Sudáfrica permiten a los usuarios visualizar gastos y elaborar presupuestos, empoderándolos con datos contables claros \parencite{Vault01}.
\end{itemize}

En conjunto, los antecedentes mundiales muestran que la educación y asesoría financiera para no expertos se ha abordado mediante una combinación de tecnología (apps, chatbots, plataformas digitales), iniciativas públicas (estrategias nacionales, campañas) y esfuerzos de sectores privados y comunitarios. Los casos de éxito evidencian que la personalización, el soporte continuado (incluso gamificado) y el apoyo institucional conducen a una mayor inclusión financiera y a mejores decisiones individuales.
