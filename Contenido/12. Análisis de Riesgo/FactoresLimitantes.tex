\subsection*{Factores limitantes}
\addcontentsline{toc}{subsection}{Factores limitantes}

Los factores limitantes son aquellos elementos u ocasiones las cuales impiden el progreso y logro de un objetivo dado dentro de la misma empresa. Estos factores pueden ser internos o externos y pueden estar relacionados hacia la falta de recursos, la competencia, las capacidades del equipo, o el entorno externo.

\begin{enumerate}
	\item \textbf{Costo de adquirir usuarios:} Debido a que es una nueva plataforma, se necesitará una inversión grande en marketing para poder atraer a los clientes interesados en temas financieros. Esto significa que en las primeras etapas, los costos por adquisición de usuarios podrían ser elevados especialmente en el caso en el que no se logren crear campañas publicitarias altamente efectivas.
	
	\item \textbf{Dependencia de oferta inicial:} Para que nuestra compañía sea atractiva, es esencial atraer una gran cantidad de clientes. En caso contrario, si la plataforma no logra atraer suficientes usuarios, entonces es más complicado generar el impulso necesario para sostener el negocio de forma rentable.
	
	\item \textbf{Competencia:} Es posible el riesgo de que competidores con mayor trayectoria lancen herramientas diseñadas para el público objetivo, lo que nos exigirá diferenciar nuestro servicio de ellos y reforzar la propuesta de valor con el paso del tiempo.
\end{enumerate}