\section*{Análisis de Riesgo}
\addcontentsline{toc}{section}{Análisis de Riesgo}

A la hora de crear nuestra compañía, se neceitan que los procesos de gestión y planificación puedan contar con un análisis de riesgos, puesto que estos análisis nos permiten identificar los riesgos de distinto índole que puedan afectar negativamente a la empresa o a las que se pueda exponer, por esta razón, crear estrategias que puedan mitigar aquellos riesgos o reducir la probabilidad de ser afectado por ellos.

El análisis de riesgos busca examinar la magnitud y la índole de aquellos efectos negativos de la introudcción propuesta, así como la probabilidad de que éstos se puedan producir. Deberá identificar medios eficaces para disminuir aquellos riesgos y buscar alternativas a la introducción propuesta.

\subsection*{Factores limitantes}
\addcontentsline{toc}{subsection}{Factores limitantes}

Los factores limitantes son aquellos elementos u ocasiones las cuales impiden el progreso y logro de un objetivo dado dentro de la misma empresa. Estos factores pueden ser internos o externos y pueden estar relacionados hacia la falta de recursos, la competencia, las capacidades del equipo, o el entorno externo.

\begin{enumerate}
	\item \textbf{Costo de adquirir usuarios:} Debido a que es una nueva plataforma, se necesitará una inversión grande en marketing para poder atraer a los clientes interesados en temas financieros. Esto significa que en las primeras etapas, los costos por adquisición de usuarios podrían ser elevados especialmente en el caso en el que no se logren crear campañas publicitarias altamente efectivas.
	
	\item \textbf{Dependencia de oferta inicial:} Para que nuestra compañía sea atractiva, es esencial atraer una gran cantidad de clientes. En caso contrario, si la plataforma no logra atraer suficientes usuarios, entonces es más complicado generar el impulso necesario para sostener el negocio de forma rentable.
	
	\item \textbf{Competencia:} Es posible el riesgo de que competidores con mayor trayectoria lancen herramientas diseñadas para el público objetivo, lo que nos exigirá diferenciar nuestro servicio de ellos y reforzar la propuesta de valor con el paso del tiempo.
\end{enumerate}

\subsection*{Factores claves de éxito}
\addcontentsline{toc}{subsection}{Factores claves de éxito}

Los factores claves del éxito (FCE) son las características del producto que son valorados de gran manera por un grupo de clientes y en las que, entonces, la compañía debe de tener éxito para superar a los competidores. La identificación de factores claves de éxito (FCE) en una empresa es necesaria para comprender qué partes ayudan al rendimiento superior y la sostenibilidad de la empresa hacia largo plazo. Varios estudios han investigado tales hechos en distintos contextos y sectores, dando una visión integral de los elementos que puedan influir dentro del éxito empresarial. \cite{Bibl043}.

Algunos de estos factores que puedan presentarse en nuestra empresa son los siguientes:

\begin{enumerate}
	\item \textbf{Estrategia de marketing digital:} Implementar campañas de marketing efectivas principalmente enfocadas a nuestro público objetivo mediante redes sociales, influencers y publicidad digital, esto será esencial para atraer a personas interesadas en la educación financiera. La estrategia deberá destacar la facilidad de uso, el manejo de las recomendaciones de inversión y el manejo de elementos audiovisuales.
	
	\item \textbf{Construcción de comunidad:} Fomentar el inicio de una comunidad activa de usuarios que puedan interactuar dentro y fuera de la plataforma, compartir sus experiencias personales y recomendar la plataforma, esto será la base para poder construir fundamentos sólidos y generar lealtad en los usuarios.
	
	\item \textbf{Escalabilidad técnica:} Buscamos asegurarnos que la plataforma logre ser escalable desde el punto de vista tecnológico haciendo que esta pueda crecer sin tener complicaciones desde la operación o técnica a medida que se aumente la cantidad de usuarios.
	
	\item \textbf{Calidad de la plataforma:} Debido a que se permite usar la plataforma de forma limitada en su versión gratis, entonces se busca asegurar generar buenas impresiones en nuestros usuarios de forma que decidan pagar la suscripción para mayores capacidades.
\end{enumerate}


\subsection*{Riesgos legales}
\addcontentsline{toc}{subsection}{Riesgos legales}

Los riesgos legales hablan acerca de ocasiones en las cuales la compañía podría tener consecuencias legales como efecto de acciones o decisiones que se podría tomar como ilegales o inapropiadas. Estos riesgos se manifiestan en varias situaciones, tales como transacciones comerciales, relaciones laborales, cumplimiento de regulaciones vigentes, cuestiones dadas con los derechos de propiedad intelectual, problemas fiscales, seguridad de la información y muchas otras áreas más. Las repercusiones de aquellos escenarios pueden generar la amenaza de demandas civiles, imposición de multas o sanciones por parte de las entidades gubernamentales, e inclusive, en el peor de los casos, poder enfrentar cargos penales. 

Dentro de nuestra empresa se identifica lo siguiente:

\begin{enumerate}
	\item \textbf{Protección de datos personales:} Debido a que es un servicio de planificación financiera, la empresa va a manejar información de datos financieros y personales altamente sensibles. Además de métodos de pago con el tema de las suscripciones. Para mitigar esto, nuestra empresa debe asegurarse de cumplir con la protección de datos y establecer protocolos de seguridad informática robustos.
	
	\item \textbf{Cumplimiento normativo:} Debido al tipo de compañía que somos, estamos sujetos a tener que operar con los registros dados por la Superintendencia Financiera y tener que cumplir las regulaciones emergentes del ecosistema fintech, para mitigar esto, nos aseguraremos de implementar correctamente estos registros a las autoridades pertinentes tales como la SARLAFT y además consultar a la Superintendencia para averiguar que tipo de actividades necesitan autorización, vigilancia o registro en específico.
	
	\item \textbf{Derechos de propiedad:} Implica los posibles conflictos generados con el manejo inadecuado de la protección de derechos de autor de los desarrollos tecnológicos propios (como el contenido educativo) y la implementación de derechos de terceros. Para ello, tendremos que establecer políticas claras para no alterar aquellos derechos.
\end{enumerate}


\subsection*{Riesgos operacionales}
\addcontentsline{toc}{subsection}{Riesgos operacionales}

La gestión de riesgos operacionales se definen como el proceso cíclico continuo que se extiende a evaluar riesgos, tomar decisiones de riesgos e implementar controles de riesgos, lo que genera como resultado la aceptación, mitigación o elusión de riesgos, esta gestión es la supervisión constante del riesgo operativo, incluyendo el riesgo de pérdida dada por los procesos y sistemas internos fallidos, factores humanos u otros eventos externos.

\begin{enumerate}
	\item \textbf{Fallas técnicas y de vulnerabilidades:} Se refiere a que la infraestructura de nuestra plataforma debe ser lo suficientemente robusta para evitar caídas del sistema o problemas de rendimiento especialmente en momentos de alta demanda junto a vulnerabilidades halladas. Para ello, se buscará implementar una infraestructura rígida e implementar protocolos de ciberseguridad y hacer pruebas constantemente.
	
	\item \textbf{Gestión del crecimiento:} A medida que se expanda las operaciones de nuestra compañía, un crecimiento acelerado y sin ser calculado podría generar problemas para la gestión de la plataforma, desde la atención hacia el cliente como la calidad del servicio podrían ser afectadas por ellos, por tal razón, nosotros nos comprometemos a hacer una expansión gradual y calculada de las operaciones de la plataforma a otras partes.
	
	\item \textbf{Incumplimiento de objetivos:} Se refiere a la incapacidad de la empresa para cumplir con los objetivos y plazos dados para iniciar una nueva parte o proyecto. Para mitigar esto, se compromete a hacer un correcto y completo estudio de proyectos que ingresen a la empresa. Así como un análisis de los miembros implicados dentro del proyecto.
\end{enumerate}


