\subsection*{Riesgos legales}
\addcontentsline{toc}{subsection}{Riesgos legales}

Los riesgos legales hablan acerca de ocasiones en las cuales la compañía podría tener consecuencias legales como efecto de acciones o decisiones que se podría tomar como ilegales o inapropiadas. Estos riesgos se manifiestan en varias situaciones, tales como transacciones comerciales, relaciones laborales, cumplimiento de regulaciones vigentes, cuestiones dadas con los derechos de propiedad intelectual, problemas fiscales, seguridad de la información y muchas otras áreas más. Las repercusiones de aquellos escenarios pueden generar la amenaza de demandas civiles, imposición de multas o sanciones por parte de las entidades gubernamentales, e inclusive, en el peor de los casos, poder enfrentar cargos penales. 

Dentro de nuestra empresa se identifica lo siguiente:

\begin{enumerate}
	\item \textbf{Protección de datos personales:} Debido a que es un servicio de planificación financiera, la empresa va a manejar información de datos financieros y personales altamente sensibles. Además de métodos de pago con el tema de las suscripciones. Para mitigar esto, nuestra empresa debe asegurarse de cumplir con la protección de datos y establecer protocolos de seguridad informática robustos.
	
	\item \textbf{Cumplimiento normativo:} Debido al tipo de compañía que somos, estamos sujetos a tener que operar con los registros dados por la Superintendencia Financiera y tener que cumplir las regulaciones emergentes del ecosistema fintech, para mitigar esto, nos aseguraremos de implementar correctamente estos registros a las autoridades pertinentes tales como la SARLAFT y además consultar a la Superintendencia para averiguar que tipo de actividades necesitan autorización, vigilancia o registro en específico.
	
	\item \textbf{Derechos de propiedad:} Implica los posibles conflictos generados con el manejo inadecuado de la protección de derechos de autor de los desarrollos tecnológicos propios (como el contenido educativo) y la implementación de derechos de terceros. Para ello, tendremos que establecer políticas claras para no alterar aquellos derechos.
\end{enumerate}
