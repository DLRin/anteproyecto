\subsection*{Riesgos operacionales}
\addcontentsline{toc}{subsection}{Riesgos operacionales}

La gestión de riesgos operacionales se definen como el proceso cíclico continuo que se extiende a evaluar riesgos, tomar decisiones de riesgos e implementar controles de riesgos, lo que genera como resultado la aceptación, mitigación o elusión de riesgos, esta gestión es la supervisión constante del riesgo operativo, incluyendo el riesgo de pérdida dada por los procesos y sistemas internos fallidos, factores humanos u otros eventos externos.

\begin{enumerate}
	\item \textbf{Fallas técnicas y de vulnerabilidades:} Se refiere a que la infraestructura de nuestra plataforma debe ser lo suficientemente robusta para evitar caídas del sistema o problemas de rendimiento especialmente en momentos de alta demanda junto a vulnerabilidades halladas. Para ello, se buscará implementar una infraestructura rígida e implementar protocolos de ciberseguridad y hacer pruebas constantemente.
	
	\item \textbf{Gestión del crecimiento:} A medida que se expanda las operaciones de nuestra compañía, un crecimiento acelerado y sin ser calculado podría generar problemas para la gestión de la plataforma, desde la atención hacia el cliente como la calidad del servicio podrían ser afectadas por ellos, por tal razón, nosotros nos comprometemos a hacer una expansión gradual y calculada de las operaciones de la plataforma a otras partes.
	
	\item \textbf{Incumplimiento de objetivos:} Se refiere a la incapacidad de la empresa para cumplir con los objetivos y plazos dados para iniciar una nueva parte o proyecto. Para mitigar esto, se compromete a hacer un correcto y completo estudio de proyectos que ingresen a la empresa. Así como un análisis de los miembros implicados dentro del proyecto.
\end{enumerate}


