\section*{Conclusiones}
\addcontentsline{toc}{section}{Conclusiones}

El plan de negocios de EduFinanzas expone una propuesta consistente y debidamente sustentada, orientada a facilitar el acceso al ámbito financiero a personas que no suelen interactuar con este sector. A través de un análisis detallado de los factores que inciden en su desarrollo, fue posible identificar tanto oportunidades como riesgos asociados a este tipo de iniciativa, apoyándose en los distintos elementos estratégicos abordados previamente.

En términos generales, EduFinanzas además de permitir introducir a los usuarios en el mundo financiero, también logra contribuir a una mejor administración de los recursos económicos en la ciudad, gracias a su facilidad de uso y al enfoque personalizado en la atención a los clientes. Esto genera un impacto favorable en la dinámica económica local.

Para alcanzar estos objetivos, se empleó el modelo de negocio CANVAS, el cual facilitó la definición de una estructura organizacional clara y una visión empresarial integral. Este modelo permitió resaltar la propuesta de valor, el fortalecimiento de la relación con los clientes y la identificación de las actividades clave, entre otros aspectos relevantes. De esta forma, se establece una base sólida para la formulación de estrategias que permitan adaptarse al mercado y atender las características de diversos segmentos, como estudiantes universitarios y trabajadores informales.

El análisis financiero complementa esta perspectiva, evidenciando que, aunque se presentan pérdidas en el corto plazo, a largo plazo se proyectan resultados favorables en relación con la inversión inicial y los costos operativos. Se espera que el flujo de caja se torne positivo conforme aumente la captación de clientes, garantizando la sostenibilidad del negocio. Asimismo, las proyecciones de ventas reflejan un crecimiento continuo de carácter exponencial, impulsado por una estrategia de marketing clara y una adecuada segmentación del público objetivo. El análisis del punto de equilibrio indica que la empresa puede alcanzar la rentabilidad en un período razonable, lo cual resulta atractivo para potenciales inversionistas del sector.

Por medio de la matriz PEYEA, se identificaron oportunidades relevantes para el negocio, como el uso creciente de plataformas digitales y redes sociales, así como amenazas relacionadas con la alta competencia existente. De igual forma, la matriz DOFA permitió evaluar de manera integral el entorno interno y externo de la empresa. Entre las fortalezas se destaca la disponibilidad de un equipo técnico capacitado en el desarrollo de sistemas y en el uso de arquitecturas por capas, mientras que las debilidades se reflejan principalmente en un flujo de caja negativo durante el primer año de operación.

Finalmente, se desarrolló un prototipo mínimo viable de EduFinanzas en una plataforma web, incorporando funcionalidades definidas a partir de los casos de uso y requerimientos identificados. Este prototipo fue construido utilizando patrones arquitectónicos modernos y tecnologías actuales, garantizando la calidad del software y una experiencia de usuario óptima. En consecuencia, la plataforma presenta un alto potencial no solo para transformar el sector financiero, sino también para generar un impacto social positivo y promover prácticas sostenibles dentro de la industria.