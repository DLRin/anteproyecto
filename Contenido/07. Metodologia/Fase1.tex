\subsection*{Fase 1: Presentación y compromiso del equipo}
\addcontentsline{toc}{subsection}{Fase 1: Presentación y compromiso del equipo}

\begin{itemize}
	\item Formación del equipo base.
	
	\item Formación del equipo de trabajo definitivo.
	
	\item Identificación de las áreas de análisis.
	
	\item Presentación del proyecto, procesos y metodologías.
	
	\item Documento de presentación aprobado (Anteproyecto). 
\end{itemize}

En la próxima tabla se establecen los distintos objetivos de la fase asociada con la presentación y el compromiso del grupo u equipo de forma que se pueda proyectar ciertos resultados que se esperan por medio de las actividades especificas de la fase.

\begin{adjustbox}{
		center,
		caption=[{Descripción de la fase 1.}]{\centering Descripción de la fase 1. \textit{Fuente:} Autores.},
		label={TablaFase1},
		nofloat=table, vspace={20px}}
	\resizebox{\textwidth}{!}{
		\begin{tabular}{|c|c|c|c|}
			\hline
			\rowcolor[HTML]{D9EAD3} 
			\multicolumn{1}{|l|}{\cellcolor[HTML]{D9EAD3}\textbf{Fase 1}} &
			\multicolumn{1}{l|}{\cellcolor[HTML]{D9EAD3}\textbf{Objetivos Especificos}} &
			\multicolumn{1}{l|}{\cellcolor[HTML]{D9EAD3}\textbf{Actividades}} &
			\multicolumn{1}{l|}{\cellcolor[HTML]{D9EAD3}\textbf{Resultados Esperados}} \\ \hline
			&  & Conformación del equipo       & \begin{tabular}[c]{@{}c@{}}Tener claro el papel de cada uno\\  de los integrantes del equipo\end{tabular} \\ \cline{3-4} 
			&  & Búsqueda inicial de literatura & Análisis del contexto                                                                                      \\ \cline{3-4} 
			\multirow{-4}{*}{\begin{tabular}[c]{@{}c@{}}Presentación y \\ compromiso del \\ equipo\end{tabular}} &
			\multirow{-4}{*}{\begin{tabular}[c]{@{}c@{}}Dar a conocer los integrantes del equipo\\  e iniciar con el análisis de la situación,\\ planteamiento del problema\\  y proyectarnos trabajando de la mano.\end{tabular}} &
			Definir el planteamiento del problema &
			Definición precisa del problema \\ \hline
		\end{tabular}
	}
\end{adjustbox}