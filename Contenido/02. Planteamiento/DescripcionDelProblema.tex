\subsection*{Descripción del Problema}
\addcontentsline{toc}{subsection}{Descripción del Problema}
En Colombia, el acceso a servicios financieros ha avanzado en cobertura, claramente observado por el porcentaje de adultos que tienen algún producto financiero, el cual es del 94.6\% en el año 2023. Con más de 30.8 millones de adultos teniendo al menos una cuenta de ahorro para ese mismo año \parencite{BancadeOportunidades01}.

Sin embargo, persiste una brecha notable en la comprensión y gestión efectiva del manejo del dinero en el tiempo, a través de un análisis de comparación hecho por el Banco Mundial, bajo una escala de 0 (menor capacidad) a 100 (mayor capacidad) se encontró que actividades relacionadas con respecto al ahorro y el seguimiento de los gastos obtuvieron los puntajes más bajes mientras que la capacidad de cubrir los gastos imprevistos y la impulsividad se encontraron en el medio \parencite[p.~34-35]{BancoMundial01}. Adicionalmente, con respecto a conocimientos financieros, en aquel mismo informe, se evidencio que las personas no eran capaces de hacer un cálculo de tasa de interés (con una tasa de éxito del 35\%) y solamente el 26\% lograron responder sobre el concepto de interés compuesto. Por lo que esta falta de compresión hace dudar de la capacidad de las personas para tomar decisiones totalmente conscientes sobre productos financieros \parencite[p.~29]{BancoMundial01}.

%pendiente por redactar%
%Desde la perspectiva de la Ingeniería de Sistemas, esta problemática se agrava por la falta de ecosistemas digitales que integren herramientas de \textit{adaptive learning} y personalización basada en datos, lo que impide una cultura de ahorro estratégica.%

Esta situación, se ve agravada por la ausencia de una plataforma digital que combine las condiciones actuales del usuario, la falta de herramientas interactivas y la personalización del usuario en sí, por lo que, esta carencia contribuye a la falta de seguimiento, ahorro y manejo de conceptos financieros. Por lo cual se vuelve fundamental abordar esta problemática para buscar fomentar una cultura de ahorro personalizada y estratégica.