\section*{Modelo de Negocio}
\addcontentsline{toc}{section}{Modelo de Negocio}

Para la fase de desarrollo de nuestro modelo de negocio, llevamos a cabo la elección de un patrón para distinguir entre los distintos grupos de clientes a los que se acoge nuestro proyecto. Para ello seguimos el marco de Alexander Osterwalder y Yves Pigneur, autores del libro "Generación de modelos de negocio". Se decidió por elegir esta opción ya que nos permitió visualizar y gestionar las relaciones complejas existentes en nuestro entorno empresarial.

\begin{figure}[H]
	\centering
	\includegraphics[width=15cm]{Imagenes/ModeloCanvas.png}
	\caption[{Modelo CANVAS.}]{\centering Modelo CANVAS. \textit{Fuente:} Autores.} 
	\label{fig:modelocanvasproy}
\end{figure}

La Figura \ref{fig:modelocanvasproy} presenta una visualización concisa acerca de la estructura de nuestro modelo de negocios. Realizada mediante el Modelo Canvas previamente definido, dentro de ella se establece el propósito principal de nuestro proyecto mediante una breve descripción, la forma en la que se financiará nuestro proyecto, el tipo de cliente que se dirige nuestro negocio y cuál es la propuesta de valor que ofrecemos al mercado. Mediante la figura se nos da una guía sólida para la toma de decisiones estratégicas y una mayor comunicación efectiva para nuestro enfoque empresarial. Siendo un paso fundamental para entender y avanzar con la implementación de nuestro proyecto.

\subsection*{Segmento del Mercado (SM)}
\addcontentsline{toc}{subsection}{Segmento del Mercado (SM)}

El mercado objetivo de la plataforma se ha delimitado mediante criterios demográficos, socioeconómicos y de comportamiento laboral en el contexto de Bogotá, identificando tres nichos estratégicos que presentan una necesidad crítica de planificación financiera personalizada:

\begin{itemize}
	\item \textbf{Jóvenes en etapa de formación y vinculación laboral (18 a 24 años):} Este segmento se caracteriza por tener el nivel más bajo de alfabetización financiera en el país, con apenas un 18,2\% de respuestas correctas en competencias básicas \parencite[p. 5]{Departamento01}. La plataforma se dirige a estudiantes y jóvenes profesionales en Bogotá que requieren herramientas digitales para la transición hacia la independencia económica.
	
	\item \textbf{Nómadas digitales y profesionales independientes:} Segmento compuesto por trabajadores remotos que, de acuerdo con \textcite{Rodriguez2025}, enfrentan una "encrucijada" en la autogestión de su seguridad social y la volatilidad de sus ingresos. Estos usuarios demandan modelos de planificación dinámica que permitan proyectar ahorros y cumplimiento de obligaciones legales en un entorno de economía digital y transfronteriza \parencite{RevistaCIES01}.
	
	\item \textbf{Población con ingresos informales en Bogotá (Estratos 1, 2 y 3):} Dirigido a ciudadanos que forman parte del 43,1\% de la población informal de la capital \parencite[p. 1]{DANE01}. Este grupo requiere soluciones tecnológicas que adapten el aprendizaje financiero a flujos de caja irregulares, facilitando el paso de una inclusión financiera nominal (tenencia de cuenta) a una bancarización efectiva y estratégica \parencite[p. 7]{Departamento01}.
\end{itemize}

\subsection*{Propuesta de Valor (PV)}
\addcontentsline{toc}{subsection}{Propuesta de Valor (PV)}
En este modulo buscamos proporcionar un diferenciador que responda a las necesidades y los deseos de los clientes, y que logre aportar ventajas sobre la competencia. Nuestra propuesta se centra en una plataforma de aprendizaje financiero interactiva y personalizada para jóvenes (18-24 años), trabajadores informales y nómadas digitales en Bogotá, que integra machine learning para rutas de aprendizaje adaptativas y seguimiento objetivo de hábitos. Facilitando a los usuarios a entender el mundo financiero de una forma mucho más didáctica, motivada y personal.


\subsection*{Canales de Distribución (CD)}
\addcontentsline{toc}{subsection}{Canales de Distribución (CD)}
Es la forma en la que especificamos como será el contacto con los clientes, para ello, el modelo de distribución de nuestro proyecto será principalmente a través de una aplicación móvil y sitio web intuitivo. Apoyos en redes sociales (haciendo presencia en plataformas donde se concentra el público objetivo) y marketing de contenidos para captación. Junto a ello se contemplan alianzas con universidades, ONGs y fintech locales para difusión (talleres, seminarios). La alta penetración de smartphones y conexión fija en Bogotá (29 accesos fijos por cada 100 hab.) garantiza cobertura.

\subsection*{Relaciones con Clientes (CLI)}
\addcontentsline{toc}{subsection}{Relaciones con Clientes (CLI)}
Este apartado busca enfatizar lo fundamental de establecer relaciones personalizadas con cada uno de nuestros clientes como parte integral del proyecto. Para ello, dentro de nuestro proyecto buscaremos establecer un servicio mayormente automatizado (alertas y seguimiento automatizado) combinado con atención opcional personalizada (chat con asesores o foros de ayuda). Se busca generar confianza mediante transparencia (informes simples) y reforzar el aprendizaje por retroalimentación positiva (insignias, retos mensuales).

\subsection*{Fuentes de Ingresos (FI)}
\addcontentsline{toc}{subsection}{Fuentes de Ingresos (FI)}
Es importante definir las fuentes de ingresos para garantizar la sostenibilidad del proyecto. La fuente de ingreso de nuestro proyecto será establecer un modelo \textit{freemium}, donde la versión básica es gratuita, y se ofrecerá una suscripción premium la cual desbloquea funciones avanzadas (planificación detallada, asesoría personalizada). También se consideran alianzas pagas con empresas y publicidad segmentada (e.g. ofertas de productos financieros adecuados al perfil del usuario). Esta combinación de suscripciones y colaboraciones corporativas es común en apps fintech exitosas.

\subsection*{Actividades Clave (AC)}
\addcontentsline{toc}{subsection}{Actividades Clave (AC)}
Se centra en describir las actividades importantes que se debe emprender dentro de nuestro proyecto, las cuales son el desarrollo y mantenimiento del software (aplicación y servidor cloud), creación de contenido didáctico digital, actualización continua de datos y algoritmos de IA. Incluye análisis de datos de uso para personalizar recomendaciones y ejecutar campañas de mercadeo digital. A fin de enfatizar la inclusión, se monitorean métricas de uso en poblaciones vulnerables.

\subsection*{Recursos Clave (RC)}
\addcontentsline{toc}{subsection}{Recursos Clave (RC)}
Esta sección detalla los insumos que se necesitan para el funcionamiento del negocio, como se dijo anteriormente, se necesitará un equipo técnico (desarrolladores, diseñadores UX/UI, especialistas en datos) y expertos financieros para contenido. Infraestructura en la nube para escalabilidad y bases de datos financieras actualizadas. Asimismo, las tecnologías de IA/ML son fundamentales para los recordatorios inteligentes y las sugerencias automáticas.

\subsection*{Socios Clave (SC)}
\addcontentsline{toc}{subsection}{Socios Clave (SC)}
Se busca establecer la red de proveedores y socios estratégicos importante que colaborarán con el funcionamiento y visibilidad del modelo de negocios, en nuestro proyecto se consideran alianzas con instituciones educativas (para validar el contenido), asociaciones de trabajadores informales, entidades públicas (p. ej. programas de inclusión financiera), y bancos/fintech (proporcionan APIs y datos agregados). Estos socios amplían el alcance y fortalecen la credibilidad de la plataforma. De hecho, el modelo se alinea con las tendencias regionales de apoyo fintech e inclusión financiera.

\subsection*{Estructura de Costos(EC)}
\addcontentsline{toc}{subsection}{Estructura de Costos(EC)}
Se detallan los costos asociados al inicio de operaciones del modelo de negocio. En el caso de nuestro proyecto los costos principales provienen de desarrollo tecnológico (salarios de equipo, servidores, licencias de software), producción de contenido y marketing. Costos variables incluyen comisiones por servicios en la nube y eventual servicio de atención al cliente. La estrategia es mantener bajos costos fijos a escala mediante uso eficiente de la tecnología, el open-source y la colaboración con aliados.