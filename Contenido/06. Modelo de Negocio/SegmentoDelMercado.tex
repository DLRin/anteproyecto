\subsection*{Segmento del Mercado (SM)}
\addcontentsline{toc}{subsection}{Segmento del Mercado (SM)}

El mercado objetivo de la plataforma se ha delimitado mediante criterios demográficos, socioeconómicos y de comportamiento laboral en el contexto de Bogotá, identificando tres nichos estratégicos que presentan una necesidad crítica de planificación financiera personalizada:

\begin{itemize}
	\item \textbf{Jóvenes en etapa de formación y vinculación laboral (18 a 24 años):} Este segmento se caracteriza por tener el nivel más bajo de alfabetización financiera en el país, con apenas un 18,2\% de respuestas correctas en competencias básicas \parencite[p. 5]{Departamento01}. La plataforma se dirige a estudiantes y jóvenes profesionales en Bogotá que requieren herramientas digitales para la transición hacia la independencia económica.
	
	\item \textbf{Nómadas digitales y profesionales independientes:} Segmento compuesto por trabajadores remotos que, de acuerdo con \textcite{Rodriguez2025}, enfrentan una "encrucijada" en la autogestión de su seguridad social y la volatilidad de sus ingresos. Estos usuarios demandan modelos de planificación dinámica que permitan proyectar ahorros y cumplimiento de obligaciones legales en un entorno de economía digital y transfronteriza \parencite{RevistaCIES01}.
	
	\item \textbf{Población con ingresos informales en Bogotá (Estratos 1, 2 y 3):} Dirigido a ciudadanos que forman parte del 43,1\% de la población informal de la capital \parencite[p. 1]{DANE01}. Este grupo requiere soluciones tecnológicas que adapten el aprendizaje financiero a flujos de caja irregulares, facilitando el paso de una inclusión financiera nominal (tenencia de cuenta) a una bancarización efectiva y estratégica \parencite[p. 7]{Departamento01}.
\end{itemize}