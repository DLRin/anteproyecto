\subsection*{Matriz DOFA}
\addcontentsline{toc}{subsection}{Matriz DOFA}

La matriz DOFA (también conocida como matriz FODA, o matriz DAFO) es una herramienta de análisis estratégico usada dentro del ámbito empresarial y para la toma de decisiones. La siglas D.O.F.A. proviene de las iniciales de los cuatro elementos clave que se analizan en esta matriz, las cuales son las \textit{Debilidades, Oportunidades, Fortalezas y Amenazas}. 

\begin{figure}[H]
	\centering
	\includegraphics[width=12cm]{Imagenes/MatrizDOFAEjemplo.png}
	\caption[{Ejemplo de matriz DOFA.}]{\centering Ejemplo de matriz DOFA. \textit{Fuente:} Sitio Web: (Pasos para implementar fácilmente la matriz DOFA de tu negocio, Actualícese, 2023)} 
	\label{fig:matrizdofaejemplo}
\end{figure}

Cada elemento dentro de la matriz DOFA tiene el siguiente significado:

\begin{itemize}
	\item \textbf{Debilidades (D):} Dentro de este apartado se identifican los factores internos que pueden representar problemas o limitaciones. Estas limitaciones pueden incluir carencias en los recursos, falta de habilidades en específico, procesos ineficientes u cualquier otro factor interno que pueda evitar o entorpecer el éxito de la compañía.
	
	\item \textbf{Oportunidades (O):} Acá se enumeran las circunstancias y aspectos externos que pueden favorecer nuestra empresa. Estas oportunidades pueden ser por ejemplo las tendencias del mercado, los cambios en la demanda de productos o servicios, avances tecnológicos, nuevas regulaciones y leyes gubernamentales y otras partes que puede usar la compañía para beneficiarse.
	
	\item \textbf{Fortalezas (F):} Las fortalezas son aquellos puntos fuertes dentro de nuestra institución, tales como los recursos disponibles, las competencias, ventajas competitivas y los activos valiosos. Estos elementos nos ayudan positivamente al desempeño de nuestra empresa y estas son capaces de ser los fundamentos de nuestro éxito dentro del mercado.
	
	\item \textbf{Amenazas (A):} Las amenazas son los factores externos que representan riesgos hacia la compañía. Estas amenazas son o pueden ser la competencia feroz, cambios económicos desfavorables, fluctuaciones dentro del mercado, cambios en las preferencias de los consumidores u otro elementos que posiblemente afecte negativamente a la institución.
\end{itemize}

Con esta definición, se inicia la construcción de los distintos aspectos de la empresa para lograr hacer después un análisis de aquella matriz en la figura \ref{fig:matrizdofaresultado}.

\begin{figure}[H]
	\centering
	\includegraphics[width=15cm]{Imagenes/ResultadoMatrizDOFA.png}
	\caption[{Resultado de matriz DOFA.}]{\centering Resultado de matriz DOFA. \textit{Fuente:} Autores.} 
	\label{fig:matrizdofaresultado}
\end{figure}