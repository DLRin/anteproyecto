\subsection*{Análisis DOFA}
\addcontentsline{toc}{subsection}{Análisis DOFA}

Con base al análisis dado previamente hecho con la matriz DOFA, se identifican una serie de estrategias que buscan como meta capitalizar las fortalezas, abordar las debilidades, aprovechar las oportunidades y gestionar las amenazas. Estas estrategias se dividen en cuatro categorías las cuales son: \textit{F.O.} (Fortalezas y Oportunidades), \textit{D.O.} (Debilidades y Oportunidades), \textit{F.A.} (Fortalezas y Amenazas) y \textit{D.A.} (Debilidades y Amenazas). A continuación, detallamos las estrategias.

\subsubsection*{Estrategias FO (Fortalezas-Oportunidades)}
\addcontentsline{toc}{subsubsection}{Estrategias FO (Fortalezas-Oportunidades)}

\begin{itemize}
	\item Desarrollar alianzas estratégicas con instituciones educativas y gremios para validar contenido y ampliar alcance, aprovechando el apoyo gubernamental a inclusión financiera.
	
	\item Capitalizar el crecimiento fintech y segmento de nómadas digitales implementando funcionalidades específicas para ingresos variables y manejo de múltiples divisas.
	
	\item Utilizar la infraestructura cloud eficiente para lanzar rápidamente contenidos educativos interactivos (videos, gamificación, simuladores) que atiendan la brecha de alfabetización financiera.
\end{itemize}

\subsubsection*{Estrategias DO (Debilidades-Oportunidades)}
\addcontentsline{toc}{subsubsection}{Estrategias DO (Debilidades-Oportunidades)}

\begin{itemize}
	\item Ofrecer versión gratuita robusta del servicio para construir base de usuarios rápidamente aprovechando alta demanda educativa sin depender exclusivamente de conversiones premium.
	
	\item Aprovechar programas gubernamentales (Banca de las Oportunidades, Global Money Week) para obtener financiamiento complementario y validación institucional que compense liquidez inicial limitada.
	
	\item Establecer alianzas con instituciones financieras y fintech consolidadas (colaboraciones con Bancolombia, BBVA, Trii) para ganar reconocimiento de marca y credibilidad rápidamente.
\end{itemize}

\subsubsection*{Estrategias FA (Fortalezas-Amenazas)}
\addcontentsline{toc}{subsubsection}{Estrategias FA (Fortalezas-Amenazas)}

\begin{itemize}
	\item Diferenciarse de neobancos y competidores mediante enfoque educativo profundo y personalización que bancos masivos no pueden ofrecer por su escala.
	
	\item Implementar estrategia de inbound marketing y contenido de valor en redes sociales (TikTok, Instagram, YouTube) para reducir costos de adquisición frente a presión competitiva.
	
	\item Diseñar modelos de suscripción flexibles y accesibles (COP 19,000/mes con versión freemium robusta) adaptados a trabajadores informales con ingresos variables.
\end{itemize}

\subsubsection*{Estrategias DA (Debilidades-Amenazas)}
\addcontentsline{toc}{subsubsection}{Estrategias DA (Debilidades-Amenazas)}

\begin{itemize}
	\item Mantener estructura de costos mínima usando tecnologías open-source y esquema operativo lean para sobrevivir en entorno competitivo durante la fase crítica inicial (meses 1-15).
	
	\item Desarrollar modelos de monetización alternativos (alianzas corporativas como beneficio para empleados, publicidad segmentada de productos financieros) reduciendo dependencia exclusiva de suscripciones.
	
	\item Enfocarse en nicho específico (jóvenes 18-30 años sin conocimientos financieros, nómadas digitales, trabajadores informales) evitando competencia directa con gigantes establecidos.
\end{itemize}