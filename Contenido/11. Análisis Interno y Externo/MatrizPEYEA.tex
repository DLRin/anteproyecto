\subsection*{Matriz PEYEA}
\addcontentsline{toc}{subsection}{Matriz PEYEA}

La Matriz PEYEA \textit{(Posición Estratégica y Evaluación de la Acción)} es una herramienta de gran importancia que permite identificar a las empresas tipo de estrategias más adecuadas dentro de una organización, con el objetivo de tomar mejores decisiones a un nivel estratégico con el menor riesgo posible, y que, al mismo tiempo, pueda mejorar sus probabilidades de crecimiento económico.

Esta matriz utiliza unidades de análisis para poder representar los ejes de la matriz, estas unidades se representan en dimensiones internas y externas. Las cuales son las siguientes:

\begin{itemize}
	\item \textbf{Internas:} La unidades internas son las fortalezas financieras (F.F.) que se encarga de observar la fortaleza económica de nuestra empresa. Mientras que las ventajas competitivas (V.C.), buscan poder observar la situación de nuestra competencia y poder analizar si tienen superioridad en sus estrategias y economía o no.
	
	\item \textbf{Externas:} Las unidades externas son la estabilidad del ambiente (E.A.) que analiza la estabilidad actual de la empresa, y finalmente la fortaleza industrial (F.I.), que busca analizar la fuerza actual de la industria.
\end{itemize}

Bajo la calificación de estas unidades internas y externas, se identifica la mejor estrategia para una empresa, posicionándola dentro de uno de los cuatro cuadrantes estratégicos: \textit{agresivo, conservador, defensivo o competitivo}. 

\begin{figure}[H]
	\centering
	\includegraphics[width=12cm]{Imagenes/MatrizPeyeaCuadrantes.png}
	\caption[{Ejemplo de matriz PEYEA.}]{\centering Ejemplo de matriz PEYEA. \textit{Fuente:} Autores.}
	\label{fig:matrizpeyeacuadrantes}
\end{figure}

Dependiendo de dónde se situé la organización dentro de la matriz, se puede recomendar una estrategia. Por ejemplo, si la empresa se encuentra dentro del cuadrante agresivo, es recomendable aprovechar las oportunidades del mercado para crecer, mientras que, en comparación, se encuentra en el cuadrante defensivo, la empresa debería centrarse en mejorar sus debilidades y protegerse de amenazas externas. La descripción de todos los cuadrantes se encuentra a continuación:

\begin{itemize}
	\item \textbf{Cuadrante 1 (Agresivo):} Si la empresa se encuentra acá significa que tiene una fuerza financiera y, por lo tanto, puede tener condiciones para tomar riesgos que puedan hacer frente a las debilidades que están tomando lugar a nivel interno.
	
	\item \textbf{Cuadrante 2 (Conservadora):} Si el vector se ubica acá se busca que la organización no tome riesgos y valore crear planes en caso de problemas o emergencias.
	
	\item \textbf{Cuadrante 3 (Defensivo):} Si el vector señala este cuadrante, la organización debe	mantenerse lejos de las amenazas y buscar focos en la disminución de todas las debilidades internas que presenta.
	
	\item \textbf{Cuadrante 4 (Competitivo):} Este cuadrante significa que la compañía puede perseguir	estrategias altamente competitiva y que es su momento de arriesgar, usar inversiones e implementar acciones de alcance comercial mucho más ambiciosas.
\end{itemize}

Los pasos que son necesarios para elaborar una matriz PEYEA son los siguientes:

\begin{enumerate}
	\item Seleccionar una serie de variables para cada una de las unidades de análisis, es decir, las fortalezas financieras, las ventajas competitivas, la estabilidad del ambiente y la fortaleza industrial.
	
	\item Asignar un valor numérico a cada una de las variables que integran las dimensiones de fortalezas financieras (F.F.) y fortaleza industrial (F.I.) con un valor que va desde el 1 al 6 siendo el +1 como el peor y +6 como el mejor, Adicionalmente, también asignamos valores numéricos a las variables que integran las dimensiones de estabilidad del ambiente (E.A.) y ventajas competitivas (V.C.) con un valor que va desde el -1 al -6 donde el -1 es el mejor y -6 como el peor.
	
	\item Calcular un puntaje promedio para cada una de las dimensiones, sumando los valores asignados a las variables y dividiéndolo por el total de ellas.
	
	\item Registrar los puntajes promedio de las dimensiones correspondientes a la matriz PEYEA.
	
	\item Sumar los dos puntajes del eje x y registrar el punto resultante en X. Y del mismo modo, sumar los dos puntos del eje y, y registrar el punto resultante en Y. Registrando la intersección del nuevo punto XY.
	
	\item Dibujar un vector direccional desde el origen del cuadro de la matriz PEYEA que pase hacia el nuevo punto de intersección. Este vector nos revela que tipo de estrategia se recomienda dependiendo de la ubicación del vector sobre el cuadrante (agresiva, conservadora, defensiva o competitiva).
\end{enumerate}

En consecuencia, la matriz PEYEA, que se ha obtenido es la siguiente, dividida en aquellas dimensiones internas y externas:

\subsubsection*{Dimensión interna}
\addcontentsline{toc}{subsubsection}{Dimensión interna}

\paragraph*{Fortalezas financieras}
\addcontentsline{toc}{paragraph}{Fortalezas financieras}

\begin{longtable}{|c|c|}
	\caption[{Calificación de fortalezas financieras.}]
	{\centering Calificación de fortalezas financieras. \textit{Fuente:} Autores.}
	\label{TablaFortalezasFinancieras} \\ \hline
	
	\rowcolor[HTML]{5398F5} 
	\multicolumn{2}{|c|}{\textbf{Posición estratégica interna}} \\ \hline
	\rowcolor[HTML]{ABCCF7} 
	\textbf{Fortalezas financieras (F.F.)} & \textbf{Calificación (1 a 6)} \\ \hline
	\endfirsthead
	
	\multicolumn{2}{c}%
	{{\bfseries \tablename\ \thetable{} -- continuación}} \\ \hline
	\rowcolor[HTML]{5398F5} 
	\multicolumn{2}{|c|}{\textbf{Posición estratégica interna}} \\ \hline
	\rowcolor[HTML]{ABCCF7} 
	\textbf{Fortalezas financieras (F.F.)} & \textbf{Calificación (1 a 6)} \\ \hline
	\endhead
	
	\hline \multicolumn{2}{r}{{Continúa en la siguiente página}} \\ \hline
	\endfoot
	
	\hline
	\endlastfoot
	
	Capital inicial disponible y acceso a financiamiento & 4 \\ \hline
	Flujo de caja proyectado y punto de equilibrio & 4 \\ \hline
	Rentabilidad esperada del modelo de suscripción & 5 \\ \hline
	Capacidad de recuperación de la inversión inicial & 4 \\ \hline
	Estructura de costos operativos eficiente & 5 \\ \hline
	Liquidez para sostener operaciones en fase inicial & 3 \\ \hline
	Diversificación de fuentes de ingreso & 4 \\ \hline
	Margen de utilidad proyectado & 4 \\ \hline
	Capacidad de endeudamiento y apalancamiento & 4 \\ \hline
	\rowcolor[HTML]{D6D6D6} 
	\textbf{Promedio} & \textbf{4.11} \\ \hline
\end{longtable}

\paragraph*{Ventajas competitivas}
\addcontentsline{toc}{paragraph}{Ventajas competitivas}

\begin{longtable}{|c|c|}
	\caption[{Calificación de ventajas competitivas.}]
	{\centering Calificación de ventajas competitivas. \textit{Fuente:} Autores.}
	\label{TablaVentajasCompetitivas} \\ \hline
	
	\rowcolor[HTML]{5398F5} 
	\multicolumn{2}{|c|}{\textbf{Posición estratégica interna}} \\ \hline
	\rowcolor[HTML]{ABCCF7} 
	\textbf{Ventajas competitivas (V.C.)} & \textbf{Calificación (1 a 6)} \\ \hline
	\endfirsthead
	
	\multicolumn{2}{c}%
	{{\bfseries \tablename\ \thetable{} -- continuación}} \\ \hline
	\rowcolor[HTML]{5398F5} 
	\multicolumn{2}{|c|}{\textbf{Posición estratégica interna}} \\ \hline
	\rowcolor[HTML]{ABCCF7} 
	\textbf{Ventajas competitivas (V.C.)} & \textbf{Calificación (1 a 6)} \\ \hline
	\endhead
	
	\hline \multicolumn{2}{r}{{Continúa en la siguiente página}} \\ \hline
	\endfoot
	
	\hline
	\endlastfoot
	
	Diferenciación por enfoque educativo personalizado & -2 \\ \hline
	Experiencia del equipo de desarrollo tecnológico & -2 \\ \hline
	Calidad del servicio de asesoría financiera & -2 \\ \hline
	Accesibilidad y facilidad de uso de la plataforma & -2 \\ \hline
	Innovación en gamificación y contenidos interactivos & -2 \\ \hline
	Participación de mercado actual vs competidores & -5 \\ \hline
	Lealtad y retención de usuarios & -3 \\ \hline
	Costos de adquisición de clientes & -4 \\ \hline
	Tiempo de respuesta y soporte al cliente & -3 \\ \hline
	\rowcolor[HTML]{D6D6D6} 
	\textbf{Promedio} & \textbf{-2.78} \\ \hline
\end{longtable}

\subsubsection*{Dimensión externa}
\addcontentsline{toc}{subsubsection}{Dimensión externa}

\paragraph*{Fortaleza industrial}
\addcontentsline{toc}{paragraph}{Fortaleza industrial}

\begin{longtable}{|c|c|}
	\caption[{Calificación de fortalezas industriales.}]
	{\centering Calificación de fortalezas industriales. \textit{Fuente:} Autores.}
	\label{TablaFortalezaIndustrial} \\ \hline
	
	\rowcolor[HTML]{FFCF00} 
	\multicolumn{2}{|c|}{\textbf{Posición estratégica externa}} \\ \hline
	\rowcolor[HTML]{FFEA8C} 
	\textbf{Fortalezas Industriales (F.I.)} & \textbf{Calificación (1 a 6)} \\ \hline
	\endfirsthead
	
	\multicolumn{2}{c}%
	{{\bfseries \tablename\ \thetable{} -- continuación}} \\ \hline
	\rowcolor[HTML]{FFCF00} 
	\multicolumn{2}{|c|}{\textbf{Posición estratégica externa}} \\ \hline
	\rowcolor[HTML]{FFEA8C} 
	\textbf{Fortalezas Industriales (F.I.)} & \textbf{Calificación (1 a 6)} \\ \hline
	\endhead
	
	\hline \multicolumn{2}{r}{{Continúa en la siguiente página}} \\ \hline
	\endfoot
	
	\hline
	\endlastfoot
	
	Potencial de crecimiento del mercado fintech en Colombia & 5 \\ \hline
	Demanda de educación financiera en Bogotá & 6 \\ \hline
	Estabilidad y madurez del sector tecnológico & 5 \\ \hline
	Disponibilidad de talento especializado & 4 \\ \hline
	Acceso a infraestructura tecnológica & 5 \\ \hline
	Apoyo gubernamental a inclusión financiera & 4 \\ \hline
	Facilidad de entrada a nuevos mercados & 4 \\ \hline
	Nivel de consolidación de la industria & 3 \\ \hline
	Rentabilidad general del sector fintech & 4 \\ \hline
	\rowcolor[HTML]{D6D6D6} 
	\textbf{Promedio} & \textbf{4.44} \\ \hline
\end{longtable}

\paragraph*{Estabilidad del ambiente}
\addcontentsline{toc}{paragraph}{Estabilidad del ambiente}

\begin{longtable}{|c|c|}
	\caption[{Calificación de estabilidad del ambiente.}]
	{\centering Calificación de estabilidad del ambiente. \textit{Fuente:} Autores.}
	\label{TablaEstabilidadAmbiente} \\ \hline
	
	\rowcolor[HTML]{FFCF00} 
	\multicolumn{2}{|c|}{\textbf{Posición estratégica externa}} \\ \hline
	\rowcolor[HTML]{FFEA8C} 
	\textbf{Estabilidad del ambiente (E.A.} & \textbf{Calificación (-1 a -6)} \\ \hline
	\endfirsthead
	
	\multicolumn{2}{c}%
	{{\bfseries \tablename\ \thetable{} -- continuación}} \\ \hline
	\rowcolor[HTML]{FFCF00} 
	\multicolumn{2}{|c|}{\textbf{Posición estratégica externa}} \\ \hline
	\rowcolor[HTML]{FFEA8C} 
	\textbf{Estabilidad del ambiente (E.A.)} & \textbf{Calificación (-1 a -6)} \\ \hline
	\endhead
	
	\hline \multicolumn{2}{r}{{Continúa en la siguiente página}} \\ \hline
	\endfoot
	
	\hline
	\endlastfoot
	
	Volatilidad económica y tasas de inflación en Colombia & -3 \\ \hline
	Cambios regulatorios en el sector fintech & -2 \\ \hline
	Competencia de bancos tradicionales y neobancos & -4 \\ \hline
	Barreras de entrada tecnológicas y regulatorias & -3 \\ \hline
	Cambios en tasas de interés y política monetaria & -3 \\ \hline
	Nivel de informalidad laboral en el mercado objetivo & -4 \\ \hline
	Riesgo de saturación del mercado fintech & -3 \\ \hline
	Presión de precios por competencia & -3 \\ \hline
	Estabilidad tecnológica y ciberseguridad & -2 \\ \hline
	\rowcolor[HTML]{D6D6D6} 
	\textbf{Promedio} & \textbf{-3.00} \\ \hline
\end{longtable}

\subsubsection*{Resultados matriz PEYEA}
\addcontentsline{toc}{subsubsection}{Resultados matriz PEYEA}

Ahora lo que hacemos en base a estos resultados es determinar los valores de las coordenadas (x,y) lo que nos permitirá crear el gráfico que va a señalar en qué cuadrante se sitúa la empresa. Para realizar esto, vamos a usar las siguientes ecuaciones:

\begin{equation}
	x = FI + VC
\end{equation}

\begin{equation}
	y = FF + EA
\end{equation}

Por esto, lo que vamos a hacer es reemplazar estos valores con los hallados dentro de la matriz PEYEA, con ello obtenemos los siguientes resultados:

\begin{equation}
	x = 4.44 + -2.78 = 1.66
\end{equation}

\begin{equation}
	y = 4.11 + -3.00 = 1.11
\end{equation}

Estos resultados, representados gráficamente, aparecen en la siguiente imagen:

\begin{figure}[H]
	\centering
	\includegraphics[width=12cm]{Imagenes/ResultadoMatrizPEYEA.png}
	\caption[{Resultado de matriz PEYEA.}]{\centering Resultado de matriz PEYEA. \textit{Fuente:} Autores.}
	\label{fig:matrizpeyearesultado}
\end{figure}

Por lo tanto, como en conclusión, nuestra empresa se ubica dentro del primer cuadrante, esto significa que está dentro del \textbf{cuadrante agresivo} y por lo tanto, nos permite usar sus fortalezas internas para lograr un adecuado uso de las oportunidades externas que se presenten, esquivando amenazas del ambiente y debilidades que posee. Algunas de las estrategias que se recomiendan son las siguientes:

\begin{itemize}
	\item Realizar inversión intensiva dentro del marketing digital para captura rápida de los usuarios.
	
	\item Desarrollo acelerado de funcionalidades premium diferenciadas.
	
	\item Establecer alianzas estratégicas con instituciones educativas y empresas.
	
	\item Expansión territorial gradual aprovechando el modelo escalable de nuestra empresa.
	
	\item Capitalización de la alta demanda de la educación financiera en Bogotá.
\end{itemize}
