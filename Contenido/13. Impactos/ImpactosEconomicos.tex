\subsection*{Impactos económicos}
\addcontentsline{toc}{subsection}{Impactos económicos}

La implementación de una plataforma de planificación financiera hace múltiples impactos económicos de múltiples dimensiones que afectan tanto a nivel microeconómico (como usuarios individuales) como macroeconómicos (fintechs). A continuación revelamos el potencial transformador que tiene nuestro proyecto más allá del potencial comercial inmediato.

A nivel microeconómico, el impacto más directo es la mejora en la gestión financiera personal de los usuarios, y el consecuente aumento en la capacidad del ahorro, reducción de endeudamiento improductivo y la acumulación de activos. Al nosotros proporcionar herramientas para hacer presupuestos digitales con interfaces intuitivas, simuladores del ahorro y uso de calculadoras, la plataforma permite a sus usuarios a optimizar el flujo de efectivo, identificar posibles fugas dentro de sus presupuestos y priorizar realizar ciertas actividades económicas sobre otras.

Esta mejora logra traducir en un aumento del patrimonio neto de los usuarios a mediano y largo plazo. Inclusive mejoras leves en su comportamiento financiero pueden tener efectos positivos a largo plazo. Además, para el sector de trabajadores informales, la plataforma tiene un impacto económico considerable, ya que al permitir a este sector estabilizar su situación económica de forma gradual, construyendo un mejor historial financiero mediante el manejo responsable de los recursos, es posible dar la transición hacia la formalización económica y el acceso a servicios financieros formales.

A nivel macroeconómico, la plataforma fortalece y diversifica el ecosistema fintech colombiano. Complementando hacia la oferta de apps financieras tales como Nequi o Daviplata. Diversificando la oferta fintech hacia servicios de educación y planificación financiera y contribuyendo a la maduración del ecosistema.