\subsection*{Impactos sociales}
\addcontentsline{toc}{subsection}{Impactos sociales}

Dentro del ámbito social, lo que la plataforma ofrece es significativo y positivo para la comunidad bogotana. Ya que debido a su naturaleza de buscar educar y planificar financieramente, hacen que la población bogotana tenga el potencial de invertir y participar activamente en temas financieros. Especialmente en segmentos tradicionalmente excluidos por el sistema financiera. Además se democratiza el acceso hacia la educación financiera y se reduce la brecha de alfabetización financiera dada anteriormente que afecta de gran forma a la población colombiana.

La plataforma, al tener herramientas digitales didácticas como simuladores financieros, calculadoras y contenido educativo tiene el potencial de empoderar a estos segmentos, para tomar decisiones más informadas y responsables. Al mejorar la capacidad de estos grupos para entender temas como las tasas de interés, planes de amortización y otros, la plataforma incrementa la inclusión financiera efectiva. 

Junto a ello, se genera un impacto social positivo a la salud mental, ya que las personas lograrán tener un mejor bienestar financiero, que definimos anteriormente como el estado en el que una persona pueda cumplir sus obligaciones económicas presentes y futuras y se sienta seguro consigo misma. El estrés financiero debido a la inhabilidad de manejar gastos, acumular deudas, etc. Tienen un efecto dentro de la salud mentar y la productividad laboral. Al dar herramientas que permitan a los usuarios manejar su situación, establecer metas de ahorro, etc. Logra disminuir de gran manera la ansiedad financiera mejorando la percepción y el control sobre la propia vida económica.

Finalmente, se espera que, un impacto a largo plazo sea la transformación cultural hacia una mayor responsabilidad financiera y planificación intergeneracional. En la cuál es posible que los comportamientos financieros dados por el aplicativo sean transmitidos a sus hijos y creando un efecto multiplicador intergeneracional. En la cual se logren elaborar presupuestos y mantener una cultura del ahorro.