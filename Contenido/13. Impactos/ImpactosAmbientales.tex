\subsection*{Impactos ambientales}
\addcontentsline{toc}{subsection}{Impactos ambientales}

Dentro de los impactos ambientales, la plataforma será totalmente digital, lo que elimina la dependencia en los materiales físicos tales como papel, convirtiendo el negocio en una operación más sostenible, principalmente dentro del ámbito educativo, en la que la tecnología nos permite automatizar tales procesos y reduciendo el uso de documentos impresos. Además permite una comunicación directa entre clientes y el soporte reduciendo la cantidad y distancia de los desplazamientos y disminuyendo las emisiones de carbono relacionadas con temas de transporte.

Sin embargo, un efecto negativo será que, para el funcionamiento del aplicativo, los servidores y otras operaciones necesitarán de energía eléctrica, por lo que contribuye hacia la huella de carbono. Para ello, buscamos mitigarlo al usar servicios de hosting o centros de datos que tengan políticas de energía renovable, fomentando prácticas sostenibles tales como la reducción del papel y el uso de energía renovable.

Además, un impacto positivo indirecto es promover comportamientos de consumo más sostenibles, al educar al usuario sobre temas financieros y hacer mejores tomas de decisiones, la plataforma indirectamente influye en los patrones de consumo, de esta forma logrando que tengan menos impacto ambiental. Por ejemplo, usuarios que aprendan a planificar compras a comparación de hacer compras impulsivas y a evaluar en que forma económica afecta la compra a su presupuesto, todas estas acciones hacen la tarea de consumir en una mucho más reflexiva y menos derrochadora.